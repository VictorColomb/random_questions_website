\documentclass[a4paper]{article}
\usepackage[T1]{fontenc}
\usepackage[utf8]{inputenc}
\usepackage{lmodern}
\usepackage{amsmath,amssymb}
\usepackage[top=3cm,bottom=2cm,left=2cm,right=2cm]{geometry}
\usepackage{fancyhdr}
\usepackage{esvect,esint}
\usepackage{xcolor}
\usepackage{tikz}\usetikzlibrary{calc}

\parskip1em\parindent0pt\let\ds\displaystyle

\begin{document}

\pagestyle{fancy}
\fancyhf{}
\setlength{\headheight}{15pt}
\fancyhead[L]{Mécanique}\fancyhead[R]{Question 4}

% Énoncé
\begin{center}
	\large{\boldmath{\textbf{Expressions de $\vec{\mathrm{grad}}\,(fg)$, $\mathrm{div}\,(f\vec{u})$, \\ $\vec{\mathrm{rot}}\,(f\vec{u})$ \\ et de $\mathrm{div}\,(\vec{A} \wedge \vec{B})$}}}
\end{center}

% Correction

\setlength{\tabcolsep}{10pt} 
\renewcommand{\arraystretch}{2.0}
Soient \(f,g\) des fonctions de l'espace et \(\vv{u},\vv{A},\vv{B}\) des fonctions vectorielles de l'espace. On a :
\begin{center}
\fcolorbox{red}{white}{
\(\begin{array}{c}
\vv{\mathrm{grad}}(fg)=f\vv{\mathrm{grad}}g+g\vv{\mathrm{grad}}f\\
\mathrm{div}(f\vv{u}=(\vv{\mathrm{grad}}f)\cdot\vv{u}+f\mathrm{div}\vv{u}\\
\vv{\mathrm{rot}}(f\vv{u})=(\vv{\mathrm{grad}}f)\wedge\vv{u}+f\vv{\mathrm{rot}}\vv{u}\\
\mathrm{div}(\vv{A}\wedge\vv{B})=(\vv{\mathrm{rot}}\vv{A})\cdot\vv{B}-(\vv{\mathrm{rot}}\vv{B})\cdot\vv{A}
\end{array}\)
}
\end{center}


\end{document}
