\documentclass[a4paper]{article}
\usepackage[T1]{fontenc}
\usepackage[utf8]{inputenc}
\usepackage{lmodern}
\usepackage{amsmath,amssymb}
\usepackage[top=3cm,bottom=2cm,left=2cm,right=2cm]{geometry}
\usepackage{fancyhdr}
\usepackage{esvect,esint}
\usepackage{xcolor}
\usepackage{tikz}\usetikzlibrary{calc}

\parskip 1em\parindent 0pt

\begin{document}

\pagestyle{fancy}
\fancyhf{}
\setlength{\headheight}{15pt}
\fancyhead[L]{Mécanique}\fancyhead[R]{Question 2}

% Énoncé
\begin{center}
	\large{\boldmath{\textbf{Théorèmes d’Ostrogradski, de Stokes et du gradient}}}
\end{center}

% Correction

Théorème d'Ostrogradski : \\
Soit un volume $V$ entouré d'une surface fermée $S$ orientée vers l'extérieur.
\begin{center}
	\fcolorbox{red}{white}{\(\displaystyle\iiint\limits_V\mathrm{div}\vv{u}\,\mathrm{d}V=\displaystyle\oiint\limits_{S}\vv{u}\cdot\vv{\mathrm{d}S}\)}
\end{center}

Théorème de Stokes : \\
Soit une surface $S$ reposant sur un contour fermé $\Gamma$ dont les orientations correspondent.
\begin{center}
	\fcolorbox{red}{white}{\(\displaystyle\iint\limits_S\vv{\mathrm{rot}}\vv{u}\,\mathrm{d}S=\displaystyle\int\limits_\Gamma \vv{u}\,\mathrm{d}\ell\)}
\end{center}

Théorème du gradient : \\
Soit un volume $V$ entouré par une surface fermée $S$ orientée vers l'extérieur.
\begin{center}
	\fcolorbox{red}{white}{\(\displaystyle\iiint\limits_V\vv{\mathrm{grad}}u\,\mathrm{d}V= \displaystyle\oiint\limits_{S}u\,\vv{\mathrm{d}S}\)}
\end{center}

\end{document}
