\documentclass[a4paper]{article}
\usepackage[T1]{fontenc}
\usepackage[utf8]{inputenc}
\usepackage{lmodern}
\usepackage{amsmath,amssymb}
\usepackage[top=3cm,bottom=2cm,left=2cm,right=2cm]{geometry}
\usepackage{fancyhdr}
\usepackage{esvect,esint}
\usepackage{xcolor}
\usepackage{tikz}\usetikzlibrary{calc}

\parskip1em\parindent0pt\let\ds\displaystyle

\begin{document}

\pagestyle{fancy}
\fancyhf{}
\setlength{\headheight}{15pt}
\fancyhead[L]{Chimie}\fancyhead[R]{Question 46}

% Énoncé
\begin{center}
	\large{\boldmath{\textbf{Définition rendement faradique d’une électrolyse}}}
\end{center}

% Correction

Lors d'une électrolyse, certaines réactions parasites peuvent empêcher que la réaction d'oxydo-réduction souhaitée soit la seule.\\
On définit alors le rendement faradique comme la quantité d'espèce souhaitée obtenue quotientée par la quantité théorique pouvant être obtenue.

\end{document}
