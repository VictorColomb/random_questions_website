\documentclass[a4paper]{article}
\usepackage[T1]{fontenc}
\usepackage[utf8]{inputenc}
\usepackage{lmodern}
\usepackage{amsmath,amssymb}
\usepackage[top=3cm,bottom=2cm,left=2cm,right=2cm]{geometry}
\usepackage{fancyhdr}
\usepackage{esvect,esint}
\usepackage{xcolor}
\usepackage{tikz}\usetikzlibrary{calc}

\parskip1em\parindent0pt\let\ds\displaystyle

\begin{document}

\pagestyle{fancy}
\fancyhf{}
\setlength{\headheight}{15pt}
\fancyhead[L]{Mécanique}\fancyhead[R]{Question 24}

% Énoncé
\begin{center}
	\large{\boldmath{\textbf{Définition quantité de mouvement, moment cinétique et énergie cinétique \\ d’un système de points matériels dans un référentiel}}}
\end{center}

% Correction

Soit un système \(S\) de points matériels \(i\) de masse \(m_i\) en mouvement dans un référentiel \(\mathcal{R}\).\\
On note \(C\) le centre d'inertie de \(S\) et \(m=\displaystyle\sum\limits_{i\in S}m_i\) sa masse totale.\\
On définit la quantité de mouvement du système par :\begin{center}\fcolorbox{red}{white}{
\(\vv{p}(S\in\mathcal{R})=\displaystyle\sum\limits_{i\in S}m_i\vv{v}(i\in\mathcal{R})=m\vv{v}(C\in\mathcal{R})\)}\end{center}
On définit le moment cinétique en \(O\) du système par :\begin{center}\fcolorbox{red}{white}{
\(\sigma(O,S\in\mathcal{R})=\displaystyle\sum\limits_{i\in S}\vv{OM_i}\wedge m_i\vv{v}(i\in\mathcal{R})\)}\end{center}
On définit l'énergie cinétique du système par :\begin{center}\fcolorbox{red}{white}{
\(E_c^{\mathcal{R}}=\displaystyle\sum\limits_{i\in S}\dfrac{1}{2}m_i\vv{v}(i\in\mathcal{R})^2\)}\end{center}

\end{document}
