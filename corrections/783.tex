\documentclass[a4paper]{article}
\usepackage[T1]{fontenc}
\usepackage[utf8]{inputenc}
\usepackage{lmodern}
\usepackage{amsmath,amssymb}
\usepackage[top=3cm,bottom=2cm,left=2cm,right=2cm]{geometry}
\usepackage{fancyhdr}
\usepackage{esvect,esint}
\usepackage{xcolor}
\usepackage{tikz}\usetikzlibrary{calc}

\parskip1em\parindent0pt\let\ds\displaystyle

\begin{document}

\pagestyle{fancy}
\fancyhf{}
\setlength{\headheight}{15pt}
\fancyhead[L]{Chimie}\fancyhead[R]{Question 39}

% Énoncé
\begin{center}
	\large{\boldmath{\textbf{Principe expérimental du relevé \\ d’une courbe intensité-potentiel}}}
\end{center}

% Correction

Le dispositif comporte trois électrodes :\begin{itemize}
\item
l’électrode étudiée dite électrode de travail notée \(T\)
\item 
l’électrode de référence au calomel saturé notée \(ECS\)
\item
l’électrode auxiliaire notée \(A\) en platine
\end{itemize}
Le courant \(I\) rentrant dans l’électrode de travail est algébrisé selon la convention habituelle.\\
On dispose d’un voltmètre qui mesure \(V = E_T - E_{ECS} = E_T - 0, 25 V\), d’un générateur qui impose la tension variable \(U = E_A - E_T\) et un ampèremètre qui mesure \(I\) .\\
Organisation de l’expérience :\\
On se donne \(U\) , on lit alors l’indication du voltmètre \(V\) soit \(E_T\) et celle de l’ampèremètre \(I\) ce
qui permet de tracer point par point \(I = f (E_T )\), c'est-à-dire la courbe intensité potentiel de l’électrode de travail.

\end{document}
