\documentclass[a4paper]{article}
\usepackage[T1]{fontenc}
\usepackage[utf8]{inputenc}
\usepackage{lmodern}
\usepackage{amsmath,amssymb}
\usepackage[top=3cm,bottom=2cm,left=2cm,right=2cm]{geometry}
\usepackage{fancyhdr}
\usepackage{esvect,esint}
\usepackage{xcolor}
\usepackage{tikz}\usetikzlibrary{calc}

\parskip1em\parindent0pt\let\ds\displaystyle

\begin{document}

\pagestyle{fancy}
\fancyhf{}
\setlength{\headheight}{15pt}
\fancyhead[L]{Mécanique}\fancyhead[R]{Question 19}

% Énoncé
\begin{center}
	\large{\boldmath{\textbf{Définition énergie mécanique dans un repère}}}
\end{center}

% Correction

Soit \(\mathcal{R}\) un repère quelconque. \\
On définit l'énergie mécanique d'un corps dans ce référentiel par la somme de ses énergies cinétique et potentielles dans ce référentiel :
\begin{center}
\fcolorbox{red}{white}{\(E_m^\mathcal{R}=E_c^\mathcal{R}+E_p^\mathcal{R}\)}
\end{center}

\end{document}
