\documentclass[a4paper]{article}
\usepackage[T1]{fontenc}
\usepackage[utf8]{inputenc}
\usepackage{lmodern}
\usepackage{amsmath,amssymb}
\usepackage[top=3cm,bottom=2cm,left=2cm,right=2cm]{geometry}
\usepackage{fancyhdr}
\usepackage{esvect,esint}
\usepackage{xcolor}
\usepackage{tikz,circuitikz}\usetikzlibrary{calc}

\parskip1em\parindent0pt\let\ds\displaystyle

\begin{document}

\pagestyle{fancy}
\fancyhf{}
\setlength{\headheight}{15pt}
\fancyhead[L]{Electrocinétique}\fancyhead[R]{Question 35}

% Énoncé
\begin{center}
	\large{\boldmath{\textbf{Montage dérivateur sur série de Fourier}}}
\end{center}

% Correction

Soit \(e\) un signal périodique de fréquence \(f\) dont on note \(e(t)=e_0+\sum e_n\cos(2\pi nft +\varphi_n)\) la décomposition en série de Fourier.\\
Soit \(f_c\) la fréquence de coupure du filtre qu'on suppose passe-haut.

Si \(f\ll f_c\), alors le signal en sortie vaut \begin{center}\fcolorbox{red}{white}{\(s(t)=-\dfrac{1}{f_c}\ds\sum\limits_{n\geq1}2\pi nfe_n\sin(2\pi nft +\varphi_n)\simeq\dfrac{1}{f_c}\dfrac{\mathrm{d}e}{\mathrm{d}t}\)}\end{center}
Si \(f\gg f_c\), alors le signal en sortie vaut \begin{center}\fcolorbox{red}{white}{\(s(t)=e(t)-\langle e\rangle\)}\end{center}

\end{document}
