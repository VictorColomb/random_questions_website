\documentclass[a4paper]{article}
\usepackage[T1]{fontenc}
\usepackage[utf8]{inputenc}
\usepackage{lmodern}
\usepackage{amsmath,amssymb}
\usepackage[top=3cm,bottom=2cm,left=2cm,right=2cm]{geometry}
\usepackage{fancyhdr}
\usepackage{esvect}
\usepackage{xcolor}
\usepackage{tikz}\usetikzlibrary{calc}

\parskip 1em\parindent 0pt

\begin{document}

\pagestyle{fancy}
\fancyhf{}
\setlength{\headheight}{15pt}
\fancyhead[L]{Mécanique}\fancyhead[R]{Question 14}

% Énoncé
\begin{center}
	\large{\boldmath{\textbf{Définition et théorème du moment cinétique}}}
\end{center}

% Correction

On s'intéresse au moment du point \(M\) en \(O\) dans le référentiel galiléen \(\mathcal{R}_g\). \\
On définit le moment cinétique par :
\begin{center}
	\fcolorbox{red}{white}{\( \vv{\sigma}(M\in\mathcal{R},O)=\vv{OM}\wedge \vv p (M\in\mathcal{R}_g) \)}
\end{center}

Théorème du moment cinétique :
\begin{center}
	\fcolorbox{red}{white}{\( \left. \dfrac{\mathrm{d}\vv{\sigma}}{\mathrm{d}t} \right|_{\mathcal{R}_g} = \displaystyle\sum\vv{OM}\wedge \vv{F_{ext}}(M\in\mathcal{R}_g)\)}
\end{center}


\end{document}
