\documentclass[a4paper]{article}
\usepackage[T1]{fontenc}
\usepackage[utf8]{inputenc}
\usepackage{lmodern}
\usepackage{amsmath,amssymb}
\usepackage[top=3cm,bottom=2cm,left=2cm,right=2cm]{geometry}
\usepackage{fancyhdr}
\usepackage{esvect,esint}
\usepackage{xcolor}
\usepackage{tikz}\usetikzlibrary{calc}

\parskip1em\parindent0pt\let\ds\displaystyle

\begin{document}

\pagestyle{fancy}
\fancyhf{}
\setlength{\headheight}{15pt}
\fancyhead[L]{Electromagnétisme}\fancyhead[R]{Question 7}

% Énoncé
\begin{center}
	\large{\boldmath{\textbf{Définition du moment dipolaire électrique}}}
\end{center}

% Correction

Soit une distribution de charges ponctuelles \(q_i\) placées en \(P_i\) telles que \(\sum_iq_i=0\).\\
On définit le moment dipolaire \(\vv{p}\) associé à cette distribution par :
\begin{center}
\fcolorbox{red}{white}{\(\vv{p}=\sum_iq_i\vv{OP_i}\)}
\end{center}
Le moment dipolaire est indépendant du point fixe \(O\) choisi.

\end{document}
