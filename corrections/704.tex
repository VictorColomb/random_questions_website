\documentclass[a4paper]{article}
\usepackage[T1]{fontenc}
\usepackage[utf8]{inputenc}
\usepackage{lmodern}
\usepackage{amsmath,amssymb}
\usepackage[top=3cm,bottom=2cm,left=2cm,right=2cm]{geometry}
\usepackage{fancyhdr}
\usepackage{esvect,esint}
\usepackage{xcolor}
\usepackage{tikz}\usetikzlibrary{calc}

\parskip1em\parindent0pt\let\ds\displaystyle

\begin{document}

\pagestyle{fancy}
\fancyhf{}
\setlength{\headheight}{15pt}
\fancyhead[L]{Thermodynamique}\fancyhead[R]{Question 14}

% Énoncé
\begin{center}
	\large{\boldmath{\textbf{Valeur de $\gamma$ et de $c_{v,n}$ pour un gaz monoatomique, diatomique}}}
\end{center}

% Correction

Pour un gaz parfait monoatomique, on a \fcolorbox{red}{white}{\(\gamma=\dfrac{5}{3}\)} et \fcolorbox{red}{white}{\(c_{v,n}=\dfrac{3}{2}R\)}\\
Pour un gaz parfait diatomique, on a \fcolorbox{red}{white}{\(\gamma=\dfrac{7}{5}\)} et \fcolorbox{red}{white}{\(c_{v,n}=\dfrac{5}{2}R\)}



\end{document}
