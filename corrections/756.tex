\documentclass[a4paper]{article}
\usepackage[T1]{fontenc}
\usepackage[utf8]{inputenc}
\usepackage{lmodern}
\usepackage{amsmath,amssymb}
\usepackage[top=3cm,bottom=2cm,left=2cm,right=2cm]{geometry}
\usepackage{fancyhdr}
\usepackage{esvect}
\usepackage{xcolor}
\usepackage{tikz}\usetikzlibrary{calc}

\parskip 1em\parindent 0pt

\begin{document}

\pagestyle{fancy}
\fancyhf{}
\setlength{\headheight}{15pt}
\fancyhead[L]{Chimie}\fancyhead[R]{Question 12}

% Énoncé
\begin{center}
	\large{\boldmath{\textbf{Relation de Clapeyron}}}
\end{center}

% Correction

\fcolorbox{red}{white}{\(L_{a\leftrightarrow b}=T\dfrac{\mathrm{d}P_{a\leftrightarrow b}}{\mathrm{d}T}(v_b-v_a)\)}\\
où \(v_b\) et \(v_a\) désignent des volumes massiques.


\end{document}
