\documentclass[a4paper]{article}
\usepackage[T1]{fontenc}
\usepackage[utf8]{inputenc}
\usepackage{lmodern}
\usepackage{amsmath,amssymb}
\usepackage[top=3cm,bottom=2cm,left=2cm,right=2cm]{geometry}
\usepackage{fancyhdr}
\usepackage{esvect,esint}
\usepackage{xcolor}
\usepackage{tikz,circuitikz}\usetikzlibrary{calc}

\parskip1em\parindent0pt\let\ds\displaystyle

\begin{document}

\pagestyle{fancy}
\fancyhf{}
\setlength{\headheight}{15pt}
\fancyhead[L]{Electrocinétique}\fancyhead[R]{Question 15}

% Énoncé
\begin{center}
	\large{\boldmath{\textbf{HP: Théorème de Millman}}}
\end{center}

% Correction


Le réseau doit être muni d'une masse qui définit le zéro des potentiels pour que ce théorème puisse s'appliquer. \vspace{0.3cm}\\
Dans le cas le plus complexe, on s'intéresse à un noeud auquel sont connectés \(n\) dipôles de la forme :\begin{center}
\begin{minipage}{0.4\linewidth}
  \begin{circuitikz}
    \draw (0,0) -- ++(.5,0) node[above] {$j$} to[short,*-,i=$i_{jk}$] ++(1,0) -- ++(0,1) to[R=$R_{jk}$] ++(1.5,0) to[vsource=$E_{jk}$] ++(1.5,0) -- ++(0,-1) to[short,-*] ++(1,0) node[above] {$k$};
    \draw (1.5,0) -- ++(0,-1) to[isource=$I_{jk}$] ++(3,0) -- ++(0,1);
  \end{circuitikz}
\end{minipage}
\end{center}
Alors 
  $i_{jk}=\dfrac{V_k-V_j-E_{jk}}{R_{jk}}+I_{jk}$\\
  Par loi des noeuds : $\displaystyle\sum\limits_ji_{jk}=0$
\\


  Cas particulier :
  $\forall j,I_{jk}=0; E_{jk}=0$\\
  Alors \begin{center}
    \fcolorbox{red}{white}{
    $V_k=\dfrac{\displaystyle\sum\limits_jV_jG_{jk}}{\displaystyle\sum\limits_jG_{jk}}$
    }\end{center}


\end{document}
