\documentclass[a4paper]{article}
\usepackage[T1]{fontenc}
\usepackage[utf8]{inputenc}
\usepackage{lmodern}
\usepackage{amsmath,amssymb}
\usepackage[top=3cm,bottom=2cm,left=2cm,right=2cm]{geometry}
\usepackage{fancyhdr}
\usepackage{esvect,esint}
\usepackage{xcolor}
\usepackage{tikz,circuitikz}\usetikzlibrary{calc}

\parskip1em\parindent0pt\let\ds\displaystyle

\begin{document}

\pagestyle{fancy}
\fancyhf{}
\setlength{\headheight}{15pt}
\fancyhead[L]{Electrocinétique}\fancyhead[R]{Question 31}

% Énoncé
\begin{center}
	\large{\boldmath{\textbf{Théorème de développement en série de Fourier}}}
\end{center}

% Correction

Soit \(g\) une fonction continue par morceaux périodique de période \(T\) et de fréquence \(f=\dfrac{1}{T}\).\\
Alors \(g\) possède un développement unique dit en série de Fourier donné par :
\begin{center}\fcolorbox{red}{white}{\(\forall t \in \mathbb{R},g(t)=a_0+\sum_n(a_n\cos(2\pi nft)+b_n\sin(2\pi nft))\)}\end{center}
Ce développement vaut \(g(t)\) en tout point \(t\) où \(g\) est continue et \(\dfrac{g(t^-)+g(t^+)}{2}\) en tout point \(t\) où \(g\) est discontinue.

\emph{HP :} Les coefficients de ce développement sont donnés par \begin{center}\(a_0=\dfrac{1}{T}\ds\int_t^{t+T}g(t)\mathrm{d}t=\langle g \rangle\)\\
\(\forall n \in \mathbb{N}^*,a_n=\dfrac{2}{T}\ds\int_t^{t+T}g(t)\cos(2\pi nft)\mathrm{d}t\)\\
\(\forall n \in \mathbb{N}^*,b_n=\dfrac{2}{T}\ds\int_t^{t+T}g(t)\sin(2\pi nft)\mathrm{d}t\)\end{center}



\end{document}
