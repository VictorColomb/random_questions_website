\documentclass[a4paper]{article}
\usepackage[T1]{fontenc}
\usepackage[utf8]{inputenc}
\usepackage{lmodern}
\usepackage{amsmath,amssymb}
\usepackage[top=3cm,bottom=2cm,left=2cm,right=2cm]{geometry}
\usepackage{fancyhdr}
\usepackage{esvect}
\usepackage{xcolor}
\usepackage{tikz,circuitikz}\usetikzlibrary{calc}

\parskip 1em\parindent 0pt

\begin{document}

\pagestyle{fancy}
\fancyhf{}
\setlength{\headheight}{15pt}
\fancyhead[L]{Electrocinétique}\fancyhead[R]{Question 18}

% Énoncé
\begin{center}
	\large{\boldmath{\textbf{Définition de la stabilité, de la stabilité large et de l’instabilité d’un système. \\ Condition sur les pôles de la fonction de transfert}}}
\end{center}

% Correction

On note \(\underline{H}=\dfrac{\sum\limits_{i}p_i(i\omega)^{i}}{\sum\limits_{j}q_j(i\omega)^j}\) la fonction de transfert et \((S):\sum\limits_{i}p_i\dfrac{\mathrm{d}^ie}{\mathrm{d}t^i}=\sum\limits_{j}q_j\dfrac{\mathrm{d}^js}{\mathrm{d}t^j}\) le système différentiel associé.

Il y a stabilité si la solution libre du système \((S)\) tend vers \(0\), soit si tous les pôles \(q_i\) sont à partie réelle strictement négative.\\
Il y a stabilité large si la solution libre du système \((S)\) est bornée, soit si tous les pôles \(q_i\) sont à partie réelle négative.\\
Il y a instabilité si la solution libre du système \((S)\) n'est pas bornée, soit si au moins un pôle \(q_i\) est à partie réelle strictement positive.


\end{document}
