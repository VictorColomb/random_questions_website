\documentclass[a4paper]{article}
\usepackage[T1]{fontenc}
\usepackage[utf8]{inputenc}
\usepackage{lmodern}
\usepackage{amsmath,amssymb}
\usepackage[top=3cm,bottom=2cm,left=2cm,right=2cm]{geometry}
\usepackage{fancyhdr}
\usepackage{esvect,esint}
\usepackage{xcolor}
\usepackage{tikz}\usetikzlibrary{calc}

\parskip1em\parindent0pt\let\ds\displaystyle

\begin{document}

\pagestyle{fancy}
\fancyhf{}
\setlength{\headheight}{15pt}
\fancyhead[L]{Mécanique}\fancyhead[R]{Question 10}

% Énoncé
\begin{center}
	\large{\boldmath{\textbf{Principe fondamental de la dynamique}}}
\end{center}

% Correction

On considère un système \( \Sigma \) dans un référentiel galiléen \( \mathcal R \). \\
Alors
\begin{center}
\fcolorbox{red}{white}{
	\( \{ \mathcal D_{\Sigma / \mathcal R} \} = \{ \mathcal T_{\mathrm{ext}\rightarrow\Sigma} \} \)
}\end{center}
C'est à dire que \( \left\{ \begin{array}{@{}l} m_\Sigma \vv a(G_\Sigma \in\Sigma / \mathcal R) = \vv F(\mathrm{ext}\rightarrow\Sigma) \\[3pt] \vv\delta(A,\Sigma / \mathcal R) = \vv M(A, \mathrm{ext}\rightarrow\Sigma) \end{array}\right. \)


\end{document}
