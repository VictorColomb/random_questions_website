\documentclass[a4paper]{article}
\usepackage[T1]{fontenc}
\usepackage[utf8]{inputenc}
\usepackage{lmodern}
\usepackage{amsmath,amssymb}
\usepackage[top=3cm,bottom=2cm,left=2cm,right=2cm]{geometry}
\usepackage{fancyhdr}
\usepackage{esvect,makecell}
\usepackage{xcolor}
\usepackage{tikz}\usetikzlibrary{calc}

\parskip 1em\parindent 0pt

\begin{document}

\pagestyle{fancy}
\fancyhf{}
\setlength{\headheight}{15pt}
\fancyhead[L]{Chimie}\fancyhead[R]{Question 10}

% Énoncé
\begin{center}
	\large{\boldmath{\textbf{Activités d’un gaz parfait, d’un mélange de gaz parfaits, \\ d’un mélange idéal de solide ou de liquide, d’un solide \\ ou d’un liquide pur, d’un solvant, d’un soluté}}}
\end{center}

% Correction

\begin{table}[h]
\begin{tabular}{l|c}
& Activité \\ \hline Gaz parfait & \gape{\(a=\dfrac{P}{1 bar}\)} \\ Mélange de gaz parfaits & \gape{\(a=\dfrac{P_i}{1 bar}\)} \\ Mélange de solides ou de liquides & \(a=x_i\) \\ Solide ou liquide pur & \(a=1\) \\ Solvant & \(a=1\) \\ Soluté & \gape{\(a=\dfrac{c_i}{1 \mathrm{mol}.\mathrm{L}^{-1}}\)}
\end{tabular}
\end{table}

\end{document}
