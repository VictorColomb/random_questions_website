\documentclass[a4paper]{article}
\usepackage[T1]{fontenc}
\usepackage[utf8]{inputenc}
\usepackage{lmodern}
\usepackage{amsmath,amssymb}
\usepackage[top=3cm,bottom=2cm,left=2cm,right=2cm]{geometry}
\usepackage{fancyhdr}
\usepackage{esvect,esint}
\usepackage{xcolor}
\usepackage{tikz}\usetikzlibrary{calc}

\parskip1em\parindent0pt\let\ds\displaystyle

\begin{document}

\pagestyle{fancy}
\fancyhf{}
\setlength{\headheight}{15pt}
\fancyhead[L]{Mécanique}\fancyhead[R]{Question 13}

% Énoncé
\begin{center}
	\large{\boldmath{\textbf{Puissance intérieure ou inter-efforts}}}
\end{center}

% Correction

\abovedisplayskip0pt

On considère deux solides \( S_1 \) et \( S_2 \). \\
La puissance des inter-efforts entre \( S_1 \) et \( S_2 \) est :
\begin{center}
\fcolorbox{red}{white}{
	\begin{minipage}{.62\linewidth}
		\begin{align*}
			P(S_1 \leftrightarrow S_2) =&\ \{\mathcal{T}_{1\rightarrow2}\} \otimes \{\mathcal{V}_{2/1}\} \\
			=&\ \vv F(S_1\rightarrow S_2) \cdot \vv V(A\in S_2/S_1) + \vv M(A, S_1\rightarrow S_2) \cdot \vv\Omega(S_2/S_1)
		\end{align*}
	\end{minipage}
}\end{center}

Pour un ensemble de solides \( \Sigma = (S_i) \), \[ P(\mathrm{int}, \Sigma) = \sum_{i\neq j} P(S_i \leftrightarrow S_j) \]

\end{document}
