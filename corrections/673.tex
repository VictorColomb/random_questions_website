\documentclass[a4paper]{article}
\usepackage[T1]{fontenc}
\usepackage[utf8]{inputenc}
\usepackage{lmodern}
\usepackage{amsmath,amssymb}
\usepackage[top=3cm,bottom=2cm,left=2cm,right=2cm]{geometry}
\usepackage{fancyhdr}
\usepackage{esvect,esint}
\usepackage{xcolor}
\usepackage{tikz}\usetikzlibrary{calc}

\parskip1em\parindent0pt\let\ds\displaystyle

\begin{document}

\pagestyle{fancy}
\fancyhf{}
\setlength{\headheight}{15pt}
\fancyhead[L]{Electromagnétisme}\fancyhead[R]{Question 58}

% Énoncé
\begin{center}
	\large{\boldmath{\textbf{Relation de dispersion d’un plasma non collisionnel}}}
\end{center}

% Correction

Pour un plasma de pulsation de plasma \(\omega_p\), on a :\begin{center}
\fcolorbox{red}{white}{\(k^2=\dfrac{\omega^2-\omega_p^2}{c^2}\)}\end{center}


\end{document}
