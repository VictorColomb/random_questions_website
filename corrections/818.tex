\documentclass[a4paper]{article}
\usepackage[T1]{fontenc}
\usepackage[utf8]{inputenc}
\usepackage{lmodern}
\usepackage{amsmath,amssymb}
\usepackage[top=3cm,bottom=2cm,left=2cm,right=2cm]{geometry}
\usepackage{fancyhdr}
\usepackage{esvect,esint}
\usepackage{xcolor,makecell}
\usepackage{tikz}\usetikzlibrary{calc}

\parskip1em\parindent0pt\let\ds\displaystyle

\begin{document}

\pagestyle{fancy}
\fancyhf{}
\setlength{\headheight}{15pt}
\fancyhead[L]{SLCI}\fancyhead[R]{Question 3}

% Énoncé
\begin{center}
	\large{\boldmath{\textbf{Transformée de Laplace de l'impulsion de Dirac, \\ d'un échelon, d'une fonction affine \\ et d'une exponentielle décroissante}}}
\end{center}

% Correction

\begin{table}[h]
	\begin{tabular}{c|c}
		\( x(t) \) & \( X(p) \) \\
		\hline
		Impulsion de Dirac & \( 1 \) \\
		\( e_0 u(t) \) & \gape{\( \dfrac{e_0}{p} \)} \\
		\( at\, u(t) \) & \gape{\( \dfrac{a}{p^2} \)} \\
		\( e^{-\alpha t} \) & \gape{\( \dfrac{1}{p+\alpha} \)}
	\end{tabular}
\end{table}


\end{document}
