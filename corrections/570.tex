\documentclass[a4paper]{article}
\usepackage[T1]{fontenc}
\usepackage[utf8]{inputenc}
\usepackage{lmodern}
\usepackage{amsmath,amssymb}
\usepackage[top=3cm,bottom=2cm,left=2cm,right=2cm]{geometry}
\usepackage{fancyhdr}
\usepackage{esvect,esint}
\usepackage{xcolor}
\usepackage{tikz,circuitikz}\usetikzlibrary{calc}

\parskip1em\parindent0pt\let\ds\displaystyle

\begin{document}

\pagestyle{fancy}
\fancyhf{}
\setlength{\headheight}{15pt}
\fancyhead[L]{Electrocinétique}\fancyhead[R]{Question 22}

% Énoncé
\begin{center}
	\large{\boldmath{\textbf{Définition puissance complexe électrocinétique, \\ lien avec la puissance moyenne}}}
\end{center}

% Correction

La puissance complexe électrocinétique est définie par
\begin{center}
\fcolorbox{red}{white}{\(\underline{P}=\underline{u}\underline{i}^*\)}
\end{center}
Elle est liée à la puissance électrocinétique moyenne par
\begin{center}
\fcolorbox{red}{white}{\(P_m=\dfrac{1}{2}\text{Re}\underline{P}\)}
\end{center}


\end{document}
