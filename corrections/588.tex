\documentclass[a4paper]{article}
\usepackage[T1]{fontenc}
\usepackage[utf8]{inputenc}
\usepackage{lmodern}
\usepackage{amsmath,amssymb}
\usepackage[top=3cm,bottom=2cm,left=2cm,right=2cm]{geometry}
\usepackage{fancyhdr}
\usepackage{esvect,esint}
\usepackage{xcolor}
\usepackage{tikz}\usetikzlibrary{calc}

\parskip1em\parindent0pt\let\ds\displaystyle

\begin{document}

\pagestyle{fancy}
\fancyhf{}
\setlength{\headheight}{15pt}
\fancyhead[L]{Mécanique}\fancyhead[R]{Question 1}

% Énoncé
\begin{center}
	\large{\boldmath{\textbf{Définition nabla, gradient, divergence, rotationnel,\\ laplacien scalaire, laplacien vectoriel}}}
\end{center}

% Correction

\setlength{\tabcolsep}{12pt}
\renewcommand{\arraystretch}{2.0}
Tous les vecteurs sont en coordonnées cartésiennes \((x,y,z)\), soit \(A\) un scalaire et soit \(\vv{u}=\left|\begin{array}{l} u_x\\ u_y\\ u_z\end{array}\right.\) un vecteur.\\
On définit l'opérateur nabla \(\vv{\nabla}\) par :\begin{center}\fcolorbox{red}{white}{
\(\vv{\nabla}=\left|\begin{array}{l} \dfrac{\partial}{\partial x}\\\dfrac{\partial}{\partial y}\\\dfrac{\partial}{\partial z}\end{array}\right.\)}\end{center}
On définit l'opérateur scalaire gradient \(\vv{\mathrm{grad}}\) par :\begin{center}\fcolorbox{red}{white}{
\(\vv{\mathrm{grad}}A=\vv{\nabla}A=\left|\begin{array}{l}\dfrac{\partial A}{\partial x}\\\dfrac{\partial A}{\partial y}\\\dfrac{\partial A}{\partial z}\end{array}\right.\)}\end{center}
On définit l'opérateur vectoriel divergence \(\mathrm{div}\) par :\begin{center}\fcolorbox{red}{white}{
\(\mathrm{div}\vv{u}=\vv{\nabla}\cdot\vv{u}=\dfrac{\partial u_x}{\partial x}+\dfrac{\partial u_y}{\partial y}+\dfrac{\partial u_z}{\partial z}\)}\end{center}
On définit l'opérateur vectoriel rotationnel \(\vv{\mathrm{rot}}\) par :\begin{center}\fcolorbox{red}{white}{
\(\vv{\mathrm{rot}}\vv{u}=\vv{\nabla}\wedge\vv{u}=\left|\begin{array}{l}\dfrac{\partial u_z}{\partial y}-\dfrac{\partial u_y}{\partial z}\\\dfrac{\partial u_x}{\partial z}-\dfrac{\partial u_z}{\partial x}\\\dfrac{\partial u_y}{\partial x}-\dfrac{\partial u_x}{\partial y}\end{array}\right.\)}\end{center}
On définit l'opérateur scalaire laplacien \(\Delta\) par :\begin{center}\fcolorbox{red}{white}{
\(\Delta A =\vv{\nabla}\cdot(\vv{\nabla}A)= \mathrm{div}(\vv{\mathrm{grad}}A)=\dfrac{\partial ^2u_x}{\partial x^2}+\dfrac{\partial ^2u_y}{\partial y^2}+\dfrac{\partial^2 u_z}{\partial z^2}\)}\end{center}
On définit l'opérateur vectoriel laplacien vectoriel \(\vv{\Delta}\) par :\begin{center}\fcolorbox{red}{white}{
\(\vv{\Delta}\vv{u}=\left|\begin{array}{l}\Delta u_x\\ \Delta u_y\\ \Delta u_z\end{array}\right.\)}\end{center}

\end{document}
