\documentclass[a4paper]{article}
\usepackage[T1]{fontenc}
\usepackage[utf8]{inputenc}
\usepackage{lmodern}
\usepackage{amsmath,amssymb}
\usepackage[top=3cm,bottom=2cm,left=2cm,right=2cm]{geometry}
\usepackage{fancyhdr}
\usepackage{esvect,esint}
\usepackage{xcolor}
\usepackage{tikz}\usetikzlibrary{calc}

\parskip1em\parindent0pt\let\ds\displaystyle

\begin{document}

\pagestyle{fancy}
\fancyhf{}
\setlength{\headheight}{15pt}
\fancyhead[L]{Mécanique}\fancyhead[R]{Question 28}

% Énoncé
\begin{center}
	\large{\boldmath{\textbf{Théorème du moment d’inertie}}}
\end{center}

% Correction

Pour un solide en rotation autour d'un axe \(O\vv{z}\). \\
On note \(\theta\) l'angle dont a tourné le solide autour de son axe.\\
Soit \(J_z\) le moment d'inertie du solide autour de son axe.\\
On a alors :
\begin{center}\fcolorbox{red}{white}{\(J_z\ddot\theta=\displaystyle\sum\vv{M}(O,\vv{F_{\mathrm{ext}}}) \cdot \vv z\)}\end{center}


\end{document}
