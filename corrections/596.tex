\documentclass[a4paper]{article}
\usepackage[T1]{fontenc}
\usepackage[utf8]{inputenc}
\usepackage{lmodern}
\usepackage{amsmath,amssymb}
\usepackage[top=3cm,bottom=2cm,left=2cm,right=2cm]{geometry}
\usepackage{fancyhdr}
\usepackage{esvect}
\usepackage{xcolor}
\usepackage{tikz}\usetikzlibrary{calc}

\parskip 1em\parindent 0pt

\begin{document}

\pagestyle{fancy}
\fancyhf{}
\setlength{\headheight}{15pt}
\fancyhead[L]{Mécanique}\fancyhead[R]{Question 9}

% Énoncé
\begin{center}
	\large{\boldmath{\textbf{Trièdre de Frenet}}}
\end{center}

% Correction

Si on s'intéresse à la trajectoire d'un corps, on peut définir en tout point de celle-ci un trièdre direct\(\vv{t},\vv{n},\vv{u}\), appelé trièdre de Frenet, tel qu'en tout point:\\
i) \(\vv{t}\) est tangent à la trajectoire\\
ii) \(\vv{n}\) est normal à la trajectoire\\
iii) \(\vv{u}=\vv{t}\wedge\vv{n}\)\\

\end{document}
