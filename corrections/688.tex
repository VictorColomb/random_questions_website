\documentclass[a4paper]{article}
\usepackage[T1]{fontenc}
\usepackage[utf8]{inputenc}
\usepackage{lmodern}
\usepackage{amsmath,amssymb}
\usepackage[top=3cm,bottom=2cm,left=2cm,right=2cm]{geometry}
\usepackage{fancyhdr}
\usepackage{esvect}
\usepackage{xcolor}
\usepackage{tikz}\usetikzlibrary{calc}

\parskip 1em\parindent 0pt

\begin{document}

\pagestyle{fancy}
\fancyhf{}
\setlength{\headheight}{15pt}
\fancyhead[L]{Optique}\fancyhead[R]{Question 14}

% Énoncé
\begin{center}
	\large{\boldmath{\textbf{Condition d’interférences constructives d’un réseau}}}
\end{center}

% Correction

On s'intéresse à deux traits consécutifs du réseau.\\
Par théorème de Malus, \( \delta_{1 ,2} = d_1 - d_2 = a(\sin(\theta) -\sin(i))\).\\
Il y a donc interférences constructives si \fcolorbox{red}{white}{\(  a(\sin(\theta) -\sin(i)) \in \lambda \mathbb{Z} \)}.

\end{document}
