\documentclass[a4paper]{article}
\usepackage[T1]{fontenc}
\usepackage[utf8]{inputenc}
\usepackage{lmodern}
\usepackage{amsmath,amssymb}
\usepackage[top=3cm,bottom=2cm,left=2cm,right=2cm]{geometry}
\usepackage{fancyhdr}
\usepackage{esvect,esint}
\usepackage{xcolor}
\usepackage{tikz}\usetikzlibrary{calc}

\parskip1em\parindent0pt\let\ds\displaystyle

\begin{document}

\pagestyle{fancy}
\fancyhf{}
\setlength{\headheight}{15pt}
\fancyhead[L]{Electromagnétisme}\fancyhead[R]{Question 20}

% Énoncé
\begin{center}
	\large{\boldmath{\textbf{Travail reçu par un condensateur}}}
\end{center}

% Correction

Si le condensateur passe de la tension \(U_i\) à la tension \(U_f\), alors il reçoit le travail :
\begin{center}
\fcolorbox{red}{white}{\(W=\dfrac{1}{2}C(U_f^2-U_i^2)\)}
\end{center}

\end{document}
