\documentclass[a4paper]{article}
\usepackage[T1]{fontenc}
\usepackage[utf8]{inputenc}
\usepackage{lmodern}
\usepackage{amsmath,amssymb}
\usepackage[top=3cm,bottom=2cm,left=2cm,right=2cm]{geometry}
\usepackage{fancyhdr}
\usepackage{esvect}
\usepackage{xcolor}
\usepackage{tikz}\usetikzlibrary{calc}

\parskip 1em\parindent 0pt

\begin{document}

\pagestyle{fancy}
\fancyhf{}
\setlength{\headheight}{15pt}
\fancyhead[L]{Electromagnétisme}\fancyhead[R]{Question 44}

% Énoncé
\begin{center}
	\large{\boldmath{\textbf{Relations de Heisenberg}}}
\end{center}

% Correction

Les incertitudes sur la mesure de la position et de la quantité de mouvement vérifient :
\begin{center}
\fcolorbox{red}{white}{\(\Delta x\Delta p_x \gtrsim\dfrac{\hbar}{2}\)}
\end{center}
Les incertitudes sur la mesure de l'énergie et du temps vérifient :
\begin{center}
\fcolorbox{red}{white}{\(\Delta t\Delta E \sim\hbar\)}
\end{center}


\end{document}
