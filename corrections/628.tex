\documentclass[a4paper]{article}
\usepackage[T1]{fontenc}
\usepackage[utf8]{inputenc}
\usepackage{lmodern}
\usepackage{amsmath,amssymb}
\usepackage[top=3cm,bottom=2cm,left=2cm,right=2cm]{geometry}
\usepackage{fancyhdr}
\usepackage{esvect,esint}
\usepackage{xcolor}
\usepackage{tikz}\usetikzlibrary{calc}

\parskip1em\parindent0pt\let\ds\displaystyle

\begin{document}

\pagestyle{fancy}
\fancyhf{}
\setlength{\headheight}{15pt}
\fancyhead[L]{Electromagnétisme}\fancyhead[R]{Question 13}

% Énoncé
\begin{center}
	\large{\boldmath{\textbf{Relations de passage du champ électrique}}}
\end{center}

% Correction

On s'intéresse à une nappe de courant surfacique \(\sigma\) et on note \((1)\) et \((2)\) les demi-espaces de chaque côté de cette nappe.\\
Alors les champs électriques de chaque côté de la nappe vérifient :\begin{center}\fcolorbox{red}{white}{\(\vv{E_2}-\vv{E_1}=\dfrac{\sigma}{\varepsilon_0}\vv{n_{12}}\)}\end{center}
où \(\vv{n_{12}}\) est le vecteur normal à la nappe allant de \((1)\) vers \((2)\).

\end{document}
