\documentclass[a4paper]{article}
\usepackage[T1]{fontenc}
\usepackage[utf8]{inputenc}
\usepackage{lmodern}
\usepackage{amsmath,amssymb}
\usepackage[top=3cm,bottom=2cm,left=2cm,right=2cm]{geometry}
\usepackage{fancyhdr}
\usepackage{esvect,esint}
\usepackage{xcolor}
\usepackage{tikz}\usetikzlibrary{calc}

\parskip1em\parindent0pt\let\ds\displaystyle

\begin{document}

\pagestyle{fancy}
\fancyhf{}
\setlength{\headheight}{15pt}
\fancyhead[L]{Mécanique}\fancyhead[R]{Question 15}

% Énoncé
\begin{center}
	\large{\boldmath{\textbf{Définition puissance et travail d’une force dans un repère}}}
\end{center}

% Correction

Soit un repère \(\mathcal{R}\) quelconque.\\
Soit un corps en mouvement à la vitesse \(\vv{v}\) dans \(\mathcal{R}\) et soumis à une force \(\vv{F}\).\\
On définit la puissance de \(\vv{F}\) dans \(\mathcal{R}\) par :\begin{center}\fcolorbox{red}{white}{\(\mathcal{P}_{\mathcal{R}}=\vv{F}\cdot\vv{v}\)}\end{center}
Si le corps se déplace du point \(A\) au point \(B\) on définit le travail de la force \(\vv{F}\) sur le trajet \(AB\) par \begin{center}\fcolorbox{red}{white}{\(W_{AB}^{\mathcal{R}}=\ds\int_{t_A}^{t_B}\mathcal{P}_{\mathcal{R}}\mathrm{d}t\)}\end{center}
Si \(W_{AB}^{\mathcal{R}}>0\), on dit que la force est motrice, et sinon on dit que la force est résistante.


\end{document}
