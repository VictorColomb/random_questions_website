\documentclass[a4paper]{article}
\usepackage[T1]{fontenc}
\usepackage[utf8]{inputenc}
\usepackage{lmodern}
\usepackage{amsmath,amssymb}
\usepackage[top=3cm,bottom=2cm,left=2cm,right=2cm]{geometry}
\usepackage{fancyhdr}
\usepackage{esvect}
\usepackage{xcolor}
\usepackage{tikz}\usetikzlibrary{calc}

\parskip 1em\parindent 0pt

\begin{document}

\pagestyle{fancy}
\fancyhf{}
\setlength{\headheight}{15pt}
\fancyhead[L]{Mécanique}\fancyhead[R]{Question 20}

% Énoncé
\begin{center}
	\large{\boldmath{\textbf{Théorème de l’énergie mécanique}}}
\end{center}

% Correction

La variation de l'énergie mécanique d'un système est égale à l'opposé du travail des forces non conservatives qui s'y applique : \(\dfrac{\mathrm{d} E_m}{\mathrm{d} t}=P_{\text{nc}}\)
\begin{center}
	\fcolorbox{red}{white}{\(\Delta E_m = W_{\text{nc}}\)}
\end{center}


\end{document}
