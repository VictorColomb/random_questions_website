\documentclass[a4paper]{article}
\usepackage[T1]{fontenc}
\usepackage[utf8]{inputenc}
\usepackage{lmodern}
\usepackage{amsmath,amssymb}
\usepackage[top=3cm,bottom=2cm,left=2cm,right=2cm]{geometry}
\usepackage{fancyhdr}
\usepackage{esvect,esint}
\usepackage{xcolor}
\usepackage{tikz,circuitikz}\usetikzlibrary{calc}

\parskip1em\parindent0pt\let\ds\displaystyle

\begin{document}

\pagestyle{fancy}
\fancyhf{}
\setlength{\headheight}{15pt}
\fancyhead[L]{Electrocinétique}\fancyhead[R]{Question 16}

% Énoncé
\begin{center}
	\large{\boldmath{\textbf{Définition ARQP}}}
\end{center}

% Correction

Pour un circuit de diamètre \(d\) excité par un signal sinusoïdal de fréquence \(f\), on dit que le circuit se trouve en ARQP si et seulement si \(d \ll \dfrac{c}{f}\) où \(c\) est la vitesse de la lumière.\\
L’ARQP est valable si on peut négliger tous les temps de propagation des différents signaux dans le circuit.


\end{document}
