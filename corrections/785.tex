\documentclass[a4paper]{article}
\usepackage[T1]{fontenc}
\usepackage[utf8]{inputenc}
\usepackage{lmodern}
\usepackage{amsmath,amssymb}
\usepackage[top=3cm,bottom=2cm,left=2cm,right=2cm]{geometry}
\usepackage{fancyhdr}
\usepackage{esvect}
\usepackage{xcolor}
\usepackage{tikz}\usetikzlibrary{calc}

\parskip 1em\parindent 0pt

\begin{document}

\pagestyle{fancy}
\fancyhf{}
\setlength{\headheight}{15pt}
\fancyhead[L]{Chimie}\fancyhead[R]{Question 41}

% Énoncé
\begin{center}
	\large{\boldmath{\textbf{Possibilités de transfert de matière autour d’une électrode}}}
\end{center}

% Correction

Il y a trois possibilités de transfert de matière autour d'une électrode:\begin{itemize}
\item
\underline{La migration} : les ions se déplacent sous l'action du champ électrique généré par la différence de potentiel entre les deux électrodes.
\item
\underline{La convection} : la matière est déplacée par action mécanique sur la solution (mélange).
\item
\underline{La diffusion} : un gradient de concentration des différentes espèces entraîne un phénomène de diffusion d'après la loi de Fick.
\end{itemize}


\end{document}
