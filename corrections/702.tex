\documentclass[a4paper]{article}
\usepackage[T1]{fontenc}
\usepackage[utf8]{inputenc}
\usepackage{lmodern}
\usepackage{amsmath,amssymb}
\usepackage[top=3cm,bottom=2cm,left=2cm,right=2cm]{geometry}
\usepackage{fancyhdr}
\usepackage{esvect,esint}
\usepackage{xcolor}
\usepackage{tikz}\usetikzlibrary{calc}

\parskip1em\parindent0pt\let\ds\displaystyle

\begin{document}

\pagestyle{fancy}
\fancyhf{}
\setlength{\headheight}{15pt}
\fancyhead[L]{Thermodynamique}\fancyhead[R]{Question 12}

% Énoncé
\begin{center}
	\large{\boldmath{\textbf{Définition indice adiabatique $\gamma$, \\ expression de $C_v$ et $C_p$ en fonction de $\gamma$}}}
\end{center}

% Correction

Soit un gaz parfait de capacités thermiques à volume constant et à pression constante \(C_v\) et \(C_p\).\\
Par définition, on a :\begin{center}\fcolorbox{red}{white}{\(\gamma=\dfrac{C_p}{C_v}\)}\end{center}
Pour un mélange de \(n\) moles, on a de plus :\begin{center}\fcolorbox{red}{white}{\(C_v=\dfrac{nR}{\gamma-1}\)} et \fcolorbox{red}{white}{\(C_p=\dfrac{nR\gamma}{\gamma-1}\)}\end{center}


\end{document}
