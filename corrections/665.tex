\documentclass[a4paper]{article}
\usepackage[T1]{fontenc}
\usepackage[utf8]{inputenc}
\usepackage{lmodern}
\usepackage{amsmath,amssymb}
\usepackage[top=3cm,bottom=2cm,left=2cm,right=2cm]{geometry}
\usepackage{fancyhdr}
\usepackage{esvect,esint}
\usepackage{xcolor}
\usepackage{tikz}\usetikzlibrary{calc}

\parskip1em\parindent0pt\let\ds\displaystyle

\begin{document}

\pagestyle{fancy}
\fancyhf{}
\setlength{\headheight}{15pt}
\fancyhead[L]{Electromagnétisme}\fancyhead[R]{Question 50}

% Énoncé
\begin{center}
	\large{\boldmath{\textbf{Effet Hall}}}
\end{center}

% Correction

Soit un barreau conducteur de longueur $\ell$ et de section $bc$ soumis à une différence de potentiel $U$ et à un champ magnétique uniforme $\vv{B}$ selon $\vv y$\par
Examen qualitatif : initialement, les électrons du conducteur ont un mouvement parallèle à $\vv{x}$ sous l'effet du champ électrique $\vv{E}$ provenant de $U$. Les électrons sont déviés vers la face inférieure sous l'effet du champ magnétostatique. Il y a une accumulation d'électrons sur la surface intérieure.\\
Il y a donc création d'un champ électrique $\vv*EH$ qui a pour fonction d'empêcher l'accumulation de diverger. Il y a donc à présent trois champs dans le matériau : $\vv E,\vv B$ et $\vv*EH$ selon $\vv z$.

On raisonne sur des électrons libres. Ils sont soumis à une force de viscosité (effet Joule) : $-\dfrac{m_e}{\tau}\vv*{v}{e}$\\
En régime permanant, PFD : $\vv{0}=-e\vv{E}-\dfrac{m_e}{\tau}\vv*{v}{e}-e\vv*{E}{H}-e\vv*{v}{e}\wedge\vv{B}$\\
On suppose que $\vv*{v}{e}=v_e\vv{x}$\\
On projette le PFD : $\vv*{v}{e}=-\dfrac{e\tau}{m_e}\vv{E}$ donc $\vv*{v}{e}$ est uniforme dans le barreau et $\vv*{E}{H}=-\vv*{v}{e}\wedge\vv{B}=-v_eB\vv{z}$ est également uniforme.\par
$\mathrm{div}(\vv*{E}{H}+\vv{E})=0=\dfrac{\rho}{\varepsilon_0}$\\
Donc la distribution de charge est surfacique dans le barreau.
\\
\begin{minipage}{0.5\linewidth}
  $V_{\text{sup}}-V_{\text{inf}}=-\displaystyle\int_{\text{inf}}^{\text{sup}}\vv*EH\cdot\vv{\mathrm{d}\ell}=-E_Hc=-v_eBc$\\
  $I=\displaystyle\iint\vv{j}\cdot\vv{\mathrm{d}S}=\rho_ev_eS=-en_ev_ebc$\\
  donc $v_e=-\dfrac{I}{en_ebc}$\\
  $V_{\text{sup}}-V_{\text{inf}}=-\dfrac{B}{en_eb}I=R_H\dfrac{BI}{b}$\\
  où $R_H=-\dfrac{1}{en_e}$ appelée constante de Hall du matériau
\end{minipage}
\\
On a fabriqué un teslamètre !\\
En pratique, la constante de Hall est tellement faible qu'il ne peut fournir un teslamètre. Pour cela on utilise des semi-conducteurs à trous pour lesquels $R_H=\dfrac{1}{n_pe}$, cela fournit une bonne sonde magnétique.



\end{document}
