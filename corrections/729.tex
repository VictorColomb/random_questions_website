\documentclass[a4paper]{article}
\usepackage[T1]{fontenc}
\usepackage[utf8]{inputenc}
\usepackage{lmodern}
\usepackage{amsmath,amssymb}
\usepackage[top=3cm,bottom=2cm,left=2cm,right=2cm]{geometry}
\usepackage{fancyhdr}
\usepackage{esvect,esint}
\usepackage{xcolor}
\usepackage{tikz}\usetikzlibrary{calc}

\parskip 1em\parindent 0pt

\begin{document}

\pagestyle{fancy}
\fancyhf{}
\setlength{\headheight}{15pt}
\fancyhead[L]{Thermodynamique}\fancyhead[R]{Question 39}

% Énoncé
\begin{center}
	\large{\boldmath{\textbf{Définition flux de conduction, de convection}}}
\end{center}

% Correction

\begin{center}
	\fcolorbox{red}{white}{ \( \phi\,^{\mathrm{cd}/\mathrm{cv}} = \displaystyle\iint \vv j\,^{\mathrm{cd}/\mathrm{cv}} \cdot \vv{\mathrm{d}S} \) }
\end{center}

Pour la convection, \( \vv j^{\mathrm{cv}} = \dfrac{\mathrm{d}H}{\mathrm{d}V} \vv v = h\rho \vv v \)\\
Pour la conduction, la loi de Fourier donne \( \vv j^{\mathrm{cd}} = -\lambda \vv{\mathrm{grad}}\,T \)


\end{document}
