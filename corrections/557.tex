\documentclass[a4paper]{article}
\usepackage[T1]{fontenc}
\usepackage[utf8]{inputenc}
\usepackage{lmodern}
\usepackage{amsmath,amssymb}
\usepackage[top=3cm,bottom=2cm,left=2cm,right=2cm]{geometry}
\usepackage{fancyhdr}
\usepackage{esvect,esint}
\usepackage{xcolor}
\usepackage{tikz,circuitikz}\usetikzlibrary{calc}

\parskip 1em\parindent 0pt

\begin{document}

\pagestyle{fancy}
\fancyhf{}
\setlength{\headheight}{15pt}
\fancyhead[L]{Electrocinétique}\fancyhead[R]{Question 9}

% Énoncé
\begin{center}
	\large{\boldmath{\textbf{Loi des noeuds}}}
\end{center}

% Correction

Soit un volume $V$ entouré par une surface $S$ orientée vers l'extérieur. \\
On sait que \( \mathrm{div} \vv j = 0 \) donc \( \displaystyle\oiint_S \vv j\cdot\vv{\mathrm{d}S} = \displaystyle\iiint_V \mathrm{div}\vv j\,\mathrm{d}V = 0 \), par théorème d'Ostrogradski. \\
Donc \( I_S = 0 \).
\begin{center}
\fcolorbox{red}{white}{La somme des courants entrant dans un noeud est nulle.}
\end{center}

\end{document}
