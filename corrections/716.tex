\documentclass[a4paper]{article}
\usepackage[T1]{fontenc}
\usepackage[utf8]{inputenc}
\usepackage{lmodern}
\usepackage{amsmath,amssymb}
\usepackage[top=3cm,bottom=2cm,left=2cm,right=2cm]{geometry}
\usepackage{fancyhdr}
\usepackage{esvect,esint}
\usepackage{xcolor}
\usepackage{tikz}\usetikzlibrary{calc}

\parskip1em\parindent0pt\let\ds\displaystyle

\begin{document}

\pagestyle{fancy}
\fancyhf{}
\setlength{\headheight}{15pt}
\fancyhead[L]{Thermodynamique}\fancyhead[R]{Question 26}

% Énoncé
\begin{center}
	\large{\boldmath{\textbf{Équations sur un cycle  d’une machine thermique ditherme}}}
\end{center}

% Correction

Soit une machine thermique ditherme dont on note \(W\) le travail et \(Q_c,Q_f\) les transferts thermiques sur un cycle.\\
Tous ces flux énergétique sont orientés vers la machine.\\
Sur un cycle, on a :\(\left\lbrace\begin{array}{ll}W+Q_c+Q_f=0\\\dfrac{Q_c}{T_c}+\dfrac{Q_f}{T_f}+S_c=0\end{array}\right.\)
D'où :\begin{center}\fcolorbox{red}{white}{\(\left\lbrace\begin{array}{ll}W+Q_c+Q_f=0\\\dfrac{Q_c}{T_c}+\dfrac{Q_f}{T_f}\leqslant0\end{array}\right.\)}\end{center}


\end{document}
