\documentclass[a4paper]{article}
\usepackage[T1]{fontenc}
\usepackage[utf8]{inputenc}
\usepackage{lmodern}
\usepackage{amsmath,amssymb}
\usepackage[top=3cm,bottom=2cm,left=2cm,right=2cm]{geometry}
\usepackage{fancyhdr}
\usepackage{esvect}
\usepackage{xcolor}
\usepackage{tikz}\usetikzlibrary{calc}

\parskip 1em\parindent 0pt

\begin{document}

\pagestyle{fancy}
\fancyhf{}
\setlength{\headheight}{15pt}
\fancyhead[L]{Mécanique}\fancyhead[R]{Question 25}

% Énoncé
\begin{center}
	\large{\boldmath{\textbf{Théorèmes du centre d’inertie, du moment cinétique \\ et de l’énergie cinétique}}}
\end{center}

% Correction

Théorème du centre d'inertie :\\
Dans un référentiel galiléen \(\mathcal{R}_g\), la somme des actions des forces extérieures s'exerçant sur un solide est proportionnelle à l'accélération de son centre d'inertie \(G\) :
\begin{center}
\fcolorbox{red}{white}{\(m\vv{a}(G\in\mathcal{R}_g)=\displaystyle\sum\vv{F_{\mathrm{ext}}}\)}
\end{center}

Théorème du moment cinétique :\\
Dans un référentiel galiléen \(\mathcal{R}_g\), la dérivée du moment cinétique en \(O\) d'un solide est égale à la somme des moments en \(O\) des forces extérieures s'exerçant sur ce solide :
\begin{center}
\fcolorbox{red}{white}{\(\dfrac{\mathrm{d}\vv{\sigma}_{\mathcal{R}_g}(O)}{\mathrm{d}t}=\displaystyle\sum\vv{M}(\vv{F},O)\)}
\end{center}
où \( \vv M(\vv F,O) = \sum\limits_i \vv*{OP}i\wedge\vv F(P_i) \).

Théorème de l'énergie cinétique :\\
Dans un référentiel galiléen, la dérivée de l'énergie cinétique d'un solide est égale à la somme des puissances des forces extérieures et \underline{intérieures} s'exerçant sur ce solide :
\begin{center}
\fcolorbox{red}{white}{\(\dfrac{\mathrm{d}E_c^{\mathcal{R}_g}}{\mathrm{d}t}=P_{\mathrm{int}}+P_{\mathrm{ext}}\)}
\end{center}




\end{document}
