\documentclass[a4paper]{article}
\usepackage[T1]{fontenc}
\usepackage[utf8]{inputenc}
\usepackage{lmodern}
\usepackage{amsmath,amssymb}
\usepackage[top=3cm,bottom=2cm,left=2cm,right=2cm]{geometry}
\usepackage{fancyhdr}
\usepackage{esvect}
\usepackage{xcolor}
\usepackage{tikz}\usetikzlibrary{calc}

\parskip 1em\parindent 0pt

\begin{document}

\pagestyle{fancy}
\fancyhf{}
\setlength{\headheight}{15pt}
\fancyhead[L]{Thermodynamique}\fancyhead[R]{Question 30}

% Énoncé
\begin{center}
	\large{\boldmath{\textbf{Définition point critique}}}
\end{center}

% Correction

Le point critique d'un corps pur est le point du diagramme température-pression, généralement noté $C$, où s'arrête la courbe d'équilibre liquide-gaz.

Au-delà de ce point, c'est à dire pour \( T > T_C \) ou pour \( P > P_C \), le changement d'état entre liquide et vapeur est continu.

\end{document}
