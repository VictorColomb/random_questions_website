\documentclass[a4paper]{article}
\usepackage[T1]{fontenc}
\usepackage[utf8]{inputenc}
\usepackage{lmodern}
\usepackage{amsmath,amssymb}
\usepackage[top=3cm,bottom=2cm,left=2cm,right=2cm]{geometry}
\usepackage{fancyhdr}
\usepackage{esvect,esint}
\usepackage{xcolor}
\usepackage{tikz}\usetikzlibrary{calc}

\parskip1em\parindent0pt\let\ds\displaystyle

\begin{document}

\pagestyle{fancy}
\fancyhf{}
\setlength{\headheight}{15pt}
\fancyhead[L]{Electromagnétisme}\fancyhead[R]{Question 32}

% Énoncé
\begin{center}
	\large{\boldmath{\textbf{Equations de Maxwell avec noms}}}
\end{center}

% Correction

Équation de Maxwell-Gauss :\begin{center}\fcolorbox{red}{white}{\(\mathrm{div}\vv{E}=\dfrac{\rho}{\varepsilon_0}\)}\end{center}
Équation de Maxwell-Ampère :\begin{center}\fcolorbox{red}{white}{\(\vv{\mathrm{rot}}\vv{B}=\mu_0\vv{j}+\mu_0\varepsilon_0\dfrac{\partial\vv{E}}{\partial t}\)}\end{center}
Équation de Maxwell-Faraday :\begin{center}\fcolorbox{red}{white}{\(\vv{\mathrm{rot}}\vv{E}=-\dfrac{\partial \vv{B}}{\partial t}\)}\end{center}
Équation de Maxwell flux conservatif :\begin{center}\fcolorbox{red}{white}{\(\mathrm{div}\vv{B}=0\)}\end{center}


\end{document}
