\documentclass[a4paper]{article}
\usepackage[T1]{fontenc}
\usepackage[utf8]{inputenc}
\usepackage{lmodern}
\usepackage{amsmath,amssymb}
\usepackage[top=3cm,bottom=2cm,left=2cm,right=2cm]{geometry}
\usepackage{fancyhdr}
\usepackage{esvect,esint}
\usepackage{xcolor}
\usepackage{tikz,circuitikz}\usetikzlibrary{calc}

\parskip1em\parindent0pt\let\ds\displaystyle

\begin{document}

\pagestyle{fancy}
\fancyhf{}
\setlength{\headheight}{15pt}
\fancyhead[L]{Electrocinétique}\fancyhead[R]{Question 11}

% Énoncé
\begin{center}
	\large{\boldmath{\textbf{Diviseur de courant}}}
\end{center}

% Correction


Pour le circuit suivant :\begin{center}
\begin{minipage}{0.3\linewidth}
  \begin{circuitikz}
    \draw (0,3) to[short,i=$I$] ++(2,0) -- ++(0,-.5) -- ++(1,0) to[R=$R_2$,i=$I_2$] ++(0,-2) -- ++(-1,0) -- ++(0,-.5) -- (0,0);
    \draw (2,2.5) -- ++(-1,0) to[R=$R_1$,i_=$I_1$] ++(0,-2) -- ++(1,0);
    \draw[->] (0,.5) -- node[left] {$U$} ++(0,2);
  \end{circuitikz}
\end{minipage}
\end{center}
On a :\begin{center}\fcolorbox{red}{white}{\(I_1=\dfrac{G_1}{G_1+G_2}I=\dfrac{R_2}{R_1+R_2}I\)}\end{center}

Donc \begin{center}\fcolorbox{red}{white}{\(G_\text{équivalent}=G_1+G_2\)}\end{center}


\end{document}
