\documentclass[a4paper]{article}
\usepackage[T1]{fontenc}
\usepackage[utf8]{inputenc}
\usepackage{lmodern}
\usepackage{amsmath,amssymb}
\usepackage[top=3cm,bottom=2cm,left=2cm,right=2cm]{geometry}
\usepackage{fancyhdr}
\usepackage{esvect,esint}
\usepackage{xcolor}
\usepackage{tikz,circuitikz}\usetikzlibrary{calc}

\parskip1em\parindent0pt\let\ds\displaystyle

\begin{document}

\pagestyle{fancy}
\fancyhf{}
\setlength{\headheight}{15pt}
\fancyhead[L]{Electrocinétique}\fancyhead[R]{Question 10}

% Énoncé
\begin{center}
	\large{\boldmath{\textbf{Diviseur de tension}}}
\end{center}

% Correction


Dans le circuit ci-dessous :\begin{center}
\begin{minipage}{0.3\linewidth}
  \begin{circuitikz}[scale=.8]
    \draw (0,4) to[short,i=$I$] (1,4) to[R=$R_1$] (1,2) to[R=$R_2$] (1,0) -- (0,0);
    \draw (1,2) to[short,i=0] (3,2);
    \draw[->] (0,.5) -- node[left] {$U$} (0,3.5);
    \draw[->] (3,0) -- node[right] {$U_2$} ++(0,1.8);
  \end{circuitikz}
\end{minipage}
\end{center}
On a :\begin{center}\fcolorbox{red}{white}{\(U_2=\dfrac{R_2}{R_1+R_1}\)}\end{center}

\end{document}
