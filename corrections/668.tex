\documentclass[a4paper]{article}
\usepackage[T1]{fontenc}
\usepackage[utf8]{inputenc}
\usepackage{lmodern}
\usepackage{amsmath,amssymb}
\usepackage[top=3cm,bottom=2cm,left=2cm,right=2cm]{geometry}
\usepackage{fancyhdr}
\usepackage{esvect,esint}
\usepackage{xcolor}
\usepackage{tikz}\usetikzlibrary{calc}

\parskip1em\parindent0pt\let\ds\displaystyle

\begin{document}

\pagestyle{fancy}
\fancyhf{}
\setlength{\headheight}{15pt}
\fancyhead[L]{Electromagnétisme}\fancyhead[R]{Question 53}

% Énoncé
\begin{center}
	\large{\boldmath{\textbf{Flux magnétique et mutuelle inductance}}}
\end{center}

% Correction

Soit \((1)\) et \((2)\) deux circuits parcourus par des courants d'intensité \(I_1\) et \(I_2\).\\
Soit \(\phi_{1\rightarrow2}\) et \(\phi_{2\rightarrow1}\) les flux magnétiques envoyés respectivement par \((1)\) à travers \((2)\) et par \((2)\) à travers \((1)\).\\
Alors \(\dfrac{\phi_{1\rightarrow2}}{I_1}\) et \(\dfrac{\phi_{2\rightarrow1}}{I_2}\) sont constants et égaux.\\
On note \(\mathcal{M}\) leur valeur commune, qui est une constante appelée mutuelle inductance ne dépendant que de la géométrie des circuits.\\
On a \fcolorbox{red}{white}{\(\phi_{1\rightarrow2}=\mathcal{M}I_1\) et \(\phi_{2\rightarrow1}=\mathcal{M}I_2\)}


\end{document}
