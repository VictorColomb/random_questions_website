\documentclass[a4paper]{article}
\usepackage[T1]{fontenc}
\usepackage[utf8]{inputenc}
\usepackage{lmodern}
\usepackage{amsmath,amssymb}
\usepackage[top=3cm,bottom=2cm,left=2cm,right=2cm]{geometry}
\usepackage{fancyhdr}
\usepackage{esvect}
\usepackage{xcolor}
\usepackage{tikz}\usetikzlibrary{calc}

\parskip 1em\parindent 0pt

\begin{document}

\pagestyle{fancy}
\fancyhf{}
\setlength{\headheight}{15pt}
\fancyhead[L]{Chimie}\fancyhead[R]{Question 23}

% Énoncé
\begin{center}
	\large{\boldmath{\textbf{Variation d’une grandeur extensive lors d’une réaction}}}
\end{center}

% Correction

On note \(\xi_i\) et \(\xi_f\) les avancements initiaux et finaux de la réaction.\\
On a alors, pout toute grandeur extensive Y, \[\Delta Y=\int_{\xi_i}^{\xi_f}\Delta_rY\partial\xi\]

\end{document}
