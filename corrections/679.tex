\documentclass[a4paper]{article}
\usepackage[T1]{fontenc}
\usepackage[utf8]{inputenc}
\usepackage{lmodern}
\usepackage{amsmath,amssymb}
\usepackage[top=3cm,bottom=2cm,left=2cm,right=2cm]{geometry}
\usepackage{fancyhdr}
\usepackage{esvect}
\usepackage{xcolor}
\usepackage{tikz}\usetikzlibrary{calc}

\parskip 1em\parindent 0pt

\begin{document}

\pagestyle{fancy}
\fancyhf{}
\setlength{\headheight}{15pt}
\fancyhead[L]{Optique}\fancyhead[R]{Question 5}

% Énoncé
\begin{center}
	\large{\boldmath{\textbf{Définition du chemin optique}}}
\end{center}

% Correction

Le chemin optique correspond à la distance que parcourerait l’onde dans le vide pendant le temps qu’elle met dans le milieu considéré.\par
\fcolorbox{red}{white}{\( [AB] = \displaystyle\int_{A}^{B}n\, \mathrm{d}\ell \)} avec \(n=\dfrac{c}{v}\) l'indice optique du milieu.


\end{document}
