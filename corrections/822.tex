\documentclass[a4paper]{article}
\usepackage[T1]{fontenc}
\usepackage[utf8]{inputenc}
\usepackage{lmodern}
\usepackage{amsmath,amssymb}
\usepackage[top=3cm,bottom=2cm,left=2cm,right=2cm]{geometry}
\usepackage{fancyhdr}
\usepackage{esvect,esint}
\usepackage{xcolor}
\usepackage{tikz}\usetikzlibrary{calc}

\parskip1em\parindent0pt\let\ds\displaystyle

\begin{document}

\pagestyle{fancy}
\fancyhf{}
\setlength{\headheight}{15pt}
\fancyhead[L]{SLCI}\fancyhead[R]{Question 7}

% Énoncé
\begin{center}
	\large{\boldmath{\textbf{Condition de stabilité d'un SLCI en rapport avec \\ les poles de sa fonction de transfert.}}}
\end{center}

% Correction

Un SLCI est stable si et seulement si les pôles de sa fonction de transfert sont de partie réelle strictement négative.

Si les pôles sont de partie réelle seulement négative, le SLCI est largement stable, dans le sens où sa sortie est bornée.

Pour les systèmes du premier ou du deuxième ordre, cela équivaut à ce que les coefficients du dénominateur de leur fonction de transfert soient de même signe.

\end{document}
