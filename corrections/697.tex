\documentclass[a4paper]{article}
\usepackage[T1]{fontenc}
\usepackage[utf8]{inputenc}
\usepackage{lmodern}
\usepackage{amsmath,amssymb}
\usepackage[top=3cm,bottom=2cm,left=2cm,right=2cm]{geometry}
\usepackage{fancyhdr}
\usepackage{esvect}
\usepackage{xcolor}
\usepackage{tikz}\usetikzlibrary{calc}

\parskip 1em\parindent 0pt

\begin{document}

\pagestyle{fancy}
\fancyhf{}
\setlength{\headheight}{15pt}
\fancyhead[L]{Thermodynamique}\fancyhead[R]{Question 7}

% Énoncé
\begin{center}
	\large{\boldmath{\textbf{Premier principe de la thermodynamique (3 propriétés)}}}
\end{center}

% Correction

Il existe une grandeur \(U\) vérifiant:\\
i) \(U\) est extensive\\
ii) \(U\) est une fonction d'état\\
iii) Pour un système fermé, on a
\begin{center}\fcolorbox{red}{white}{\(\Delta (U+E_m)=Q+W\)}\end{center}
où \(E_m\) est l'énergie mécanique du système et \(Q\) et \(W\) la chaleur et le travail qu'il reçoit.

\end{document}
