\documentclass[a4paper]{article}
\usepackage[T1]{fontenc}
\usepackage[utf8]{inputenc}
\usepackage{lmodern}
\usepackage{amsmath,amssymb}
\usepackage[top=3cm,bottom=2cm,left=2cm,right=2cm]{geometry}
\usepackage{fancyhdr}
\usepackage{esvect}
\usepackage{xcolor}
\usepackage{tikz}\usetikzlibrary{calc}

\parskip 1em\parindent 0pt

\begin{document}

\pagestyle{fancy}
\fancyhf{}
\setlength{\headheight}{15pt}
\fancyhead[L]{Thermodynamique}\fancyhead[R]{Question 4}

% Énoncé
\begin{center}
	\large{\boldmath{\textbf{Définition transformation quasi-statique}}}
\end{center}

% Correction

Une transformation quasi-statique est une transformation dans laquelle le système est dans un état thermodynamique à tout instant, c'est-à-dire on peut définir le système par un petit nombre de variables.


\end{document}
