\documentclass[a4paper]{article}
\usepackage[T1]{fontenc}
\usepackage[utf8]{inputenc}
\usepackage{lmodern}
\usepackage{amsmath,amssymb}
\usepackage[top=3cm,bottom=2cm,left=2cm,right=2cm]{geometry}
\usepackage{fancyhdr}
\usepackage{esvect}
\usepackage{xcolor}
\usepackage{tikz}\usetikzlibrary{calc}

\parskip 1em\parindent 0pt

\begin{document}

\pagestyle{fancy}
\fancyhf{}
\setlength{\headheight}{15pt}
\fancyhead[L]{Electromagnétisme}\fancyhead[R]{Question 43}

% Énoncé
\begin{center}
	\large{\boldmath{\textbf{Vitesse de groupe}}}
\end{center}

% Correction

Soit une onde électromagnétique de pulsation $\omega$ et de vecteur d'onde $\vv k$.\\
On définit la vitesse de groupe par : \fcolorbox{red}{white}{ \( v_g = \dfrac{\mathrm{d}\,\omega}{\mathrm{d}\,k} \) }
\par

\textit{Remarque} : La vitesse de groupe est la vitesse de propagation de l'énergie.

\end{document}
