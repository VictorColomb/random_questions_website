\documentclass[a4paper]{article}
\usepackage[T1]{fontenc}
\usepackage[utf8]{inputenc}
\usepackage{lmodern}
\usepackage{amsmath,amssymb}
\usepackage[top=3cm,bottom=2cm,left=2cm,right=2cm]{geometry}
\usepackage{fancyhdr}
\usepackage{esvect}
\usepackage{xcolor}
\usepackage{tikz}\usetikzlibrary{calc}

\parskip 1em\parindent 0pt

\begin{document}

\pagestyle{fancy}
\fancyhf{}
\setlength{\headheight}{15pt}
\fancyhead[L]{Thermodynamique}\fancyhead[R]{Question 19}

% Énoncé
\begin{center}
	\large{\boldmath{\textbf{Second principe de la thermodynamique (3 propriétés)}}}
\end{center}

% Correction

Il existe une fonction \(S\) vérifiant:
\begin{enumerate}
\item \(S\) est une fonction d'état.
\item \(S\) est extensive.
\item Pour une transformation spontanée d'un système isolé, \(S\) est croissante.
\end{enumerate}

\end{document}
