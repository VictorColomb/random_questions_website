\documentclass[a4paper]{article}
\usepackage[T1]{fontenc}
\usepackage[utf8]{inputenc}
\usepackage{lmodern}
\usepackage{amsmath,amssymb}
\usepackage[top=3cm,bottom=2cm,left=2cm,right=2cm]{geometry}
\usepackage{fancyhdr}
\usepackage{esvect,esint}
\usepackage{xcolor}
\usepackage{tikz,circuitikz}\usetikzlibrary{calc}

\parskip1em\parindent0pt\let\ds\displaystyle

\begin{document}

\pagestyle{fancy}
\fancyhf{}
\setlength{\headheight}{15pt}
\fancyhead[L]{Electrocinétique}\fancyhead[R]{Question 25}

% Énoncé
\begin{center}
	\large{\boldmath{\textbf{Étude de $\underline{H}=jx$}}}
\end{center}

% Correction


On a $\underline{H}=jx=j\dfrac{\omega}{\omega_0}$\\
  donc $G=20\log\omega-20\log\omega_0$ \hspace{1cm} (droite de pente 20dB/décade)\\
  et $\varphi=\dfrac{\pi}{2}=90^{\circ}$
\begin{center}
  \begin{tikzpicture}
  \draw [thick,->] (-2.5,0) to (2.5,0) node[anchor=north west]{\(\log\omega\)};
  \draw[thick,->](0,-2.5) to (0,2.5) node[anchor=south east]{\(G_{\mathrm{dB}}\)};
  \draw (-2.5,-2.5) to (2.5,2.5) node[anchor=south west]{20dB/décade};
  \draw (0,0) node[anchor=north west]{\(\log\omega_0\)};
\end{tikzpicture}
\end{center}
\begin{center}
\begin{tikzpicture}
  \draw [thick,->] (-2.5,0) to (2.5,0) node[anchor=north west]{\(\log\omega\)};
  \draw[thick,->](0,-2.5) to (0,2.5) node[anchor=south east]{\(\varphi\)};
  \draw (-2.5,2) to (2.5,2);
  \draw (0,2) node [anchor=south east ]{\(90^{\circ}\)};
  
\end{tikzpicture}
\end{center}


\end{document}
