\documentclass[a4paper]{article}
\usepackage[T1]{fontenc}
\usepackage[utf8]{inputenc}
\usepackage{lmodern}
\usepackage{amsmath,amssymb}
\usepackage[top=3cm,bottom=2cm,left=2cm,right=2cm]{geometry}
\usepackage{fancyhdr}
\usepackage{esvect,esint}
\usepackage{xcolor}
\usepackage{tikz}\usetikzlibrary{calc}

\parskip1em\parindent0pt\let\ds\displaystyle

\begin{document}

\pagestyle{fancy}
\fancyhf{}
\setlength{\headheight}{15pt}
\fancyhead[L]{Chimie}\fancyhead[R]{Question 19}

% Énoncé
\begin{center}
	\large{\boldmath{\textbf{Deuxième loi de Kirschoff}}}
\end{center}

% Correction

\fcolorbox{red}{white}{\(\Delta_rS^{\circ}(T)=\Delta_rS^{\circ}(T_0)+\displaystyle\int_{T_0}^T\dfrac{\Delta_rC_p^{\circ}}{T}\mathrm{d}T\simeq\Delta_rS^{\circ}(T_0)\)}


\end{document}
