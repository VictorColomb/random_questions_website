\documentclass[a4paper]{article}
\usepackage[T1]{fontenc}
\usepackage[utf8]{inputenc}
\usepackage{lmodern}
\usepackage{amsmath,amssymb}
\usepackage[top=3cm,bottom=2cm,left=2cm,right=2cm]{geometry}
\usepackage{fancyhdr}
\usepackage{esvect,esint}
\usepackage{xcolor}
\usepackage{tikz,circuitikz}\usetikzlibrary{calc}

\parskip1em\parindent0pt\let\ds\displaystyle

\begin{document}

\pagestyle{fancy}
\fancyhf{}
\setlength{\headheight}{15pt}
\fancyhead[L]{Electrocinétique}\fancyhead[R]{Question 13}

% Énoncé
\begin{center}
	\large{\boldmath{\textbf{Théorème de superposition}}}
\end{center}

% Correction

Soit un réseau linéaire composé de \(k\) branches dont le courant est \(i_j,j\in\lbrace 1,\cdots,k\rbrace\).\\
Soit l'état \((I)\) du réseau défini par \(N\) sources de tension et \(M\) sources de courant.\\
On représente cet état par \(\left\lbrace E_1^{(I)},\cdots,E_N^{(I)}\right\rbrace,\left\lbrace I_1^{(I)},\cdots,I_M^{(I)}\right\rbrace\).\\
Soit de même l'état \((II)\) représenté par \(\left\lbrace E_1^{(II)},\cdots,E_N^{(II)}\right\rbrace,\left\lbrace I_1^{(II)},\cdots,I_M^{(II)}\right\rbrace\).

Alors pour l'état \(\alpha(I)+\beta(II)\), le courant \(i_j\) vérifie \begin{center}\fcolorbox{red}{white}{\(i_j=\alpha i_j^{(I)}+\beta i_j^{(II)}\)}\end{center} pour tout \(j\in\lbrace 1,\cdots,k\rbrace\).

\end{document}
