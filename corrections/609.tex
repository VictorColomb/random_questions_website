\documentclass[a4paper]{article}
\usepackage[T1]{fontenc}
\usepackage[utf8]{inputenc}
\usepackage{lmodern}
\usepackage{amsmath,amssymb}
\usepackage[top=3cm,bottom=2cm,left=2cm,right=2cm]{geometry}
\usepackage{fancyhdr}
\usepackage{esvect,esint}
\usepackage{xcolor}
\usepackage{tikz}\usetikzlibrary{calc}

\parskip 1em\parindent 0pt

\begin{document}

\pagestyle{fancy}
\fancyhf{}
\setlength{\headheight}{15pt}
\fancyhead[L]{Mécanique}\fancyhead[R]{Question 22}

% Énoncé
\begin{center}
	\large{\boldmath{\textbf{Définition centre d’inertie d’un système de points matériels}}}
\end{center}

% Correction

On note \(C\) le centre d'inertie du système.\\
Alors
\begin{center}
	\fcolorbox{red}{white}{\(\vv{OC}=\displaystyle\iiint\limits_V\mu(P)\vv{OP}\,\mathrm{d}V(P)\)}
\end{center}


\end{document}
