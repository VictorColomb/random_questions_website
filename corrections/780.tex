\documentclass[a4paper]{article}
\usepackage[T1]{fontenc}
\usepackage[utf8]{inputenc}
\usepackage{lmodern}
\usepackage{amsmath,amssymb}
\usepackage[top=3cm,bottom=2cm,left=2cm,right=2cm]{geometry}
\usepackage{fancyhdr}
\usepackage{esvect,esint}
\usepackage{xcolor}
\usepackage{tikz}\usetikzlibrary{calc}

\parskip1em\parindent0pt\let\ds\displaystyle

\begin{document}

\pagestyle{fancy}
\fancyhf{}
\setlength{\headheight}{15pt}
\fancyhead[L]{Chimie}\fancyhead[R]{Question 36}

% Énoncé
\begin{center}
	\large{\boldmath{\textbf{Réaction sur une anode/une cathode, \\ orientation canonique du courant et \\ signe sur une anode/une cathode}}}
\end{center}

% Correction

Le courant est canoniquement orienté de l'extérieur vers l'intérieur de l'électrode, l'extérieur étant le reste du circuit.\\
A la cathode, la réaction qui se produit est une réduction, elle reçoit des électrons de l'électrode et le courant est négatif.\\
A l'anode, la réaction qui se produit est une réduction, elle cède des électrons à l'électrode et le courant est positif.\\
\emph{Moyens mnémotchniques: } Pour une pile\\
\emph{"L'occident c'est ceux qui ont tout volé"} donc l'oxydant vole les électrons. Ainsi l'oxydation a lieu à la borne négative.\\
\begin{table}[h!]
\center
\begin{tabular}{*{25}{c}}
O&X&Y&D&A&T&I&O&N&&&&&&&&R&É&D&U&C&T&I&O&N\\
&&&&N&&&&&&&&&&&&&&&&A\\
&&&&O&&&&&&&&&&&&&&&&T\\
&&&&D&&&&&&&&&&&&&&&&H\\
&&&&E&&&&&&&&&&&&&&&&O\\
&&&&&&&&&&&&&&&&&&&&D\\
&&&&&&&&&&&&&&&&&&&&E

\end{tabular}
\end{table}









\end{document}
