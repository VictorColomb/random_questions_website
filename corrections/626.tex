\documentclass[a4paper]{article}
\usepackage[T1]{fontenc}
\usepackage[utf8]{inputenc}
\usepackage{lmodern}
\usepackage{amsmath,amssymb}
\usepackage[top=3cm,bottom=2cm,left=2cm,right=2cm]{geometry}
\usepackage{fancyhdr}
\usepackage{esvect}
\usepackage{xcolor}
\usepackage{tikz}\usetikzlibrary{calc}

\parskip 1em\parindent 0pt

\begin{document}

\pagestyle{fancy}
\fancyhf{}
\setlength{\headheight}{15pt}
\fancyhead[L]{Electromagnétisme}\fancyhead[R]{Question 11}

% Énoncé
\begin{center}
	\large{\boldmath{\textbf{Énergie potentielle d’un dipôle dans un champ électrique}}}
\end{center}

% Correction

On considère un dipôle électrique $\vv p$ dans un champ électrique $\vv E$\\
\fcolorbox{red}{white}{ Alors, \( U_{\mathrm{pot}} = -\vv p \cdot \vv E \) }
\par

Cela découle directement de l'expression de la force exercée par un champ électrique sur un dipôle : \\
\( \vv F = \left.\mathrm{grad}\,(\vv p \cdot \vv E)\right|_{\vv p,\;\text{figé}} \)

\end{document}
