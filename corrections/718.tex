\documentclass[a4paper]{article}
\usepackage[T1]{fontenc}
\usepackage[utf8]{inputenc}
\usepackage{lmodern}
\usepackage{amsmath,amssymb}
\usepackage[top=3cm,bottom=2cm,left=2cm,right=2cm]{geometry}
\usepackage{fancyhdr}
\usepackage{esvect,esint}
\usepackage{xcolor}
\usepackage{tikz}\usetikzlibrary{calc}

\parskip1em\parindent0pt\let\ds\displaystyle

\begin{document}

\pagestyle{fancy}
\fancyhf{}
\setlength{\headheight}{15pt}
\fancyhead[L]{Thermodynamique}\fancyhead[R]{Question 28}

% Énoncé
\begin{center}
	\large{\boldmath{\textbf{Efficacité d’un réfrigérateur et  d’une pompe à chaleur, \\ efficacité de Carnot}}}
\end{center}

% Correction

On note \(Q_f,Q_c\) les transferts thermiques à la source froide (respectivement chaude) et \(T_f,T_c\) les températures de ces sources.\\
L'efficacité d'un réfrigérateur est définie par :
\begin{center}\fcolorbox{red}{white}{\(e=\dfrac{Q_f}{T_f}\)}\end{center}
Elle est majorée par l'efficacité de Carnot :
\begin{center}\fcolorbox{red}{white}{\(e_c=\dfrac{T_f}{T_c-T_f}\)}\end{center}
L'efficacité d'une pompe à chaleur est définie par :
\begin{center}\fcolorbox{red}{white}{\(e=-\dfrac{Q_c}{T_c}\)}\end{center}
Elle est majorée par l'efficacité de Carnot :
\begin{center}\fcolorbox{red}{white}{\(e_c=\dfrac{T_c}{T_c-T_f}\)}\end{center}

\end{document}
