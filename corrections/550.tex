\documentclass[a4paper]{article}
\usepackage[T1]{fontenc}
\usepackage[utf8]{inputenc}
\usepackage{lmodern}
\usepackage{amsmath,amssymb}
\usepackage[top=3cm,bottom=2cm,left=2cm,right=2cm]{geometry}
\usepackage{fancyhdr}
\usepackage{esvect,esint}
\usepackage{xcolor}
\usepackage{tikz,circuitikz}\usetikzlibrary{calc}

\parskip1em\parindent0pt\let\ds\displaystyle

\begin{document}

\pagestyle{fancy}
\fancyhf{}
\setlength{\headheight}{15pt}
\fancyhead[L]{Electrocinétique}\fancyhead[R]{Question 2}

% Énoncé
\begin{center}
	\large{\boldmath{\textbf{Définition convention génératrice}}}
\end{center}

% Correction


\begin{minipage}{0.4\linewidth}
  \begin{circuitikz}
    \draw (0,0) to (0.5,0) node[draw,circle,fill=black,inner sep=1pt] {} node[above] {A} to[generic,i>_=$I$] (3.5,0) node[draw,circle,fill=black,inner sep=1pt] {} node[above] {B} to (4,0);
    \draw [thick,<-] (3.5,0.5) to (0.5,0.5) ;
    \draw (2,0.5) node[anchor=south ]{$U$};
  \end{circuitikz}
\end{minipage}
\begin{minipage}{0.6\linewidth}
  $U = V_B-V_A$ (différence de potentiel)\\
  $U$ est appelée tension aux bornes du dipôle.\\
  $I$ est l'intensité du courant traversant le dipôle.
\end{minipage}
\end{document}
