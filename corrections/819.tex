\documentclass[a4paper]{article}
\usepackage[T1]{fontenc}
\usepackage[utf8]{inputenc}
\usepackage{lmodern}
\usepackage{amsmath,amssymb}
\usepackage[top=3cm,bottom=2cm,left=2cm,right=2cm]{geometry}
\usepackage{fancyhdr}
\usepackage{esvect,esint}
\usepackage{xcolor}
\usepackage{tikz}\usetikzlibrary{calc}

\parskip1em\parindent0pt\let\ds\displaystyle

\begin{document}

\pagestyle{fancy}
\fancyhf{}
\setlength{\headheight}{15pt}
\fancyhead[L]{SLCI}\fancyhead[R]{Question 4}

% Énoncé
\begin{center}
	\large{\boldmath{\textbf{Théorèmes des valeurs initiale et finale}}}
\end{center}

% Correction

On note $F$ la transformée de Laplace d'une fonction $f$. \\
Alors,
\begin{center}
\fcolorbox{red}{white}{
	\begin{tabular}{c}
    	\( \lim\limits_{t\rightarrow 0} f(t) = \lim\limits_{p\rightarrow+\infty} pF(p) \) \\
        \( \lim\limits_{t\rightarrow+\infty} f(t) = \lim\limits_{p\rightarrow 0} pF(p) \)
    \end{tabular}
}\end{center}

\end{document}
