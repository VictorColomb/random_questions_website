\documentclass[a4paper]{article}
\usepackage[T1]{fontenc}
\usepackage[utf8]{inputenc}
\usepackage{lmodern}
\usepackage{amsmath,amssymb}
\usepackage[top=3cm,bottom=2cm,left=2cm,right=2cm]{geometry}
\usepackage{fancyhdr}
\usepackage{esvect,esint}
\usepackage{xcolor}
\usepackage{tikz}\usetikzlibrary{calc}

\parskip1em\parindent0pt\let\ds\displaystyle

\begin{document}

\pagestyle{fancy}
\fancyhf{}
\setlength{\headheight}{15pt}
\fancyhead[L]{Electromagnétisme}\fancyhead[R]{Question 24}

% Énoncé
\begin{center}
	\large{\boldmath{\textbf{Champ sur l’axe d’une spire}}}
\end{center}

% Correction

Soit une spire circulaire de rayon \(a\), d'axe \(\vv{z}\) parcourue par un courant d'intensité \(I\). Soit \(M\) un point de l'axe de la spire, repéré l'angle \(\alpha\) avec lequel il "voit" la spire.\\
Alors le champ magnétique en \(M\) est donné par
\begin{center}
\fcolorbox{red}{white}{\(\vv{B}(M)=\dfrac{\mu_0I}{2a}\sin^3\alpha\vv{z}\)}
\end{center}

\end{document}
