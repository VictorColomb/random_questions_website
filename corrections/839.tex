\documentclass[a4paper]{article}
\usepackage[T1]{fontenc}
\usepackage[utf8]{inputenc}
\usepackage{lmodern}
\usepackage{amsmath,amssymb}
\usepackage[top=3cm,bottom=2cm,left=2cm,right=2cm]{geometry}
\usepackage{fancyhdr}
\usepackage{esvect,esint}
\usepackage{xcolor}
\usepackage{tikz}\usetikzlibrary{calc}

\parskip1em\parindent0pt\let\ds\displaystyle

\begin{document}

\pagestyle{fancy}
\fancyhf{}
\setlength{\headheight}{15pt}
\fancyhead[L]{Mécanique}\fancyhead[R]{Question 15}

% Énoncé
\begin{center}
	\large{\boldmath{\textbf{Définition de l'opérateur comoment $\otimes$
}}}
\end{center}

% Correction
Soient deux torseurs \( \left\{ \begin{matrix} \vv A \\ \vv B \end{matrix} \right\} \) et \( \left\{ \begin{matrix} \vv C \\ \vv D \end{matrix} \right\} \). \\
On définit leur comoment par
\begin{center}\fcolorbox{red}{white}{
	\( \left\{ \begin{matrix} \vv A \\ \vv B \end{matrix} \right\} \otimes \left\{ \begin{matrix} \vv C \\ \vv D \end{matrix} \right\} = \vv A \cdot \vv D + \vv B \cdot \vv C \)
}\end{center}


\end{document}
