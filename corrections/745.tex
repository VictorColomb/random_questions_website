\documentclass[a4paper]{article}
\usepackage[T1]{fontenc}
\usepackage[utf8]{inputenc}
\usepackage{lmodern}
\usepackage{amsmath,amssymb}
\usepackage[top=3cm,bottom=2cm,left=2cm,right=2cm]{geometry}
\usepackage{fancyhdr}
\usepackage{esvect,esint}
\usepackage{xcolor}
\usepackage{tikz}\usetikzlibrary{calc}

\parskip 1em\parindent 0pt

\begin{document}

\pagestyle{fancy}
\fancyhf{}
\setlength{\headheight}{15pt}
\fancyhead[L]{Chimie}\fancyhead[R]{Question 1}

% Énoncé
\begin{center}
	\large{\boldmath{\textbf{Définition grandeur molaire partielle}}}
\end{center}

% Correction

Soit \(Y\) une grandeur chimique associée à un mélange d'espèces dont on note \(n_1,\cdots,n_p\) les quantités de matière. \\
Alors la grandeur molaire partielle associée à \(Y\) et à l'espèce \(i\) est définie par :
\begin{center}
	\fcolorbox{red}{white}{\(\left. y_i=\dfrac{\partial Y}{\partial n_i}(T,P,(n_k)) \right|_{T,P,n_j\neq n_i}\)}
\end{center}

\end{document}
