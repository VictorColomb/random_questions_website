\documentclass[a4paper]{article}
\usepackage[T1]{fontenc}
\usepackage[utf8]{inputenc}
\usepackage{lmodern}
\usepackage{amsmath,amssymb}
\usepackage[top=3cm,bottom=2cm,left=2cm,right=2cm]{geometry}
\usepackage{fancyhdr}
\usepackage{esvect,esint}
\usepackage{xcolor}
\usepackage{tikz}\usetikzlibrary{calc}

\parskip1em\parindent0pt\let\ds\displaystyle

\begin{document}

\pagestyle{fancy}
\fancyhf{}
\setlength{\headheight}{15pt}
\fancyhead[L]{Thermodynamique}\fancyhead[R]{Question 3}

% Énoncé
\begin{center}
	\large{\boldmath{\textbf{Définition état thermodynamique, \\ état thermodynamique d’équilibre, \\ équilibre mécanique stable}}}
\end{center}

% Correction

Un état thermodynamique est un état qui peut être défini par un petit nombre de variables.

Un état thermodynamique d'équilibre est un état thermodynamique dont toutes les variables sont constantes dans le temps.

Un état d'équilibre mécanique stable correspond à un état dans lequel la pression intérieure est homogène et égale à la pression extérieure.

\end{document}
