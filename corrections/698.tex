\documentclass[a4paper]{article}
\usepackage[T1]{fontenc}
\usepackage[utf8]{inputenc}
\usepackage{lmodern}
\usepackage{amsmath,amssymb}
\usepackage[top=3cm,bottom=2cm,left=2cm,right=2cm]{geometry}
\usepackage{fancyhdr}
\usepackage{esvect,esint}
\usepackage{xcolor}
\usepackage{tikz}\usetikzlibrary{calc}

\parskip1em\parindent0pt\let\ds\displaystyle

\begin{document}

\pagestyle{fancy}
\fancyhf{}
\setlength{\headheight}{15pt}
\fancyhead[L]{Thermodynamique}\fancyhead[R]{Question 8}

% Énoncé
\begin{center}
	\large{\boldmath{\textbf{Détente de Joule Gay-Lussac}}}
\end{center}

% Correction

Un volume $V$ est isolé par des parois adiabatiques et est divisé en $V_1$ et $V_2$ par une paroi amovible. Le volume $V_1$ est occupé par un gaz à la pression $P$ et le volume $V_2$ est vide.\\
À l'instant $t=0$, on enlève la paroi centrale.\par
On considère le système composé des volumes $V_1$ et $V_2$ et de la paroi centrale : $\Sigma=$ \{gaz + vide\}.\\
Premier principe : $\Delta U_{\text{gaz}}=\Delta U_{\text{gaz}}+\Delta U_{\text{vide}}=\Delta U=W+Q$\\
Or $Q=0$ et $W=0$ car le système est isolé.\\
Donc $\Delta U=0$\par
Pour certains gaz, $T_F=T_I$. On dit qu'ils suivent la première loi de Joule.\\
C'est le cas des gaz parfaits.


\end{document}
