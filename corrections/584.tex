\documentclass[a4paper]{article}
\usepackage[T1]{fontenc}
\usepackage[utf8]{inputenc}
\usepackage{lmodern}
\usepackage{amsmath,amssymb}
\usepackage[top=3cm,bottom=2cm,left=2cm,right=2cm]{geometry}
\usepackage{fancyhdr}
\usepackage{esvect}
\usepackage{xcolor}
\usepackage{tikz,circuitikz}\usetikzlibrary{calc}

\parskip 1em\parindent 0pt

\begin{document}

\pagestyle{fancy}
\fancyhf{}
\setlength{\headheight}{15pt}
\fancyhead[L]{Electrocinétique}\fancyhead[R]{Question 36}

% Énoncé
\begin{center}
	\large{\boldmath{\textbf{Montage intégrateur sur série de Fourier}}}
\end{center}

% Correction

On note \(f_c\) la fréquence de coupure du montage intégrateur, qu'on suppose passe bas.\\
On note \(e=\displaystyle\sum_{n=0}^{+\infty} e_n\cos(2\pi nft+\varphi_n)\) la décomposition en série de Fourier de \(e\).\\
On note \(s\) le signal en sortie du filtre.

Si \(f\gg f_c\), alors
\begin{center}
\fcolorbox{red}{white}{\(s=\langle e \rangle+ f_c\displaystyle\int\left(e(t)-\langle e\rangle\right)\mathrm{d}t\)}
\end{center}
Si \(f\ll f_c\), alors
\begin{center}
\fcolorbox{red}{white}{\(s=e\)}
\end{center}


\end{document}
