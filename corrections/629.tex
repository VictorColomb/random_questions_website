\documentclass[a4paper]{article}
\usepackage[T1]{fontenc}
\usepackage[utf8]{inputenc}
\usepackage{lmodern}
\usepackage{amsmath,amssymb}
\usepackage[top=3cm,bottom=2cm,left=2cm,right=2cm]{geometry}
\usepackage{fancyhdr}
\usepackage{esvect}
\usepackage{xcolor}
\usepackage{tikz}\usetikzlibrary{calc}

\parskip 1em\parindent 0pt

\begin{document}

\pagestyle{fancy}
\fancyhf{}
\setlength{\headheight}{15pt}
\fancyhead[L]{Electromagnétisme}\fancyhead[R]{Question 14}

% Énoncé
\begin{center}
	\large{\boldmath{\textbf{Théorème de Coulomb}}}
\end{center}

% Correction

On considère un conducteur parfait et un vecteur $\vv n$ unitaire normal à sa surface en un point $M$.\par
\fcolorbox{red}{white}{ Alors, \( \vv E(M) = \dfrac{\sigma(M)}{\varepsilon_0} \vv n \) }

\end{document}
