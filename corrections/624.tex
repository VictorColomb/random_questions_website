\documentclass[a4paper]{article}
\usepackage[T1]{fontenc}
\usepackage[utf8]{inputenc}
\usepackage{lmodern}
\usepackage{amsmath,amssymb}
\usepackage[top=3cm,bottom=2cm,left=2cm,right=2cm]{geometry}
\usepackage{fancyhdr}
\usepackage{esvect,esint}
\usepackage{xcolor}
\usepackage{tikz}\usetikzlibrary{calc}

\parskip1em\parindent0pt\let\ds\displaystyle

\begin{document}

\pagestyle{fancy}
\fancyhf{}
\setlength{\headheight}{15pt}
\fancyhead[L]{Electromagnétisme}\fancyhead[R]{Question 9}

% Énoncé
\begin{center}
	\large{\boldmath{\textbf{Force agissant sur un dipôle}}}
\end{center}

% Correction

Soit un dipôle électrique \(\vv{p}\) placé en \(A\) plongé dans un champ électrique \(\vv{E_{\mathrm{ext}}}\).\\
La force exercée par le champ sur le dipôle est donnée par :\begin{center}\fcolorbox{red}{white}{\(\vv{F}=(\vv{p}\cdot\vv{\nabla})\vv{E_{\mathrm{ext}}}(A)\)}\end{center}
On a de plus :\begin{center}\fcolorbox{red}{white}{\(\vv{F}=\vv{\mathrm{grad}}(\vv{p}\cdot\vv{E_{\mathrm{ext}}})(A)\)}\end{center}




\end{document}
