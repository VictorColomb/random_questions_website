\documentclass[a4paper]{article}
\usepackage[T1]{fontenc}
\usepackage[utf8]{inputenc}
\usepackage{lmodern}
\usepackage{amsmath,amssymb}
\usepackage[top=3cm,bottom=2cm,left=2cm,right=2cm]{geometry}
\usepackage{fancyhdr}
\usepackage{esvect}
\usepackage{xcolor}
\usepackage{tikz}\usetikzlibrary{calc}

\parskip 1em\parindent 0pt

\begin{document}

\pagestyle{fancy}
\fancyhf{}
\setlength{\headheight}{15pt}
\fancyhead[L]{Mécanique quantique}\fancyhead[R]{Question 8}

% Énoncé
\begin{center}
	\large{\boldmath{\textbf{Action d’un électron}}}
\end{center}

% Correction

On rapelle que \( [A] = [p][L] \).\\[2pt]
C'est pourquoi on a \fcolorbox{red}{white}{\( A = mva \)} où \( a \) est le rayon de l'électron.

\end{document}
