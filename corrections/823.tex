\documentclass[a4paper]{article}
\usepackage[T1]{fontenc}
\usepackage[utf8]{inputenc}
\usepackage{lmodern}
\usepackage{amsmath,amssymb}
\usepackage[top=3cm,bottom=2cm,left=2cm,right=2cm]{geometry}
\usepackage{fancyhdr}
\usepackage{esvect,esint}
\usepackage{xcolor}
\usepackage{tikz}\usetikzlibrary{calc}

\parskip1em\parindent0pt\let\ds\displaystyle

\begin{document}

\pagestyle{fancy}
\fancyhf{}
\setlength{\headheight}{15pt}
\fancyhead[L]{SLCI}\fancyhead[R]{Question 8}

% Énoncé
\begin{center}
	\large{\boldmath{\textbf{Définition des marges de gain et de phase}}}
\end{center}

% Correction

Si \( \omega_G \) est tel que \( \mathrm{arg}(\mathrm{FTBO}(j\omega_G)) = -180^\circ \), on définit la marge de gain comme
\begin{center}\fcolorbox{red}{white}{
	\( M_G = -G_{\mathrm{dB}}(\omega_G) \)
}\end{center}

Si \( \omega_\varphi \) est tel que \( |\mathrm{FTBO}(j\omega_\varphi)| = 1 \), on définit la marge de phase comme
\begin{center}\fcolorbox{red}{white}{
	\( M_\varphi = 180^\circ + \mathrm{arg}(\mathrm{FTBO}(j\omega_\varphi)) \)
}\end{center}


\end{document}
