\documentclass[a4paper]{article}
\usepackage[T1]{fontenc}
\usepackage[utf8]{inputenc}
\usepackage{lmodern}
\usepackage{amsmath,amssymb}
\usepackage[top=3cm,bottom=2cm,left=2cm,right=2cm]{geometry}
\usepackage{fancyhdr}
\usepackage{esvect}
\usepackage{xcolor}
\usepackage{tikz}\usetikzlibrary{calc}

\parskip 1em\parindent 0pt

\begin{document}

\pagestyle{fancy}
\fancyhf{}
\setlength{\headheight}{15pt}
\fancyhead[L]{Chimie}\fancyhead[R]{Question 26}

% Énoncé
\begin{center}
	\large{\boldmath{\textbf{Définition quotient de réaction}}}
\end{center}

% Correction

Pour la réaction \(\sum_i\alpha_iR_i \leftrightarrow \sum_j\beta_jP_j\), le coefficient de réaction est défini par :
\begin{center}
\fcolorbox{red}{white}{\(\dfrac{\prod_jb_j^{\beta_j}}{\prod_ia_i^{\alpha_i}}\)}
\end{center}
où les \(a_i,b_j\) sont les activités respectives des espèces \(A_i,B_j\).


\end{document}
