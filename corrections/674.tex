\documentclass[a4paper]{article}
\usepackage[T1]{fontenc}
\usepackage[utf8]{inputenc}
\usepackage{lmodern}
\usepackage{amsmath,amssymb}
\usepackage[top=3cm,bottom=2cm,left=2cm,right=2cm]{geometry}
\usepackage{fancyhdr}
\usepackage{esvect,esint}
\usepackage{xcolor}
\usepackage{tikz}\usetikzlibrary{calc}

\parskip1em\parindent0pt\let\ds\displaystyle

\begin{document}

\pagestyle{fancy}
\fancyhf{}
\setlength{\headheight}{15pt}
\fancyhead[L]{Electromagnétisme}\fancyhead[R]{Question 59}

% Énoncé
\begin{center}
	\large{\boldmath{\textbf{Relation de passage du champ magnétique}}}
\end{center}

% Correction

Soit une distribution surfacique de courant \(\vv{j_s}\), le champ magnétique de chaque côté de cette distribution vérifie :\begin{center}\fcolorbox{red}{white}{\(\vv{B_2}-\vv{B_1}=\mu_0\vv{j_s}\wedge\vv{n_{12}}\)}\end{center}où \(\vv{n_{12}}\) est le vecteur unitaire normal à la surface allant de \(1\) vers \(2\).


\end{document}
