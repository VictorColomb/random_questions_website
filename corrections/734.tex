\documentclass[a4paper]{article}
\usepackage[T1]{fontenc}
\usepackage[utf8]{inputenc}
\usepackage{lmodern}
\usepackage{amsmath,amssymb}
\usepackage[top=3cm,bottom=2cm,left=2cm,right=2cm]{geometry}
\usepackage{fancyhdr}
\usepackage{esvect,esint}
\usepackage{xcolor}
\usepackage{tikz}\usetikzlibrary{calc}

\parskip1em\parindent0pt\let\ds\displaystyle

\begin{document}

\pagestyle{fancy}
\fancyhf{}
\setlength{\headheight}{15pt}
\fancyhead[L]{Thermodynamique}\fancyhead[R]{Question 44}

% Énoncé
\begin{center}
	\large{\boldmath{\textbf{Expression locale de l’équilibre hydrostatique \\ dans un champ de pesanteur uniforme pour un fluide homogène}}}
\end{center}

% Correction

On rappelle que l'expression locale de l'équilibre hydrostatique donne : \begin{center}\(\rho\vv{g}=\vv{\mathrm{grad}}P\)\end{center}
Pour un fluide homogène dans un champ de pesanteur uniforme \(\vv{g}=-g\vv{z}\), on a : 
\begin{center}\(\vv{\mathrm{grad}}(-\rho gz)=\vv{\mathrm{grad}}P\)\end{center}
D'où : \begin{center}\fcolorbox{red}{white}{\(P+\rho gz=\mathrm{cste}\)}\end{center}




\end{document}
