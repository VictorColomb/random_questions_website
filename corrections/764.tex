\documentclass[a4paper]{article}
\usepackage[T1]{fontenc}
\usepackage[utf8]{inputenc}
\usepackage{lmodern}
\usepackage{amsmath,amssymb}
\usepackage[top=3cm,bottom=2cm,left=2cm,right=2cm]{geometry}
\usepackage{fancyhdr}
\usepackage{esvect,esint}
\usepackage{xcolor}
\usepackage{tikz}\usetikzlibrary{calc}

\parskip1em\parindent0pt\let\ds\displaystyle

\begin{document}

\pagestyle{fancy}
\fancyhf{}
\setlength{\headheight}{15pt}
\fancyhead[L]{Chimie}\fancyhead[R]{Question 20}

% Énoncé
\begin{center}
	\large{\boldmath{\textbf{Relations de Helmholtz}}}
\end{center}

% Correction

\(1^{\text{ère}}\) relation de Helmholtz :\begin{center}\fcolorbox{red}{white}{\(\dfrac{\mathrm{d}\Delta_rG^{\circ}}{\mathrm{d}T}=-\Delta_rS^{\circ}\)}\end{center}
\(2^{\text{ème}}\) relation de Helmholtz :\begin{center}\fcolorbox{red}{white}{\(\dfrac{\mathrm{d}}{\mathrm{d}T}\left(\dfrac{\Delta_rG^{\circ}}{T}\right)=-\dfrac{\Delta_rH^{\circ}}{T^2}\)}\end{center}



\end{document}
