\documentclass[a4paper]{article}
\usepackage[T1]{fontenc}
\usepackage[utf8]{inputenc}
\usepackage{lmodern}
\usepackage{amsmath,amssymb}
\usepackage[top=3cm,bottom=2cm,left=2cm,right=2cm]{geometry}
\usepackage{fancyhdr}
\usepackage{esvect}
\usepackage{xcolor}
\usepackage{tikz}\usetikzlibrary{calc}

\parskip 1em\parindent 0pt

\begin{document}

\pagestyle{fancy}
\fancyhf{}
\setlength{\headheight}{15pt}
\fancyhead[L]{Mécanique}\fancyhead[R]{Question 8}

% Énoncé
\begin{center}
	\large{\boldmath{\textbf{Formule de dérivation composée, \\ définition du vecteur rotation}}}
\end{center}

% Correction

Soient un premier référentiel \( \mathcal{R}_1 \), de point fixe \( O_1 \), et un deuxième \( \mathcal{R}_2 \) en mouvement par rapport au premier, de point fixe \( O_2 \). \\
Dans le cas le plus général, il y a trois rotations entre \( \mathcal{R}_2 \) et \( \mathcal{R}_1 \), qu'on note \( \alpha_1\vv*h1 \), \( \alpha_2\vv*h2 \) et \( \alpha_3\vv*h3 \). \\
On pose alors \( \vv\Omega = \dot{\alpha_1}\vv*h1 + \dot{\alpha_2}\vv*h2 + \dot{\alpha_3}\vv*h3 \), vecteur rotation du référentiel \( \mathcal{R}_2 \) par rapport au référentiel \( \mathcal{R}_1 \).

On a alors
\begin{center}
	\fcolorbox{red}{white}{ \( \left. \dfrac{\mathrm{d}\vv u}{\mathrm{d}t} \right|_{\mathcal{R}_2} = \left. \dfrac{\mathrm{d}\vv u}{\mathrm{d}t} \right|_{\mathcal{R}_1} + \left. \dfrac{\mathrm{d}\vv{O_1O_2}}{\mathrm{d}t} \right|_{\mathcal{R}_1} + \vv\Omega\wedge\vv u \) }
\end{center}


\end{document}
