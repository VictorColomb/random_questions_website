\documentclass[a4paper]{article}
\usepackage[T1]{fontenc}
\usepackage[utf8]{inputenc}
\usepackage{lmodern}
\usepackage{amsmath,amssymb}
\usepackage[top=3cm,bottom=2cm,left=2cm,right=2cm]{geometry}
\usepackage{fancyhdr}
\usepackage{esvect}
\usepackage{xcolor}
\usepackage{tikz}\usetikzlibrary{calc}

\parskip 1em\parindent 0pt

\begin{document}

\pagestyle{fancy}
\fancyhf{}
\setlength{\headheight}{15pt}
\fancyhead[L]{Mécanique}\fancyhead[R]{Question 11}

% Énoncé
\begin{center}
	\large{\boldmath{\textbf{Définition de la quantité de mouvement et de l’énergie cinétique \\ d’un point matériel dans un repère}}}
\end{center}

% Correction

Soit \(\mathcal{R}\) un référentiel quelconque et \(M\) un point matériel de masse \(m\).\\
Sa quantité de mouvement dans \(\mathcal{R}\) est définie par \fcolorbox{red}{white}{\(\vv{p}(M\in\mathcal{R})=m\vv{v}(M\in\mathcal{R})\)}\\
Son énergie cinétique dans \(\mathcal{R}\) est définie par \fcolorbox{red}{white}{\(E_c^{\mathcal{R}}=\dfrac{1}{2}m\vv{v}(M\in\mathcal{R})^2\)}

\end{document}
