\documentclass[a4paper]{article}
\usepackage[T1]{fontenc}
\usepackage[utf8]{inputenc}
\usepackage{lmodern}
\usepackage{amsmath,amssymb}
\usepackage[top=3cm,bottom=2cm,left=2cm,right=2cm]{geometry}
\usepackage{fancyhdr}
\usepackage{esvect,esint}
\usepackage{xcolor}
\usepackage{tikz}\usetikzlibrary{calc}

\parskip1em\parindent0pt\let\ds\displaystyle

\begin{document}

\pagestyle{fancy}
\fancyhf{}
\setlength{\headheight}{15pt}
\fancyhead[L]{Electromagnétisme}\fancyhead[R]{Question 36}

% Énoncé
\begin{center}
	\large{\boldmath{\textbf{Densité d’énergie électromagnétique}}}
\end{center}

% Correction

La densité d'énergie électromagnétique est la somme des densités d'énergie électrique et magnétique.
\begin{center}
\fcolorbox{red}{white}{\(U_{em}=U_e+U_m=\dfrac{\varepsilon_0\vv{E}^2}{2}+\dfrac{\vv{B}^2}{2\mu_0}\)}
\end{center}

\end{document}
