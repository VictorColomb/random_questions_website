\documentclass[a4paper]{article}
\usepackage[T1]{fontenc}
\usepackage[utf8]{inputenc}
\usepackage{lmodern}
\usepackage{amsmath,amssymb}
\usepackage[top=3cm,bottom=2cm,left=2cm,right=2cm]{geometry}
\usepackage{fancyhdr}
\usepackage{esvect}
\usepackage{xcolor}
\usepackage{tikz}\usetikzlibrary{calc}

\parskip 1em\parindent 0pt

\begin{document}

\pagestyle{fancy}
\fancyhf{}
\setlength{\headheight}{15pt}
\fancyhead[L]{Thermodynamique}\fancyhead[R]{Question 23}

% Énoncé
\begin{center}
	\large{\boldmath{\textbf{Définition entropie d’échange, de création}}}
\end{center}

% Correction

On définit l'entropie d'échange par \fcolorbox{red}{white}{\( S_e = \dfrac{Q}{T_{th}} \)} où \( T_{th} \) est la température du thermostat. \\
On peut avoir \( S_e = \displaystyle\sum_i \dfrac{Q_i}{T_i} \) avec plusieurs thermostats \( (T_i) \).

Alors \( \Delta S = S_e + S_c \) où \fcolorbox{red}{white}{\( S_c \geqslant 0 \)} est appelée entropie de création.

\end{document}
