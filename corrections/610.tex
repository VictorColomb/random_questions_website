\documentclass[a4paper]{article}
\usepackage[T1]{fontenc}
\usepackage[utf8]{inputenc}
\usepackage{lmodern}
\usepackage{amsmath,amssymb}
\usepackage[top=3cm,bottom=2cm,left=2cm,right=2cm]{geometry}
\usepackage{fancyhdr}
\usepackage{esvect,esint}
\usepackage{xcolor}
\usepackage{tikz}\usetikzlibrary{calc}

\parskip 1em\parindent 0pt

\begin{document}

\pagestyle{fancy}
\fancyhf{}
\setlength{\headheight}{15pt}
\fancyhead[L]{Mécanique}\fancyhead[R]{Question 23}

% Énoncé
\begin{center}
	\large{\boldmath{\textbf{Définition moment d’inertie d’un solide}}}
\end{center}

% Correction

Pour un solide de volume \(V\) et de masse volumique \(\rho\) tournant autour d'un axe \(\Delta\), on définit son moment d'inertie en \(\Delta\) par :
\begin{center}
\fcolorbox{red}{white}{\(J_{\Delta}=\displaystyle\iiint_V\rho\vv{HP}^2\mathrm{d}V(P)\)}
\end{center}
où pour tout point \(P\), \(H\) est le projeté orthogonal de \(P\) sur \(\Delta\).

\end{document}
