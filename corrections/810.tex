\documentclass[a4paper]{article}
\usepackage[T1]{fontenc}
\usepackage[utf8]{inputenc}
\usepackage{lmodern}
\usepackage{amsmath,amssymb}
\usepackage[top=3cm,bottom=2cm,left=2cm,right=2cm]{geometry}
\usepackage{fancyhdr}
\usepackage{esvect}
\usepackage{xcolor}
\usepackage{tikz}\usetikzlibrary{calc}

\parskip 1em\parindent 0pt

\begin{document}

\pagestyle{fancy}
\fancyhf{}
\setlength{\headheight}{15pt}
\fancyhead[L]{Mécanique quantique}\fancyhead[R]{Question 16}

% Énoncé
\begin{center}
	\large{\boldmath{\textbf{Discontinuité de $V$ et continuité de $\varphi$ et $\dfrac{\mathrm{d}\varphi}{\mathrm{d}x}$}}}
\end{center}

% Correction

\(\varphi\) est toujours continue.\\
En un point de continuité ou de discontinuité finie de \(V\), \(\dfrac{\mathrm{d}\varphi}{\mathrm{d}x}\) est continue.\\
On ne peut par contre pas prévoir de la continuité de \(\dfrac{\mathrm{d}\varphi}{\mathrm{d}x}\) en un point de discontinuité infini de \(V\).


\end{document}
