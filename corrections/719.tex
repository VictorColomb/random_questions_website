\documentclass[a4paper]{article}
\usepackage[T1]{fontenc}
\usepackage[utf8]{inputenc}
\usepackage{lmodern}
\usepackage{amsmath,amssymb}
\usepackage[top=3cm,bottom=2cm,left=2cm,right=2cm]{geometry}
\usepackage{fancyhdr}
\usepackage{esvect,esint}
\usepackage{xcolor}
\usepackage{tikz}\usetikzlibrary{calc}

\parskip1em\parindent0pt\let\ds\displaystyle

\begin{document}

\pagestyle{fancy}
\fancyhf{}
\setlength{\headheight}{15pt}
\fancyhead[L]{Thermodynamique}\fancyhead[R]{Question 29}

% Énoncé
\begin{center}
	\large{\boldmath{\textbf{Impossibilité du moteur monotherme}}}
\end{center}

% Correction

Par l'absurde, supposons qu'il existe un moteur utilisant une unique source de chaleur.\\
Ainsi, sur un cycle, \(\left\lbrace \begin{array}{l} W+Q=0 \\ \dfrac{Q}{T}\leqslant 0 \end{array}\right.\)\\
Donc \(Q\leqslant0\) donc \(W\geqslant0\) or \(W<0\) par définition. Absurde.
\begin{center}
\fcolorbox{red}{white}{Donc le moteur monotherme est impossible.}
\end{center}


\end{document}
