\documentclass[a4paper]{article}
\usepackage[T1]{fontenc}
\usepackage[utf8]{inputenc}
\usepackage{lmodern}
\usepackage{amsmath,amssymb}
\usepackage[top=3cm,bottom=2cm,left=2cm,right=2cm]{geometry}
\usepackage{fancyhdr}
\usepackage{esvect,esint}
\usepackage{xcolor}
\usepackage{tikz,circuitikz}\usetikzlibrary{calc}

\parskip1em\parindent0pt\let\ds\displaystyle

\begin{document}

\pagestyle{fancy}
\fancyhf{}
\setlength{\headheight}{15pt}
\fancyhead[L]{Electrocinétique}\fancyhead[R]{Question 7}

% Énoncé
\begin{center}
	\large{\boldmath{\textbf{Définition d’une source réelle de courant }}}
\end{center}

% Correction


Pour une source réelle de courant, on a : $I=I_g-G(V_A-V_B)$, en posant $G=\dfrac{1}{R}$\\
  \begin{minipage}{0.2\linewidth}
    Représentation :
  \end{minipage}
  \begin{minipage}{0.4\linewidth}
    \begin{circuitikz}
      \draw (3,0) node[right] {B} to[short,*-] (0,0) to[isource=$I_g$] (0,2) -- (2,2) to[short,i>^=$I$,-*] (3,2) node[right] {A};
      \draw (2,0) to[R=$R$] (2,2);
    \end{circuitikz}
  \end{minipage}
\end{document}
