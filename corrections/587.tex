\documentclass[a4paper]{article}
\usepackage[T1]{fontenc}
\usepackage[utf8]{inputenc}
\usepackage{lmodern}
\usepackage{amsmath,amssymb}
\usepackage[top=3cm,bottom=2cm,left=2cm,right=2cm]{geometry}
\usepackage{fancyhdr}
\usepackage{esvect,esint}
\usepackage{xcolor}
\usepackage{tikz}\usetikzlibrary{calc}

\parskip1em\parindent0pt\let\ds\displaystyle

\begin{document}

\pagestyle{fancy}
\fancyhf{}
\setlength{\headheight}{15pt}
\fancyhead[L]{Electrocinétique}\fancyhead[R]{Question 39}

% Énoncé
\begin{center}
	\large{\boldmath{\textbf{Graphe $U-I$ d’une source réelle de tension}}}
\end{center}

% Correction

Pour une source réelle de tension, on a :\begin{center}\fcolorbox{red}{white}{\(U=E_g-RI\)}\end{center}
\begin{center}
    \begin{tikzpicture}[scale=2]
        \draw[thick,->](-2.5,0)to(2.5,0)node[anchor=north west]{\(U\)};
        \draw[thick,->](0,-2.5)to(0,2.5)node[anchor=south east]{\(I\)};
        \draw (-0.75,2)to(2,-0.75);
        \draw(1.33,0)node[anchor=north east ]{\(E_g\)};
        \draw(0.67,0.67)node[anchor=south west]{\(\dfrac{1}{R}\)};
    \end{tikzpicture}
\end{center}

\end{document}
