\documentclass[a4paper]{article}
\usepackage[T1]{fontenc}
\usepackage[utf8]{inputenc}
\usepackage{lmodern}
\usepackage{amsmath,amssymb}
\usepackage[top=3cm,bottom=2cm,left=2cm,right=2cm]{geometry}
\usepackage{fancyhdr}
\usepackage{esvect,esint}
\usepackage{xcolor}
\usepackage{tikz}\usetikzlibrary{calc}

\parskip1em\parindent0pt\let\ds\displaystyle

\begin{document}

\pagestyle{fancy}
\fancyhf{}
\setlength{\headheight}{15pt}
\fancyhead[L]{Mécanique}\fancyhead[R]{Question 17}

% Énoncé
\begin{center}
	\large{\boldmath{\textbf{Définition force conservative}}}
\end{center}

% Correction

Une force \(\vv{F}\) est dite conservative si elle dérive d'une énergie potentielle, soit si il existe \(E_p\) telle que \fcolorbox{red}{white}{\(\vv{F}=-\vv{\mathrm{grad}}E_p\)}

\end{document}
