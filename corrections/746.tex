\documentclass[a4paper]{article}
\usepackage[T1]{fontenc}
\usepackage[utf8]{inputenc}
\usepackage{lmodern}
\usepackage{amsmath,amssymb}
\usepackage[top=3cm,bottom=2cm,left=2cm,right=2cm]{geometry}
\usepackage{fancyhdr}
\usepackage{esvect,esint}
\usepackage{xcolor}
\usepackage{tikz}\usetikzlibrary{calc}

\parskip1em\parindent0pt\let\ds\displaystyle

\begin{document}

\pagestyle{fancy}
\fancyhf{}
\setlength{\headheight}{15pt}
\fancyhead[L]{Chimie}\fancyhead[R]{Question 2}

% Énoncé
\begin{center}
	\large{\boldmath{\textbf{Théorème d’Euler}}}
\end{center}

% Correction

Soit une grandeur chimique \(Z\) associée à un mélange d'espèces \(i\) de quantité de matière \(n_i\).\\
On note \(z_i=\left.\dfrac{\partial Z}{\partial n_i}\right|_{P,T,n_j\neq n_i}\)\\
Alors :\begin{center}\fcolorbox{red}{white}{\(Z=\sum\limits_i z_i n_i\)}\end{center}



\end{document}
