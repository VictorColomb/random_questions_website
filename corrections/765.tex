\documentclass[a4paper]{article}
\usepackage[T1]{fontenc}
\usepackage[utf8]{inputenc}
\usepackage{lmodern}
\usepackage{amsmath,amssymb}
\usepackage[top=3cm,bottom=2cm,left=2cm,right=2cm]{geometry}
\usepackage{fancyhdr}
\usepackage{esvect,esint}
\usepackage{xcolor}
\usepackage{tikz}\usetikzlibrary{calc}

\parskip1em\parindent0pt\let\ds\displaystyle

\begin{document}

\pagestyle{fancy}
\fancyhf{}
\setlength{\headheight}{15pt}
\fancyhead[L]{Chimie}\fancyhead[R]{Question 21}

% Énoncé
\begin{center}
	\large{\boldmath{\textbf{Loi de Hess}}}
\end{center}

% Correction

Pour une grandeur chimique \(Y\) associée à la réaction \(\sum_i\nu_iB_i\), on a :
\begin{center}\fcolorbox{red}{white}{\(\Delta_rY^{\circ}=\sum_i\nu_i\Delta_fY_i^{\circ}\)}\end{center}où \(\Delta_fY_i^{\circ}\) est la grandeur standard de formation associées à \(Y\) et à l'espèce \(i\).

\end{document}
