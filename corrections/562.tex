\documentclass[a4paper]{article}
\usepackage[T1]{fontenc}
\usepackage[utf8]{inputenc}
\usepackage{lmodern}
\usepackage{amsmath,amssymb}
\usepackage[top=3cm,bottom=2cm,left=2cm,right=2cm]{geometry}
\usepackage{fancyhdr}
\usepackage{esvect}
\usepackage{xcolor}
\usepackage{tikz,circuitikz}\usetikzlibrary{calc}

\parskip 1em\parindent 0pt

\begin{document}

\pagestyle{fancy}
\fancyhf{}
\setlength{\headheight}{15pt}
\fancyhead[L]{Electrocinétique}\fancyhead[R]{Question 14}

% Énoncé
\begin{center}
	\large{\boldmath{\textbf{Théorème de Thévenin}}}
\end{center}

% Correction

\begin{minipage}{.55\linewidth}
Toute association de dipôles linéaires peut être assimilée à un dipôle de la force ci-contre. \\
On trouve la résistance à vide en passivant toutes les sources de tension et de courant dans le circuit. \\
On trouve force électro-motrice en calculant la tension aux bornes du dipôle à courant entrant nul.
\end{minipage}
\hspace{.05\linewidth}
\begin{minipage}{.4\linewidth}
  \centering
  \begin{circuitikz}
    \draw (1,3) to[short] (0,3) to[R=$R$] (0,1.5) to[V=$E$] (0,0) -- (1,0);
  \end{circuitikz}
\end{minipage}

\end{document}
