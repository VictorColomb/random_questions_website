\documentclass[a4paper]{article}
\usepackage[T1]{fontenc}
\usepackage[utf8]{inputenc}
\usepackage{lmodern}
\usepackage{amsmath,amssymb}
\usepackage[top=3cm,bottom=2cm,left=2cm,right=2cm]{geometry}
\usepackage{fancyhdr}
\usepackage{esvect,esint}
\usepackage{xcolor}
\usepackage{tikz,circuitikz}\usetikzlibrary{calc}

\parskip1em\parindent0pt\let\ds\displaystyle

\begin{document}

\pagestyle{fancy}
\fancyhf{}
\setlength{\headheight}{15pt}
\fancyhead[L]{Electrocinétique}\fancyhead[R]{Question 17}

% Énoncé
\begin{center}
	\large{\boldmath{\textbf{Définition circuit linéaire}}}
\end{center}

% Correction

Un circuit est dit linéaire s'il obéit au théorème de superposition. En particulier, les circuit composés uniquement de résistances, de sources de tension et de sources de courant sont linéaires.

\end{document}
