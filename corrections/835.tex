\documentclass[a4paper]{article}
\usepackage[T1]{fontenc}
\usepackage[utf8]{inputenc}
\usepackage{lmodern}
\usepackage{amsmath,amssymb}
\usepackage[top=3cm,bottom=2cm,left=2cm,right=2cm]{geometry}
\usepackage{fancyhdr}
\usepackage{esvect,esint}
\usepackage{xcolor}
\usepackage{tikz}\usetikzlibrary{calc}

\parskip1em\parindent0pt\let\ds\displaystyle

\begin{document}

\pagestyle{fancy}
\fancyhf{}
\setlength{\headheight}{15pt}
\fancyhead[L]{Mécanique}\fancyhead[R]{Question 11}

% Énoncé
\begin{center}
	\large{\boldmath{\textbf{Définition de l'énergie cinétique}}}
\end{center}

% Correction

\abovedisplayskip0pt

On considère un solide \( S \). On définit son énergie cinétique par :
\begin{center}
\fcolorbox{red}{white}{%
	\begin{minipage}[t]{.65\linewidth}
    	\begin{align*}
        	T(S / \mathrm{R}) =&\ \dfrac{1}{2} \{\mathcal{C}_{S / \mathcal{R}}\} \otimes \{\mathcal{V}_{S / \mathcal{R}}\} \\
            =&\ \dfrac{1}{2} \vv V(A\in S / \mathcal{R}) \cdot m_S\vv V(G\in S / \mathcal{R}) + \dfrac{1}{2} \vv\Omega(S / \mathcal{R}) \cdot \vv\sigma(A,S / \mathcal{R})
        \end{align*}
    \end{minipage}
}\end{center}

\end{document}
