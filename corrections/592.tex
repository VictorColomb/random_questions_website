\documentclass[a4paper]{article}
\usepackage[T1]{fontenc}
\usepackage[utf8]{inputenc}
\usepackage{lmodern}
\usepackage{amsmath,amssymb}
\usepackage[top=3cm,bottom=2cm,left=2cm,right=2cm]{geometry}
\usepackage{fancyhdr}
\usepackage{esvect,esint}
\usepackage{xcolor}
\usepackage{tikz}\usetikzlibrary{calc}

\parskip1em\parindent0pt\let\ds\displaystyle

\begin{document}

\pagestyle{fancy}
\fancyhf{}
\setlength{\headheight}{15pt}
\fancyhead[L]{Mécanique}\fancyhead[R]{Question 5}

% Énoncé
\begin{center}
	\large{\boldmath{\textbf{Gradient en coordonnées cartésiennes, cylindriques et sphériques}}}
\end{center}

% Correction

Soit \(f\) une fonction de l'espace.\\
En coordonnées cartésiennes, on a :
\begin{center}
\(\vv{\mathrm{grad}}f = \left| \begin{array}{l}
\dfrac{\partial f}{\partial x} \\[5pt]
\dfrac{\partial f}{\partial y} \\[5pt]
\dfrac{\partial f}{\partial z} \end{array} \right.
\)
\end{center}

En coordonnées cylindro-polaires, on a :\begin{center}
\( \vv{\mathrm{grad}}f = \left| \begin{array}{l}
\dfrac{\partial f}{\partial r} \\[5pt]
\dfrac{1}{r}\dfrac{\partial f}{\partial \theta} \\[5pt]
\dfrac{\partial f}{\partial z} \end{array} \right. \)
\end{center}

En coordonnées sphériques, on a :\begin{center}
\( \vv{\mathrm{grad}}f = \left| \begin{array}{l}
\dfrac{\partial f}{\partial r} \\[5pt]
\dfrac{1}{r}\dfrac{\partial f}{\partial \theta} \\[5pt]
\dfrac{1}{r\sin\theta}\dfrac{\partial f}{\partial \varphi} \end{array} \right. \)
\end{center}






\end{document}
