\documentclass[a4paper]{article}
\usepackage[T1]{fontenc}
\usepackage[utf8]{inputenc}
\usepackage{lmodern}
\usepackage{amsmath,amssymb}
\usepackage[top=3cm,bottom=2cm,left=2cm,right=2cm]{geometry}
\usepackage{fancyhdr}
\usepackage{esvect,esint}
\usepackage{xcolor}
\usepackage{tikz}\usetikzlibrary{calc}

\parskip1em\parindent0pt\let\ds\displaystyle

\begin{document}

\pagestyle{fancy}
\fancyhf{}
\setlength{\headheight}{15pt}
\fancyhead[L]{Electromagnétisme}\fancyhead[R]{Question 19}

% Énoncé
\begin{center}
	\large{\boldmath{\textbf{Définition de la capacité}}}
\end{center}

% Correction

Pour deux surfaces métalliques en vis à vis aux potentiels respectifs \(V_1\) et \(V_2\), la charge sur l'une est l'opposée de la charge sur l'autre et en notant \(Q\) cette charge, on définit la capacité de condensateur ainsi formé par :\begin{center}
\fcolorbox{red}{white}{\(C=\left|\dfrac{Q}{V_1-V_2}\right|\)}\end{center}

\end{document}
