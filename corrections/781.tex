\documentclass[a4paper]{article}
\usepackage[T1]{fontenc}
\usepackage[utf8]{inputenc}
\usepackage{lmodern}
\usepackage{amsmath,amssymb}
\usepackage[top=3cm,bottom=2cm,left=2cm,right=2cm]{geometry}
\usepackage{fancyhdr}
\usepackage{esvect,esint}
\usepackage{xcolor}
\usepackage{tikz}\usetikzlibrary{calc}

\parskip1em\parindent0pt\let\ds\displaystyle

\begin{document}

\pagestyle{fancy}
\fancyhf{}
\setlength{\headheight}{15pt}
\fancyhead[L]{Chimie}\fancyhead[R]{Question 37}

% Énoncé
\begin{center}
	\large{\boldmath{\textbf{Valeur de $I$ en fonction de $v$ dans une électrode}}}
\end{center}

% Correction

Si \(I\) est l'intensité du courant orienté depuis le circuit vers l'électrode et \(v\) est la vitesse molaire de réaction à l'électrode, alors si cette réaction met en jeu \(n\) électrons, la valeur absolue de l'intensité du courant dans l'électrode vaut :\begin{center}
\fcolorbox{red}{white}{\(|I|=nFv\)}\end{center}
avec \(I\) positif si l'électrode est une anode et négatif si l'électrode est une cathode.\\
\emph{Rappel} Constante de Faraday: \(F = e \mathcal{N}_A\)



\end{document}
