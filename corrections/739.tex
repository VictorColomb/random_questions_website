\documentclass[a4paper]{article}
\usepackage[T1]{fontenc}
\usepackage[utf8]{inputenc}
\usepackage{lmodern}
\usepackage{amsmath,amssymb}
\usepackage[top=3cm,bottom=2cm,left=2cm,right=2cm]{geometry}
\usepackage{fancyhdr}
\usepackage{esvect,esint}
\usepackage{xcolor}
\usepackage{tikz}\usetikzlibrary{calc}

\parskip1em\parindent0pt\let\ds\displaystyle

\begin{document}

\pagestyle{fancy}
\fancyhf{}
\setlength{\headheight}{15pt}
\fancyhead[L]{Thermodynamique}\fancyhead[R]{Question 49}

% Énoncé
\begin{center}
	\large{\boldmath{\textbf{Définition fonction de partition $Z$}}}
\end{center}

% Correction

Pour une particule, la probabilité de se trouver dans l'état d'énergie \(E\) est proportionnelle à \(\exp(-\dfrac{E}{k_BT})\).\\
On note alors \(\dfrac{1}{Z}\) la constante de proportionnalité.\\
\(Z\) est appelée fonction de partition et vaut \fcolorbox{red}{white}{\(Z=\sum\limits_{E \,possible}\exp(-\dfrac{E}{k_BT})\)} par normalisation.


\end{document}
