\documentclass[a4paper]{article}
\usepackage[T1]{fontenc}
\usepackage[utf8]{inputenc}
\usepackage{lmodern}
\usepackage{amsmath,amssymb}
\usepackage[top=3cm,bottom=2cm,left=2cm,right=2cm]{geometry}
\usepackage{fancyhdr}
\usepackage{esvect}
\usepackage{xcolor}
\usepackage{tikz}\usetikzlibrary{calc}

\parskip 1em\parindent 0pt

\begin{document}

\pagestyle{fancy}
\fancyhf{}
\setlength{\headheight}{15pt}
\fancyhead[L]{Thermodynamique}\fancyhead[R]{Question 45}

% Énoncé
\begin{center}
	\large{\boldmath{\textbf{Poussée d’Archimède}}}
\end{center}

% Correction

\emph{"Tout corps plongé dans un liquide subit, de la part de celui-ci, une poussée exercée du bas vers le haut et égale, en intensité, au poids du volume de liquide déplacé."}\quad-Archimède\\
\fcolorbox{red}{white}{\(\vv{\Pi}=-\rho_{\mathrm{fluide}}\,\vv g\,V_{\mathrm{déplacé}}\)}


\end{document}
