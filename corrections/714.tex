\documentclass[a4paper]{article}
\usepackage[T1]{fontenc}
\usepackage[utf8]{inputenc}
\usepackage{lmodern}
\usepackage{amsmath,amssymb}
\usepackage[top=3cm,bottom=2cm,left=2cm,right=2cm]{geometry}
\usepackage{fancyhdr}
\usepackage{esvect,esint}
\usepackage{xcolor}
\usepackage{tikz}\usetikzlibrary{calc}

\parskip1em\parindent0pt\let\ds\displaystyle

\begin{document}

\pagestyle{fancy}
\fancyhf{}
\setlength{\headheight}{15pt}
\fancyhead[L]{Thermodynamique}\fancyhead[R]{Question 24}

% Énoncé
\begin{center}
	\large{\boldmath{\textbf{Lois de Laplace}}}
\end{center}

% Correction

Soit un gaz parfait d'indice adiabatique \(\gamma\) subissant une transformation adiabatique quasi-statique.\\
On note \(P\) sa pression, \(V\) son volume et \(T\) sa température.\\
Alors \begin{center}
\fcolorbox{red}{white}{
\begin{tabular}{c}
\(PV^{\gamma}=\mathrm{cste}\)\\
\(TV^{\gamma-1}=\mathrm{cste}\)\\
\(T^{\gamma}P^{1-\gamma}=\mathrm{cste}\)
\end{tabular}}
\end{center}

\end{document}
