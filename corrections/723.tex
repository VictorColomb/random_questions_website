\documentclass[a4paper]{article}
\usepackage[T1]{fontenc}
\usepackage[utf8]{inputenc}
\usepackage{lmodern}
\usepackage{amsmath,amssymb}
\usepackage[top=3cm,bottom=2cm,left=2cm,right=2cm]{geometry}
\usepackage{fancyhdr}
\usepackage{esvect,esint}
\usepackage{xcolor}
\usepackage{tikz}\usetikzlibrary{calc}

\parskip1em\parindent0pt\let\ds\displaystyle

\begin{document}

\pagestyle{fancy}
\fancyhf{}
\setlength{\headheight}{15pt}
\fancyhead[L]{Thermodynamique}\fancyhead[R]{Question 33}

% Énoncé
\begin{center}
	\large{\boldmath{\textbf{Lien entre une grandeur et son titre massique}}}
\end{center}

% Correction

Pour une grandeur massique \(y\) lors d'un changement d'état \(a\leftrightarrow b\), si on note \( x \) le titre massique de \(a\), on a :
\begin{center}\fcolorbox{red}{white}{\(y=xy_a+(1-x)y_b\)}\end{center} où \(y_a/y_b\) sont les valeurs de \(y\) pour l'espèce seulement à l'état \(a/b\).

\end{document}
