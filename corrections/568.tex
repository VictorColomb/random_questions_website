\documentclass[a4paper]{article}
\usepackage[T1]{fontenc}
\usepackage[utf8]{inputenc}
\usepackage{lmodern}
\usepackage{amsmath,amssymb}
\usepackage[top=3cm,bottom=2cm,left=2cm,right=2cm]{geometry}
\usepackage{fancyhdr}
\usepackage{esvect,esint}
\usepackage{xcolor}
\usepackage{tikz,circuitikz}\usetikzlibrary{calc}

\parskip1em\parindent0pt\let\ds\displaystyle

\begin{document}

\pagestyle{fancy}
\fancyhf{}
\setlength{\headheight}{15pt}
\fancyhead[L]{Electrocinétique}\fancyhead[R]{Question 20}

% Énoncé
\begin{center}
	\large{\boldmath{\textbf{Équations complexes $\underline{U}-\underline{I}$ d’une résistance, d’une source réelle de courant, \\ d’une source réelle de tension, d’un condensateur, d’une inductance}}}
\end{center}

% Correction

On note \(\underline{U}\) la tension complexe et \(\underline{I}\) l'intensité complexe dans le dipôle en convention génératrice.\\
\begin{table}[h]
\centering
\begin{tabular}{l|l}
Dipôle                   & Equation \\ \hline
Résistance \(R\)             & \(\underline{U}=R\underline{I}\)     \\
Source réelle de courant & \(\underline{I}=\underline{I_0}-\dfrac{\underline{U}}{R}\)   \\
Source réelle de tension & \(\underline{U}=\underline{U_0}-R\underline{I}\)   \\
Condensateur \(C\)           & \(\underline{I}=jC\omega\underline{U}\)      \\
Inductance \(L\)             & \(\underline{U}=jL\omega\underline{I}\)   
\end{tabular}
\end{table}


\end{document}
