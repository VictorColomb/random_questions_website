\documentclass[a4paper]{article}
\usepackage[T1]{fontenc}
\usepackage[utf8]{inputenc}
\usepackage{lmodern}
\usepackage{amsmath,amssymb}
\usepackage[top=3cm,bottom=2cm,left=2cm,right=2cm]{geometry}
\usepackage{fancyhdr}
\usepackage{esvect,esint}
\usepackage{xcolor}
\usepackage{tikz}\usetikzlibrary{calc}

\parskip1em\parindent0pt\let\ds\displaystyle

\begin{document}

\pagestyle{fancy}
\fancyhf{}
\setlength{\headheight}{15pt}
\fancyhead[L]{Chimie}\fancyhead[R]{Question 13}

% Énoncé
\begin{center}
	\large{\boldmath{\textbf{Définition de l’avancement d’une réaction}}}
\end{center}

% Correction

Pour une réaction chimique notée \(\sum_i\nu_iB_i\), avec \(\nu_i\) négatif si \(B_i\) est un réactif et positif sinon, on définit l'avancement \(\xi\) par la grandeur vérifiant :
\begin{center}
\fcolorbox{red}{white}{\(\forall i,\mathrm{d}\xi=\dfrac{\mathrm{d}n_i}{\nu_i}\)}
\end{center}
où \(n_i\) est la quantité de matière de l'espèce \(B_i\) au cours de la réaction.

\end{document}
