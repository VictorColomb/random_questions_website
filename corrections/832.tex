\documentclass[a4paper]{article}
\usepackage[T1]{fontenc}
\usepackage[utf8]{inputenc}
\usepackage{lmodern}
\usepackage{amsmath,amssymb}
\usepackage[top=3cm,bottom=2cm,left=2cm,right=2cm]{geometry}
\usepackage{fancyhdr}
\usepackage{esvect,esint}
\usepackage{xcolor}
\usepackage{tikz}\usetikzlibrary{calc}

\parskip1em\parindent0pt\let\ds\displaystyle

\begin{document}

\pagestyle{fancy}
\fancyhf{}
\setlength{\headheight}{15pt}
\fancyhead[L]{Mécanique}\fancyhead[R]{Question 8}

% Énoncé
\begin{center}
	\large{\boldmath{\textbf{Définition du torseur dynamique}}}
\end{center}

% Correction

On consière un système \( \Sigma \). \\
On en définit d'abord le moment dynamique :
\begin{center}
\fcolorbox{red}{white}{\( \vv\delta(A,S / \mathcal{R}) = \ds\int_\Sigma \vv{AP}\wedge \vv a(P\in\Sigma / \mathcal R) \, \mathrm dm(P) \)}
\end{center}

Pour un solide, on a plus simplement
\begin{center}
\fcolorbox{red}{white}{\( \delta(A,S / \mathcal R) = \left.\dfrac{\mathrm d \vv\sigma(A,S / \mathcal R)}{\mathrm dt}\right|_{\mathcal R} + m_S \vv V(A / \mathcal R) \wedge \vv V(G,S / \mathcal R) \)}
\end{center}

Enfin, on définit le torseur dynamique du système \( \Sigma \) au point $A$ :
\begin{center}
\fcolorbox{red}{white}{\( \{\mathcal D_{\Sigma / \mathcal R}\} = \left\{ \begin{matrix} m_\Sigma \vv a(G_\Sigma \in\Sigma / \mathcal R) \\ \vv\delta(A,\Sigma / \mathcal R) \end{matrix} \right\}_A \)}
\end{center}


\end{document}
