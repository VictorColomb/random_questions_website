\documentclass[a4paper]{article}
\usepackage[T1]{fontenc}
\usepackage[utf8]{inputenc}
\usepackage{lmodern}
\usepackage{amsmath,amssymb}
\usepackage[top=3cm,bottom=2cm,left=2cm,right=2cm]{geometry}
\usepackage{fancyhdr}
\usepackage{esvect,esint}
\usepackage{xcolor}
\usepackage{tikz,circuitikz}\usetikzlibrary{calc}

\parskip1em\parindent0pt\let\ds\displaystyle

\begin{document}

\pagestyle{fancy}
\fancyhf{}
\setlength{\headheight}{15pt}
\fancyhead[L]{Electrocinétique}\fancyhead[R]{Question 6}

% Énoncé
\begin{center}
	\large{\boldmath{\textbf{Graphe $U-I$ d’une source idéale de courant}}}
\end{center}

% Correction


\begin{center}
  \begin{tikzpicture}
    \draw[thick,->](-2.5,0)--(2.5,0) node[anchor=north west ]{$U$};
    \draw[thick,->](0,-2.5)--(0,2.5) node[anchor=south east]{$I$};
    \draw (-2.5,1)--(2.5,1) ;
    \draw (0,1) node[anchor=south east]{$I_0$};
  \end{tikzpicture}
\end{center}

\end{document}
