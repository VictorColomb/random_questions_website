\documentclass[a4paper]{article}
\usepackage[T1]{fontenc}
\usepackage[utf8]{inputenc}
\usepackage{lmodern}
\usepackage{amsmath,amssymb}
\usepackage[top=3cm,bottom=2cm,left=2cm,right=2cm]{geometry}
\usepackage{fancyhdr}
\usepackage{esvect}
\usepackage{xcolor}
\usepackage{tikz,circuitikz}\usetikzlibrary{calc}

\parskip 1em\parindent 0pt

\begin{document}

\pagestyle{fancy}
\fancyhf{}
\setlength{\headheight}{15pt}
\fancyhead[L]{Electrocinétique}\fancyhead[R]{Question 3}

% Énoncé
\begin{center}
	\large{\boldmath{\textbf{Puissance consommée par un dipôle}}}
\end{center}

% Correction

En convention réceptrice, on note \(U\) la tension dans le dipôle et \(I\) l'intensité qui le parcourt.\\
Alors la puissance consommée par le dipôle est \fcolorbox{red}{white}{\(P=UI\)}

\end{document}
