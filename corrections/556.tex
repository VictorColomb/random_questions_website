\documentclass[a4paper]{article}
\usepackage[T1]{fontenc}
\usepackage[utf8]{inputenc}
\usepackage{lmodern}
\usepackage{amsmath,amssymb}
\usepackage[top=3cm,bottom=2cm,left=2cm,right=2cm]{geometry}
\usepackage{fancyhdr}
\usepackage{esvect,esint}
\usepackage{xcolor}
\usepackage{tikz,circuitikz}\usetikzlibrary{calc}

\parskip1em\parindent0pt\let\ds\displaystyle

\begin{document}

\pagestyle{fancy}
\fancyhf{}
\setlength{\headheight}{15pt}
\fancyhead[L]{Electrocinétique}\fancyhead[R]{Question 8}

% Énoncé
\begin{center}
	\large{\boldmath{\textbf{Définition d’une source réelle de tension}}}
\end{center}

% Correction


Si pour tout $I$, $P_{\mathrm{délivrée}} = \alpha I-\beta I^2$ alors $\forall I\quad V_A-V_B=\alpha-\beta I$\\
  Alors on note $\alpha=E_g$ la force électromotrice et $\beta=R$ la résistance interne\\
  On a :\begin{center}\fcolorbox{red}{white}{\(V_A-V_B=E_g-RI\)}\end{center}
C'est une source réelle de tension.
  \begin{minipage}{0.2\linewidth}
    Représentation :
  \end{minipage}
  \begin{minipage}{0.5\linewidth}
    \begin{circuitikz}[scale=.7]
      \draw (1,0) node[right] {B} to[short,*-,i=$I$] (0,0) to[R=$R$] (0,2) to[vsource=$E_g$] (0,4) to[short,-*] (1,4) node[right] {A};
    \end{circuitikz}
  \end{minipage}
\end{document}
