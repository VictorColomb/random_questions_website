\documentclass[a4paper]{article}
\usepackage[T1]{fontenc}
\usepackage[utf8]{inputenc}
\usepackage{lmodern}
\usepackage{amsmath,amssymb}
\usepackage[top=3cm,bottom=2cm,left=2cm,right=2cm]{geometry}
\usepackage{fancyhdr}
\usepackage{esvect}
\usepackage{xcolor}
\usepackage{tikz}\usetikzlibrary{calc}

\parskip 1em\parindent 0pt

\begin{document}

\pagestyle{fancy}
\fancyhf{}
\setlength{\headheight}{15pt}
\fancyhead[L]{Mécanique}\fancyhead[R]{Question 26}

% Énoncé
\begin{center}
	\large{\boldmath{\textbf{Lois de Coulomb pour le contact entre deux solides (2)}}}
\end{center}

% Correction

Loi de Coulomb pour la statique:\\
Si deux solides en contact sont immobiles l'un par rapport à l'autre, alors les forces de frottements tangentielles \(\vv{T}\) et normales \(\vv{N}\) vérifient:
\begin{center}
\fcolorbox{red}{white}{\(\Vert\vv{T}\Vert\leq f_s\Vert\vv{N}\Vert\)}
\end{center}
où \(f_s\) est une constante appelée coefficient de frottement statique.

Loi de Coulomb pour le mouvement:\\
Si deux solides sont en mouvement l'un par rapport à l'autre, alors les forces de frottements tangentielles \(\vv{T}\) et normales \(\vv{N}\) vérifient:
\begin{center}
\fcolorbox{red}{white}{\(\vv{T}=-f_d\Vert\vv{N}\Vert\dfrac{\vv{v}}{v}\)}
\end{center}
où \(\vv{v}\) est la vitesse relative du solide sur lequel la force s'applique par rapport à l'autre et \(f_d\) est une constante appelée coefficient de frottement dynamique.


\end{document}
