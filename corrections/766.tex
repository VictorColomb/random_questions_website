\documentclass[a4paper]{article}
\usepackage[T1]{fontenc}
\usepackage[utf8]{inputenc}
\usepackage{lmodern}
\usepackage{amsmath,amssymb}
\usepackage[top=3cm,bottom=2cm,left=2cm,right=2cm]{geometry}
\usepackage{fancyhdr}
\usepackage{esvect}
\usepackage{xcolor}
\usepackage{tikz}\usetikzlibrary{calc}

\parskip 1em\parindent 0pt

\begin{document}

\pagestyle{fancy}
\fancyhf{}
\setlength{\headheight}{15pt}
\fancyhead[L]{Chimie}\fancyhead[R]{Question 22}

% Énoncé
\begin{center}
	\large{\boldmath{\textbf{Troisième principe de la thermodynamique}}}
\end{center}

% Correction

\fcolorbox{red}{white}{\(s_m^{\circ}(0)=0\) pour un corps pur}

Par définition, on pose \[s_m^{\circ}(T)=s_m^{\circ}(0)+\Delta S_{\mathrm{chauffage}}+\Delta S_{\text{changement d'état}}\] pour un corps pur \\
et \(\Delta_rS^{\circ}\simeq\Delta_r\nu_gS_m^{\circ}\) pour une réaction.

On peut alors prévoir le signe de \(\Delta_rS^{\circ}\) en évaluant \(\Delta_r\nu_g\) car \(S^{\circ}>0\).



\end{document}
