\documentclass[a4paper]{article}
\usepackage[T1]{fontenc}
\usepackage[utf8]{inputenc}
\usepackage{lmodern}
\usepackage{amsmath,amssymb}
\usepackage[top=3cm,bottom=2cm,left=2cm,right=2cm]{geometry}
\usepackage{fancyhdr}
\usepackage{esvect,esint}
\usepackage{xcolor}
\usepackage{tikz}\usetikzlibrary{calc}

\parskip1em\parindent0pt\let\ds\displaystyle

\begin{document}

\pagestyle{fancy}
\fancyhf{}
\setlength{\headheight}{15pt}
\fancyhead[L]{SLCI}\fancyhead[R]{Question 1}

% Énoncé
\begin{center}
	\large{\boldmath{\textbf{Formule du taux de dépassement}}}
\end{center}

% Correction

On note \( t_k \) le temps du \( k^{\mathrm{e}} \) dépassement. \\
Alors,
\begin{center}
\fcolorbox{red}{white}{ \( D_k^{\%} = \left| \dfrac{s(t_k) - s_\infty}{s_\infty} \right| \) }
\end{center}

\end{document}
