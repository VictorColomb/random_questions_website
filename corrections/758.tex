\documentclass[a4paper]{article}
\usepackage[T1]{fontenc}
\usepackage[utf8]{inputenc}
\usepackage{lmodern}
\usepackage{amsmath,amssymb}
\usepackage[top=3cm,bottom=2cm,left=2cm,right=2cm]{geometry}
\usepackage{fancyhdr}
\usepackage{esvect,esint}
\usepackage{xcolor}
\usepackage{tikz}\usetikzlibrary{calc}

\parskip1em\parindent0pt\let\ds\displaystyle

\begin{document}

\pagestyle{fancy}
\fancyhf{}
\setlength{\headheight}{15pt}
\fancyhead[L]{Chimie}\fancyhead[R]{Question 14}

% Énoncé
\begin{center}
	\large{\boldmath{\textbf{Définition grandeur de réaction}}}
\end{center}

% Correction

Soit \(Y\) une grandeur chimique associée à un système et soit \(\sum\limits_i\nu_iB_i\) une réaction.\\
On définit la grandeur de réaction \(\Delta_rY\) associée à la réaction par :\begin{center}\fcolorbox{red}{white}{\(\Delta_rY=\sum\limits_i\nu_iy_i=\sum\limits_i\nu_i\left.\dfrac{\partial Y}{\partial n_i}\right|_{T,P,n_j\neq n_i}=\left.\dfrac{\partial Y}{\partial \xi}\right|_{T,P}\)}\end{center}

\end{document}
