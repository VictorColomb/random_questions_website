\documentclass[a4paper]{article}
\usepackage[T1]{fontenc}
\usepackage[utf8]{inputenc}
\usepackage{lmodern}
\usepackage{amsmath,amssymb}
\usepackage[top=3cm,bottom=2cm,left=2cm,right=2cm]{geometry}
\usepackage{fancyhdr}
\usepackage{esvect,esint}
\usepackage{xcolor}
\usepackage{tikz}\usetikzlibrary{calc}

\parskip1em\parindent0pt\let\ds\displaystyle

\begin{document}

\pagestyle{fancy}
\fancyhf{}
\setlength{\headheight}{15pt}
\fancyhead[L]{Chimie}\fancyhead[R]{Question 28}

% Énoncé
\begin{center}
	\large{\boldmath{\textbf{Sens d’une réaction en fonction de $Q_r$ et de $K^{\circ}$}}}
\end{center}

% Correction

On rappelle que \(\Delta_rG=\mathrm{RT ln}\left(\dfrac{Q_r}{K^{\circ}}\right)\) et que \(\Delta_rG\mathrm{d}\xi=-T\delta S_C\leqslant 0\).\\
Si \(Q_r< K^{\circ}\), alors \(\Delta_rG<0\) donc \(\mathrm{d}\xi>0\) donc la réaction avance dans le sens des produits.\\
Si \(Q_r> K^{\circ}\), alors \(\Delta_rG>0\) donc \(\mathrm{d}\xi<0\) donc la réaction avance dans le sens des réactifs.\\
Si \(Q_r=K^{\circ}\), alors \(\Delta_rG=0\) donc \(\delta S_c=0\) et comme une réaction chimique crée toujours de l'entropie, la réaction chimique n'a pas lieu et le système est à l'équilibre chimique.


\end{document}
