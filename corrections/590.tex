\documentclass[a4paper]{article}
\usepackage[T1]{fontenc}
\usepackage[utf8]{inputenc}
\usepackage{lmodern}
\usepackage{amsmath,amssymb}
\usepackage[top=3cm,bottom=2cm,left=2cm,right=2cm]{geometry}
\usepackage{fancyhdr}
\usepackage{esvect}
\usepackage{xcolor}
\usepackage{tikz}\usetikzlibrary{calc}

\parskip 1em\parindent 0pt

\begin{document}

\pagestyle{fancy}
\fancyhf{}
\setlength{\headheight}{15pt}
\fancyhead[L]{Mécanique}\fancyhead[R]{Question 3}

% Énoncé
\begin{center}
	\large{\boldmath{\textbf{Théorèmes du rotationnel et de la divergence}}}
\end{center}

% Correction

Théorème du rotationnel :
\begin{center}
	\fcolorbox{red}{white}{\(\vv{\mathrm{rot}}\,\vv{A}=\vv{0}\) si et seulement si il existe \(\vv{B}\) tel que \(\vv{A}=\vv{\mathrm{grad}}\,\vv{B}\)}
\end{center}
Dans tous les cas, \(\vv{\mathrm{rot}}\,\vv{\mathrm{grad}}=\vv{0}\).

Théorème de la divergence :
\begin{center}
	\fcolorbox{red}{white}{\(\mathrm{div}\,\vv{A}=0\) si et seulement si il existe \(\vv{B}\) tel que \(\vv{A}=\vv{\mathrm{rot}}\,\vv{B}\)}
\end{center}
Dans tous les cas, \(\mathrm{div}\,\vv{\mathrm{rot}}=0\).

\end{document}
