\documentclass[a4paper]{article}
\usepackage[T1]{fontenc}
\usepackage[utf8]{inputenc}
\usepackage{lmodern}
\usepackage{amsmath,amssymb}
\usepackage[top=3cm,bottom=2cm,left=2cm,right=2cm]{geometry}
\usepackage{fancyhdr}
\usepackage{esvect,esint}
\usepackage{xcolor}
\usepackage{tikz}\usetikzlibrary{calc}

\parskip1em\parindent0pt\let\ds\displaystyle

\begin{document}

\pagestyle{fancy}
\fancyhf{}
\setlength{\headheight}{15pt}
\fancyhead[L]{Thermodynamique}\fancyhead[R]{Question 46}

% Énoncé
\begin{center}
	\large{\boldmath{\textbf{Équilibre d’une atmosphère isotherme}}}
\end{center}

% Correction

L'atmosphère est à l'équilibre donc \(\vv{\mathrm{grad}}p=-\rho g\vv{z}\).\\
Or le gaz de l'atmosphère est considéré parfait donc \(\rho=\dfrac{m}{V}=\dfrac{Mn}{V}=\dfrac{Mp}{RT}\).\\
Donc \fcolorbox{red}{white}{\(p=p_0\exp{-\dfrac{Mg}{RT}z}\)}




\end{document}
