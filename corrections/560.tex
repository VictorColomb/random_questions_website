\documentclass[a4paper]{article}
\usepackage[T1]{fontenc}
\usepackage[utf8]{inputenc}
\usepackage{lmodern}
\usepackage{amsmath,amssymb}
\usepackage[top=3cm,bottom=2cm,left=2cm,right=2cm]{geometry}
\usepackage{fancyhdr}
\usepackage{esvect,esint}
\usepackage{xcolor}
\usepackage{tikz,circuitikz}\usetikzlibrary{calc}

\parskip1em\parindent0pt\let\ds\displaystyle

\begin{document}

\pagestyle{fancy}
\fancyhf{}
\setlength{\headheight}{15pt}
\fancyhead[L]{Electrocinétique}\fancyhead[R]{Question 12}

% Énoncé
\begin{center}
	\large{\boldmath{\textbf{Théorème d’équivalence entre une source réelle \\ de courant et une source réelle de tension}}}
\end{center}

% Correction

Il y a équivalence entre :
\begin{center}
\begin{minipage}{0.2\linewidth}
\centering
  \begin{circuitikz}
    \draw (1,0) node[right] {B} to[short,*-] (0,0) to[vsource=$E_g$] ++(0,1.5) to[R=$R$] ++(0,1.5) to[short,i=$I$,-*] ++(1,0) node[right] {A};
  \end{circuitikz}
\end{minipage}
et
\begin{minipage}{0.3\linewidth}
\centering
  \begin{circuitikz}
    \draw (3,0) node[right] {B} to[short,*-] (0,0) to[isource=$\dfrac{E_g}{R}$] ++(0,2) -- ++(2,0) to[short,i=$I$,-*] ++(1,0) node[right] {A};
    \draw (2,0) to[R=$R$] ++(0,2);
  \end{circuitikz}
\end{minipage}
\end{center}


\end{document}
