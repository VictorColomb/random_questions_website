\documentclass[a4paper]{article}
\usepackage[T1]{fontenc}
\usepackage[utf8]{inputenc}
\usepackage{lmodern}
\usepackage{amsmath,amssymb}
\usepackage[top=3cm,bottom=2cm,left=2cm,right=2cm]{geometry}
\usepackage{fancyhdr}
\usepackage{esvect}
\usepackage{xcolor}
\usepackage{tikz}\usetikzlibrary{calc}

\parskip 1em\parindent 0pt

\begin{document}

\pagestyle{fancy}
\fancyhf{}
\setlength{\headheight}{15pt}
\fancyhead[L]{Optique}\fancyhead[R]{Question 1}

% Énoncé
\begin{center}
	\large{\boldmath{\textbf{Définition de la vergence}}}
\end{center}

% Correction

\fcolorbox{red}{white}{\( V = \dfrac{n}{f'} \)}\par
Dans l'air, \( V = \dfrac{1}{f'} \).\\
\( [V] = L^{ - 1} \)\\
La vergence s'exprime en dioptrie (\( \delta \) ).\\
La vergence est positive pour un système convergent et négative pour un système divergent.


\end{document}
