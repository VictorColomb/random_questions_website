\documentclass[a4paper]{article}
\usepackage[T1]{fontenc}
\usepackage[utf8]{inputenc}
\usepackage{lmodern}
\usepackage{amsmath,amssymb}
\usepackage[top=3cm,bottom=2cm,left=2cm,right=2cm]{geometry}
\usepackage{fancyhdr}
\usepackage{esvect}
\usepackage{xcolor}
\usepackage{tikz}\usetikzlibrary{calc}

\parskip 1em\parindent 0pt

\begin{document}

\pagestyle{fancy}
\fancyhf{}
\setlength{\headheight}{15pt}
\fancyhead[L]{Mécanique}\fancyhead[R]{Question 12}

% Énoncé
\begin{center}
	\large{\boldmath{\textbf{Lois de Newton (3)}}}
\end{center}

% Correction

Première loi :\\
\emph{"Tout corps persévère dans l'état de repos ou de mouvement uniforme en ligne droite dans lequel il se trouve, à moins que quelque force n'agisse sur lui, et ne le contraigne à changer d'état."}

Deuxième loi :\\
\emph{"Les changements qui arrivent dans le mouvement sont proportionnels à la force motrice ; et se font dans la ligne droite dans laquelle cette force a été imprimée."}\\
\fcolorbox{red}{white}{\(\dfrac{d \vv p}{dt}=\sum\vv{F_{\mathrm{ext}}}\)}

Troisième loi :\\
\emph{"L'action est toujours égale à la réaction ; c'est-à-dire que les actions de deux corps l'un sur l'autre sont toujours égales et de sens contraires."}

\end{document}
