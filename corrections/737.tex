\documentclass[a4paper]{article}
\usepackage[T1]{fontenc}
\usepackage[utf8]{inputenc}
\usepackage{lmodern}
\usepackage{amsmath,amssymb}
\usepackage[top=3cm,bottom=2cm,left=2cm,right=2cm]{geometry}
\usepackage{fancyhdr}
\usepackage{esvect}
\usepackage{xcolor}
\usepackage{tikz}\usetikzlibrary{calc}

\parskip 1em\parindent 0pt

\begin{document}

\pagestyle{fancy}
\fancyhf{}
\setlength{\headheight}{15pt}
\fancyhead[L]{Thermodynamique}\fancyhead[R]{Question 47}

% Énoncé
\begin{center}
	\large{\boldmath{\textbf{Hypothèses de la théorie cinétique des gaz (3)}}}
\end{center}

% Correction

\begin{enumerate}
	\item Les molécules gazeuses sont assimilées à des points matériels sans volume.
	\item Il n'y a pas d'interaction intermoléculaire.
	\item Dans l'espace des vitesses, toutes les directions sont équiprobables.
\end{enumerate}

\end{document}
