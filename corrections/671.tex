\documentclass[a4paper]{article}
\usepackage[T1]{fontenc}
\usepackage[utf8]{inputenc}
\usepackage{lmodern}
\usepackage{amsmath,amssymb}
\usepackage[top=3cm,bottom=2cm,left=2cm,right=2cm]{geometry}
\usepackage{fancyhdr}
\usepackage{esvect}
\usepackage{xcolor}
\usepackage{tikz}\usetikzlibrary{calc}

\parskip 1em\parindent 0pt

\begin{document}

\pagestyle{fancy}
\fancyhf{}
\setlength{\headheight}{15pt}
\fancyhead[L]{Electromagnétisme}\fancyhead[R]{Question 56}

% Énoncé
\begin{center}
	\large{\boldmath{\textbf{Approximations du rayonnement (3)}}}
\end{center}

% Correction

On note \(r\) la distance d'observation, \(d\) le diamètre du dipôle et \(\lambda\) sa longueur d'onde.

Approximation de la phase :
\begin{center}
\fcolorbox{red}{white}{\(d\ll \lambda\)}
\end{center}

Approximation de la zone de rayonnement (ou du champ lointain) :
\begin{center}
\fcolorbox{red}{white}{\(\lambda\ll r\)}
\end{center}

Approximation dipolaire :
\begin{center}
\fcolorbox{red}{white}{\(d\ll r\)}
\end{center}


\end{document}
