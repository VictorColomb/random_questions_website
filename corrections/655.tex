\documentclass[a4paper]{article}
\usepackage[T1]{fontenc}
\usepackage[utf8]{inputenc}
\usepackage{lmodern}
\usepackage{amsmath,amssymb}
\usepackage[top=3cm,bottom=2cm,left=2cm,right=2cm]{geometry}
\usepackage{fancyhdr}
\usepackage{esvect,esint}
\usepackage{xcolor}
\usepackage{tikz}\usetikzlibrary{calc}

\parskip1em\parindent0pt\let\ds\displaystyle

\begin{document}

\pagestyle{fancy}
\fancyhf{}
\setlength{\headheight}{15pt}
\fancyhead[L]{Electromagnétisme}\fancyhead[R]{Question 40}

% Énoncé
\begin{center}
	\large{\boldmath{\textbf{Définition du vecteur de Poynting complexe}}}
\end{center}

% Correction

Si \(\vv{\underline{E}}\) et \(\vv{\underline{B}}\) sont les champs électriques et magnétiques complexes, alors le vecteur de Poynting complexe est défini par :
\begin{center}\fcolorbox{red}{white}{\(\vv{\underline{R}}=\dfrac{\vv{\underline{E}}\wedge\vv{\underline{B}}^*}{\mu_0}\)}\end{center}

\end{document}
