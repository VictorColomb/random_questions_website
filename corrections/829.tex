\documentclass[a4paper]{article}
\usepackage[T1]{fontenc}
\usepackage[utf8]{inputenc}
\usepackage{lmodern}
\usepackage{amsmath,amssymb}
\usepackage[top=3cm,bottom=2cm,left=2cm,right=2cm]{geometry}
\usepackage{fancyhdr}
\usepackage{esvect,esint}
\usepackage{xcolor}
\usepackage{tikz}\usetikzlibrary{calc}

\parskip1em\parindent0pt\let\ds\displaystyle

\begin{document}

\pagestyle{fancy}
\fancyhf{}
\setlength{\headheight}{15pt}
\fancyhead[L]{Mécanique}\fancyhead[R]{Question 5}

% Énoncé
\begin{center}
	\large{\boldmath{\textbf{Définition de l'opérateur d'inertie}}}
\end{center}

% Correction

On prend un point fixe $O$ et une base \( (x,y,z) \). On utilise ce système de coordonnées cartésiennes.

Soit $S$ un système. \\
On définit alors l'opérateur d'inertie par :
\begin{center}
\fcolorbox{red}{white}{
	\( I(O,S) = \begin{pmatrix} \ds\int_S (y^2 + z^2) \,\mathrm{d}m & -\ds\int_S xy \mathrm{d}m & -\ds\int_S xz \mathrm{d}m \\[8pt] -\ds\int_S xy \mathrm{d}m & \ds\int_S (x^2 + z^2) \,\mathrm{d}m & -\ds\int_S yz \mathrm{d}m \\[8pt] -\ds\int_S xz \mathrm{d}m & -\ds\int_S yz \mathrm{d}m & \ds\int_S (x^2 + y^2) \,\mathrm{d}m \end{pmatrix} \)
}\end{center}

\end{document}
