\documentclass[a4paper]{article}
\usepackage[T1]{fontenc}
\usepackage[utf8]{inputenc}
\usepackage{lmodern}
\usepackage{amsmath,amssymb}
\usepackage[top=3cm,bottom=2cm,left=2cm,right=2cm]{geometry}
\usepackage{fancyhdr}
\usepackage{esvect,esint}
\usepackage{xcolor}
\usepackage{tikz}\usetikzlibrary{calc}

\parskip1em\parindent0pt\let\ds\displaystyle

\begin{document}

\pagestyle{fancy}
\fancyhf{}
\setlength{\headheight}{15pt}
\fancyhead[L]{Chimie}\fancyhead[R]{Question 8}

% Énoncé
\begin{center}
	\large{\boldmath{\textbf{Définition grandeur standard}}}
\end{center}

% Correction

Soit \(Y\) une grandeur chimique associé à un système \((S)\).\\
La grandeur standard \(Y^{\circ}\) associée à \(Y\) est définie par :\begin{center}
\fcolorbox{red}{white}{\(Y^{\circ}(T)=Y(T,p_0=1 \mathrm{bar})\)}\end{center}


\end{document}
