\documentclass[a4paper]{article}
\usepackage[T1]{fontenc}
\usepackage[utf8]{inputenc}
\usepackage{lmodern}
\usepackage{amsmath,amssymb}
\usepackage[top=3cm,bottom=2cm,left=2cm,right=2cm]{geometry}
\usepackage{fancyhdr}
\usepackage{esvect,esint}
\usepackage{xcolor}\usepackage[version=4]{mhchem}
\usepackage{tikz}\usetikzlibrary{calc}

\parskip1em\parindent0pt\let\ds\displaystyle

\begin{document}

\pagestyle{fancy}
\fancyhf{}
\setlength{\headheight}{15pt}
\fancyhead[L]{Chimie}\fancyhead[R]{Question 33}

% Énoncé
\begin{center}
	\large{\boldmath{\textbf{Formule de Nernst}}}
\end{center}

% Correction

On s'intéresse à la réaction \(\alpha Ox +\ce{xH+}+ne^-\ce{<=>}\beta Red\).\\
On note \(E^{\circ}=E_{Ox/Red}^{\circ}\).\\
Alors les activités à l'équilibre vérifient :
\begin{center}
\fcolorbox{red}{white}{\(E=E^{\circ}+\dfrac{U}{n}\log\left(\dfrac{h^x(Ox)^{\alpha}}{(Red)^{\beta}}\right)\)}
\end{center}


\end{document}
