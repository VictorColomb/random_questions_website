\documentclass[a4paper]{article}
\usepackage[T1]{fontenc}
\usepackage[utf8]{inputenc}
\usepackage{lmodern}
\usepackage{amsmath,amssymb}
\usepackage[top=3cm,bottom=2cm,left=2cm,right=2cm]{geometry}
\usepackage{fancyhdr}
\usepackage{esvect,esint}
\usepackage{xcolor}
\usepackage{tikz}\usetikzlibrary{calc}

\parskip1em\parindent0pt\let\ds\displaystyle

\begin{document}

\pagestyle{fancy}
\fancyhf{}
\setlength{\headheight}{15pt}
\fancyhead[L]{Chimie}\fancyhead[R]{Question 35}

% Énoncé
\begin{center}
	\large{\boldmath{\textbf{Valeur de $K^\circ$ en fonction des potentiels standards \\ et du nombre d’électrons échangés}}}
\end{center}

% Correction

Pour la réaction \(n_2Ox_1+n_1Red_2=n_1Ox_2+n_2Red_1\), on a :\begin{center}\fcolorbox{red}{white}{\(\mathrm{log}K^{\circ}=\dfrac{n_1n_2(E_1^{\circ}-E_2^{\circ})}{U}\)}\end{center}
où \(U=0,06V\).


\end{document}
