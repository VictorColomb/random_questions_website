\documentclass[a4paper]{article}
\usepackage[T1]{fontenc}
\usepackage[utf8]{inputenc}
\usepackage{lmodern}
\usepackage{amsmath,amssymb}
\usepackage[top=3cm,bottom=2cm,left=2cm,right=2cm]{geometry}
\usepackage{fancyhdr}
\usepackage{esvect,esint}
\usepackage{xcolor}
\usepackage{tikz}\usetikzlibrary{calc}

\parskip 1em\parindent 0pt

\begin{document}

\pagestyle{fancy}
\fancyhf{}
\setlength{\headheight}{15pt}
\fancyhead[L]{Thermodynamique}\fancyhead[R]{Question 40}

% Énoncé
\begin{center}
	\large{\boldmath{\textbf{Equation de la chaleur, démonstration monodimensionnelle}}}
\end{center}

% Correction

\begin{center}
	\fcolorbox{red}{white}{ \( \rho c \dfrac{\partial T }{\partial t} = \lambda \Delta T \) }
\end{center}

On fait un bilan enthalpique sur une tranche entre $x$ et $x$ + d$x$.
\\Donc \(\mathrm{d}(\delta H(t + \mathrm{d}t)) - \mathrm{d} (\delta H (t)) = (\Phi(x) -\Phi(x +\mathrm{d}x)) \mathrm{d}t\) \\
Donc \(
\dfrac{\partial (\delta H) }{\partial t} \mathrm{d}t = - \dfrac{\partial \Phi}{\partial x} \mathrm{d}x \mathrm{d}t \)

Par loi phénoménologique de Fourier : \(c\, \rho\, \mathrm{d}x\, S \dfrac{\partial T}{\partial t} = - \dfrac{\partial }{\partial x} \left(\displaystyle\iint - \lambda \dfrac{\partial T }{\partial x} \mathrm{d}S \right)\mathrm{d}x\)\\
Donc \(c\, \rho\, S\, \dfrac{\partial T}{\partial t} = \lambda \dfrac{\partial^2 T }{\partial x^2} S\)

Donc:
\begin{center}
    \fcolorbox{red}{white}{
        \(c\, \rho\, \dfrac{\partial T}{\partial t} = \lambda \dfrac{\partial^2 T }{\partial x^2} \)}
\end{center}

\end{document}
