\documentclass[a4paper]{article}
\usepackage[T1]{fontenc}
\usepackage[utf8]{inputenc}
\usepackage{lmodern}
\usepackage{amsmath,amssymb}
\usepackage[top=3cm,bottom=2cm,left=2cm,right=2cm]{geometry}
\usepackage{fancyhdr}
\usepackage{esvect,esint}
\usepackage{xcolor}
\usepackage{tikz}\usetikzlibrary{calc}

\parskip1em\parindent0pt\let\ds\displaystyle

\begin{document}

\pagestyle{fancy}
\fancyhf{}
\setlength{\headheight}{15pt}
\fancyhead[L]{Thermodynamique}\fancyhead[R]{Question 6}

% Énoncé
\begin{center}
	\large{\boldmath{\textbf{Définition transformation isochore, \\ monobare, isobare, isotherme}}}
\end{center}

% Correction

Une transformation isochore est une transformation qui se fait à volume constant.\\
Une transformation monobare est une transformation avec mêmes pressions initiale et finale.\\
Une transformation isobare est une transformation qui se fait à pression constante.\\
Une transformation isotherme est une transformation qui se fait à température constante.

\end{document}
