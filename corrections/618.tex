\documentclass[a4paper]{article}
\usepackage[T1]{fontenc}
\usepackage[utf8]{inputenc}
\usepackage{lmodern}
\usepackage{amsmath,amssymb,esint}
\usepackage[top=3cm,bottom=2cm,left=2cm,right=2cm]{geometry}
\usepackage{fancyhdr}
\usepackage{esvect}
\usepackage{xcolor}
\usepackage{tikz}\usetikzlibrary{calc}

\parskip 1em\parindent 0pt

\begin{document}

\pagestyle{fancy}
\fancyhf{}
\setlength{\headheight}{15pt}
\fancyhead[L]{Electromagnétisme}\fancyhead[R]{Question 3}

% Énoncé
\begin{center}
	\large{\boldmath{\textbf{Théorème de Gauss}}}
\end{center}

% Correction

Soient un volume $V$ et une surface $S$ qui l'entoure orientée vers l'extérieur\\
Soit $Q_{\mathrm{int}\,V}$ la quantité de charge à l'intérieur du volume $V$
Alors,
\begin{center}
\fcolorbox{red}{white}{ $\displaystyle\oiint_S \vv{E}\cdot \vec{\mathrm{d}S}=\dfrac{Q_{\mathrm{int}\,V}}{\varepsilon_0}$ }
\end{center}

\end{document}
