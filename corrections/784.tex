\documentclass[a4paper]{article}
\usepackage[T1]{fontenc}
\usepackage[utf8]{inputenc}
\usepackage{lmodern}
\usepackage{amsmath,amssymb}
\usepackage[top=3cm,bottom=2cm,left=2cm,right=2cm]{geometry}
\usepackage{fancyhdr}
\usepackage{esvect}
\usepackage{xcolor}
\usepackage{tikz}\usetikzlibrary{calc}

\parskip 1em\parindent 0pt

\begin{document}

\pagestyle{fancy}
\fancyhf{}
\setlength{\headheight}{15pt}
\fancyhead[L]{Chimie}\fancyhead[R]{Question 40}

% Énoncé
\begin{center}
	\large{\boldmath{\textbf{Définition surtension cathodique/anodique}}}
\end{center}

% Correction

La surtension cathodique ou anodique est la différence entre le potentiel de l'électrode à courant non nul et le potentiel d'électrode en l'absence de courant.\\
La surtension anodique \(\eta_a\) est positive alors que la surtension cathodique \(\eta_c\) est négative.

\end{document}
