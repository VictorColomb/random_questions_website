\documentclass[a4paper]{article}
\usepackage[T1]{fontenc}
\usepackage[utf8]{inputenc}
\usepackage{lmodern}
\usepackage{amsmath,amssymb}
\usepackage[top=3cm,bottom=2cm,left=2cm,right=2cm]{geometry}
\usepackage{fancyhdr}
\usepackage{esvect}
\usepackage{xcolor}
\usepackage{tikz}\usetikzlibrary{calc}

\parskip 1em\parindent 0pt

\begin{document}

\pagestyle{fancy}
\fancyhf{}
\setlength{\headheight}{15pt}
\fancyhead[L]{Mécanique quantique}\fancyhead[R]{Question 7}

% Énoncé
\begin{center}
	\large{\boldmath{\textbf{Condition de traitement quantique d’une particule}}}
\end{center}

% Correction

On définit l'action d'une particule par \( A = [\) moment cinétique \(] = ML^2T^{ - 1} \).\par
Si \fcolorbox{red}{white}{$A\sim\hbar$} alors la particule doit être appréhendée de façon quantique.\\[2pt]
Si \fcolorbox{red}{white}{\( A \gg \hbar \)} alors on traite la particule de façon classique.

\end{document}
