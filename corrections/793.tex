\documentclass[a4paper]{article}
\usepackage[T1]{fontenc}
\usepackage[utf8]{inputenc}
\usepackage{lmodern}
\usepackage{amsmath,amssymb}
\usepackage[top=3cm,bottom=2cm,left=2cm,right=2cm]{geometry}
\usepackage{fancyhdr}
\usepackage{esvect,esint}
\usepackage{xcolor}
\usepackage{tikz}\usetikzlibrary{calc}

\parskip1em\parindent0pt\let\ds\displaystyle

\begin{document}

\pagestyle{fancy}
\fancyhf{}
\setlength{\headheight}{15pt}
\fancyhead[L]{Chimie}\fancyhead[R]{Question 49}

% Énoncé
\begin{center}
	\large{\boldmath{\textbf{Valeur de la force électromotrice d’une pile en fonction \\ des potentiels électriques et des surtensions}}}
\end{center}

% Correction

On note \(E_c,E_a,\eta_c(I),\eta_a(I)\) les potentiels et les surtensions à la cathode et à l'anode.\\
On note \(r\) la chute ohmique aux bornes de la pile.\\
Alors la force électro-motrice fournie par la pile est donnée par :\begin{center}\fcolorbox{red}{white}{\(U=E_c-E_a-(\eta_c-\eta_a)(I)-rI\)}\end{center}

\end{document}
