\documentclass[a4paper]{article}
\usepackage[T1]{fontenc}
\usepackage[utf8]{inputenc}
\usepackage{lmodern}
\usepackage{amsmath,amssymb}
\usepackage[top=3cm,bottom=2cm,left=2cm,right=2cm]{geometry}
\usepackage{fancyhdr}
\usepackage{esvect,esint}
\usepackage{xcolor}
\usepackage{tikz,circuitikz}\usetikzlibrary{calc}

\parskip1em\parindent0pt\let\ds\displaystyle

\begin{document}

\pagestyle{fancy}
\fancyhf{}
\setlength{\headheight}{15pt}
\fancyhead[L]{Electrocinétique}\fancyhead[R]{Question 34}

% Énoncé
\begin{center}
	\large{\boldmath{\textbf{Filtrage d’une série de Fourier}}}
\end{center}

% Correction

Soit \(s\) un signal périodique dont on note \(e(t)=e_0+\sum e_n\cos(2\pi nft+\varphi_n)\) la décomposition en série de Fourier.\\
Soit \(\underline{H}\) la fonction de transfert d'un quadripôle linéaire stable, c'est-à-dire dont tous les pôles sont à partie réelle strictement négative.\\
Alors le régime libre disparaît rapidement et le régime forcé \(s_f\) vaut :\begin{center}
\fcolorbox{red}{white}{\(s_f(t)=e_0|\underline{H}(0)|+\ds\sum_ne_n|\underline{H}(nf)|\cos(2\pi nft +\varphi_n+\mathrm{arg}\underline{H}(nf))\)}
\end{center}
On remarque que le signal filtré par un quadripôle linéaire stable possède la même richesse spectrale que le signal entrant.


\end{document}
