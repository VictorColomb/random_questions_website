\documentclass[a4paper]{article}
\usepackage[T1]{fontenc}
\usepackage[utf8]{inputenc}
\usepackage{lmodern}
\usepackage{amsmath,amssymb}
\usepackage[top=3cm,bottom=2cm,left=2cm,right=2cm]{geometry}
\usepackage{fancyhdr}
\usepackage{esvect}
\usepackage{xcolor}
\usepackage{tikz}\usetikzlibrary{calc}

\parskip 1em\parindent 0pt

\begin{document}

\pagestyle{fancy}
\fancyhf{}
\setlength{\headheight}{15pt}
\fancyhead[L]{Mécanique}\fancyhead[R]{Question 10}

% Énoncé
\begin{center}
	\large{\boldmath{\textbf{Composition des vitesses et des accélérations}}}
\end{center}

% Correction

On note \(O_1\) le centre du référentiel galiléen \(\mathcal{R}_g\) et \(O_2\) le centre du référentiel \(\mathcal{R}\) en mouvement dans ce galiléen.\\
On note \(M\) le centre de masse de l'objet supposé ponctuel.\\
On note \(\vv{\Omega}\) le vecteur rotation de \(\mathcal{R}\) dans \(\mathcal{R}_g\). \\
On a
\begin{center}
	\fcolorbox{red}{white}{\(\vv{v}(M\in\mathcal{R}_g)=\vv{v}(M\in\mathcal{R})+\vv v(O_2 / \mathcal R_g)+\vv{\Omega}\wedge \vv{O_2M}\)}
\end{center}
et
\begin{center}
	\fcolorbox{red}{white}{\(\vv{a}(M\in\mathcal{R}_g)=\vv{a}(M\in\mathcal{R})+\vv{a}_{\mathrm{ent}}+\vv{a}_{\mathrm{cor}}\)}
\end{center}
avec \(\vv{a}_{\mathrm{ent}}=\vv{a}(O_2\in\mathcal{R}_g)+\vv{\Omega}\wedge(\vv{\Omega}\wedge\vv{O_2M})+
\left.\dfrac{\mathrm{d}\vv{\Omega}}{\mathrm{d}t}\right|_{\mathcal{R}_g}\wedge\vv{O_2M}\) \\
et \(\vv{a}_{\mathrm{cor}}=2\vv{\Omega}\wedge\vv{v}(M\in\mathcal{R})\)


\end{document}
