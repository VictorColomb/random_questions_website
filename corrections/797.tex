\documentclass[a4paper]{article}
\usepackage[T1]{fontenc}
\usepackage[utf8]{inputenc}
\usepackage{lmodern}
\usepackage{amsmath,amssymb}
\usepackage[top=3cm,bottom=2cm,left=2cm,right=2cm]{geometry}
\usepackage{fancyhdr}
\usepackage{esvect}
\usepackage{xcolor}
\usepackage{tikz}\usetikzlibrary{calc}

\parskip 1em\parindent 0pt

\begin{document}

\pagestyle{fancy}
\fancyhf{}
\setlength{\headheight}{15pt}
\fancyhead[L]{Mécanique quantique}\fancyhead[R]{Question 3}

% Énoncé
\begin{center}
	\large{\boldmath{\textbf{Relations de Louis de Broglie pour les particules}}}
\end{center}

% Correction

\fcolorbox{red}{white}{ \( \begin{array}{l} \vv p = \hbar \vv k \\[2pt] E = h\nu = \hbar\omega \end{array}\) }

\end{document}
