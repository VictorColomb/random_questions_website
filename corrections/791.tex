\documentclass[a4paper]{article}
\usepackage[T1]{fontenc}
\usepackage[utf8]{inputenc}
\usepackage{lmodern}
\usepackage{amsmath,amssymb}
\usepackage[top=3cm,bottom=2cm,left=2cm,right=2cm]{geometry}
\usepackage{fancyhdr}
\usepackage{esvect,esint}
\usepackage{xcolor}
\usepackage{tikz}\usetikzlibrary{calc}

\parskip1em\parindent0pt\let\ds\displaystyle

\begin{document}

\pagestyle{fancy}
\fancyhf{}
\setlength{\headheight}{15pt}
\fancyhead[L]{Chimie}\fancyhead[R]{Question 47}

% Énoncé
\begin{center}
	\large{\boldmath{\textbf{Définition de la corrosion et \\ domaines d’immunité, de corrosion et de passivité}}}
\end{center}

% Correction

La corrosion d’un métal est l’oxydation du métal à l’état d’ion métallique.

Le domaine d'immunité correspond à l'ensemble des couples \((pH,E)\) tels que le métal ne soit pas corrodé.

Le domaine de corrosion correspond à l'ensemble des couples \((pH,E)\) tels que le métal s'oxyde en ions métalliques.

Le domaine de passivité correspond à l'ensemble des couples \((pH,E)\) tels que le métal s'oxyde en oxyde métallique solide. Celui-ci protège alors le métal en formant un dépôt et empêche la destruction totale de la plaque métallique.


\end{document}
