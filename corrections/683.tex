\documentclass[a4paper]{article}
\usepackage[T1]{fontenc}
\usepackage[utf8]{inputenc}
\usepackage{lmodern}
\usepackage{amsmath,amssymb}
\usepackage[top=3cm,bottom=2cm,left=2cm,right=2cm]{geometry}
\usepackage{fancyhdr}
\usepackage{esvect}
\usepackage{xcolor}
\usepackage{tikz}\usetikzlibrary{calc}

\parskip 1em\parindent 0pt

\begin{document}

\pagestyle{fancy}
\fancyhf{}
\setlength{\headheight}{15pt}
\fancyhead[L]{Optique}\fancyhead[R]{Question 9}

% Énoncé
\begin{center}
	\large{\boldmath{\textbf{Critère de brouillage temporel et spatial}}}
\end{center}

% Correction
Brouillage spatial : \\
Soient $O$ le centre de la source et $B$ un point de la périphérie de la source.\\
$\Delta\delta=\delta_B(M)-\delta_O(M)\qquad\Delta p=\dfrac{\Delta\delta}{\lambda_0}$.\par
Brouillage temporel :\\
$p=\dfrac{\delta}{\lambda}=\sigma\delta$.
$\Delta p=p_{\nu_{\text{max}}}-p_{\nu_0}=\dfrac{\Delta\sigma}{2}\delta(M)$
\par

Dans les deux cas :\\
\fcolorbox{red}{white}{Il y a brouillage si \( |\Delta p| > \dfrac{1}{2} \)}

\end{document}
