\documentclass[a4paper]{article}
\usepackage[T1]{fontenc}
\usepackage[utf8]{inputenc}
\usepackage{lmodern}
\usepackage{amsmath,amssymb}
\usepackage[top=3cm,bottom=2cm,left=2cm,right=2cm]{geometry}
\usepackage{fancyhdr}
\usepackage{esvect,esint}
\usepackage{xcolor}
\usepackage{tikz}\usetikzlibrary{calc}

\parskip1em\parindent0pt\let\ds\displaystyle

\begin{document}

\pagestyle{fancy}
\fancyhf{}
\setlength{\headheight}{15pt}
\fancyhead[L]{Thermodynamique}\fancyhead[R]{Question 41}

% Énoncé
\begin{center}
	\large{\boldmath{\textbf{Loi de Newton pour  un échange conducto-convectif}}}
\end{center}

% Correction

Pour un échange conducto-convectif entre deux surfaces de températures différentes \(T_1\) et \(T_2\), le flux thermique est proportionnel à la différence de température.\\
On note \(h\) la constante de proportionnalité, appelée constante de Newton. On a :\begin{center}\fcolorbox{red}{white}{\(\phi=\pm hS(T_1-T_2)\)}\end{center}
Le signe dépend bien sûr de l'orientation du flux, celui-ci étant positif du chaud vers le froid.


\end{document}
