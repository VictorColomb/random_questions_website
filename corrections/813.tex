\documentclass[a4paper]{article}
\usepackage[T1]{fontenc}
\usepackage[utf8]{inputenc}
\usepackage{lmodern}
\usepackage{amsmath,amssymb}
\usepackage[top=3cm,bottom=2cm,left=2cm,right=2cm]{geometry}
\usepackage{fancyhdr}
\usepackage{esvect,esint}
\usepackage{xcolor}
\usepackage{tikz}\usetikzlibrary{calc}

\parskip1em\parindent0pt\let\ds\displaystyle

\begin{document}

\pagestyle{fancy}
\fancyhf{}
\setlength{\headheight}{15pt}
\fancyhead[L]{Mécanique quantique}\fancyhead[R]{Question 19}

% Énoncé
\begin{center}
	\large{\boldmath{\textbf{Niveaux d’énergie de l’atome d’hydrogène}}}
\end{center}

% Correction

On utilise la formule de Rydberg.\\
Soient \(n_1, n_2 \in \mathbb{N}^*\) tels que \(n_2 > n_1\),
\[\frac{1}{\lambda} = R_H\left(\frac{1}{n_1^2} - \frac{1}{n_2^2}\right)\]
\(E =h \nu\) donc\\
\begin{center}\fcolorbox{red}{white}{\(\Delta E = hcR_H\left(\dfrac{1}{n_1^2} - \dfrac{1}{n_2^2}\right) \)}\\
\fcolorbox{red}{white}{\( E_n = -hcR_H\dfrac{1}{n^2}\)}\end{center}
\emph{Rappel: }\(R_H = 1.097\cdot 10^7 m^{-1}\) est la constante de Rydberg pour l'hydrogène. À ne pas confondre avec \(R\) la constante universel des gaz parfaits.







\end{document}
