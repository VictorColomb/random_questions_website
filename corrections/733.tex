\documentclass[a4paper]{article}
\usepackage[T1]{fontenc}
\usepackage[utf8]{inputenc}
\usepackage{lmodern}
\usepackage{amsmath,amssymb}
\usepackage[top=3cm,bottom=2cm,left=2cm,right=2cm]{geometry}
\usepackage{fancyhdr}
\usepackage{esvect,esint}
\usepackage{xcolor}
\usepackage{tikz}\usetikzlibrary{calc}

\parskip1em\parindent0pt\let\ds\displaystyle

\begin{document}

\pagestyle{fancy}
\fancyhf{}
\setlength{\headheight}{15pt}
\fancyhead[L]{Thermodynamique}\fancyhead[R]{Question 43}

% Énoncé
\begin{center}
	\large{\boldmath{\textbf{Expression locale de l’équilibre hydrostatique}}}
\end{center}

% Correction

Soit un volume \(V\) de masse \(m\) défini au sein d'un fluide parfait et soumis aux forces de pression s'exerçant sur sa surface et au champ de pesanteur.\\
On a \(\vv{P}-\ds\iint\limits_{S^{\mathrm{int}}}P\vv{\mathrm{d}S}=\vv{0}\) soit \(\ds\iiint\limits_V\rho\vv{g}\mathrm{d}V-\ds\iiint\limits_V\vv{\mathrm{grad}}P\mathrm{d}V=\vv{0}\).\\
Comme ceci est vrai sur tout volume, c'est vrai localement et \begin{center}\fcolorbox{red}{white}{\(\rho\vv{g}=\vv{\mathrm{grad}}P\)}\end{center}


\end{document}
