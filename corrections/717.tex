\documentclass[a4paper]{article}
\usepackage[T1]{fontenc}
\usepackage[utf8]{inputenc}
\usepackage{lmodern}
\usepackage{amsmath,amssymb}
\usepackage[top=3cm,bottom=2cm,left=2cm,right=2cm]{geometry}
\usepackage{fancyhdr}
\usepackage{esvect}
\usepackage{xcolor}
\usepackage{tikz}\usetikzlibrary{calc}

\parskip 1em\parindent 0pt

\begin{document}

\pagestyle{fancy}
\fancyhf{}
\setlength{\headheight}{15pt}
\fancyhead[L]{Thermodynamique}\fancyhead[R]{Question 27}

% Énoncé
\begin{center}
	\large{\boldmath{\textbf{Rendement moteur ditherme, rendement de Carnot}}}
\end{center}

% Correction

Pour un cycle dont on note \(W\) le travail reçu (négatif car on s'intéresse à un moteur) et \(Q_c,Q_f\) les transferts thermiques chaud et froid, le rendement est défini par :\begin{center}
\fcolorbox{red}{white}{\(\rho = - \dfrac{W}{Q_c}\)}\end{center}
Il est majoré par le rendement de Carnot :\begin{center}\fcolorbox{red}{white}{\(\rho_c=1-\dfrac{T_f}{T_c}\)}\end{center}


\end{document}
