\documentclass[a4paper]{article}
\usepackage[T1]{fontenc}
\usepackage[utf8]{inputenc}
\usepackage{lmodern}
\usepackage{amsmath,amssymb}
\usepackage[top=3cm,bottom=2cm,left=2cm,right=2cm]{geometry}
\usepackage{fancyhdr}
\usepackage{esvect}
\usepackage{xcolor}
\usepackage{tikz}\usetikzlibrary{calc}

\parskip 1em\parindent 0pt

\begin{document}

\pagestyle{fancy}
\fancyhf{}
\setlength{\headheight}{15pt}
\fancyhead[L]{Thermodynamique}\fancyhead[R]{Question 21}

% Énoncé
\begin{center}
	\large{\boldmath{\textbf{Valeur de d$S$ pour une transformation quasi-statique \\ mécaniquement réversible}}}
\end{center}

% Correction

Pour une transformation quasi-statique mécaniquement réversible, \fcolorbox{red}{white}{\(\mathrm{d}S=\dfrac{\delta Q}{T}\)}

\end{document}
