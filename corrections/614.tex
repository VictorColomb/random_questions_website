\documentclass[a4paper]{article}
\usepackage[T1]{fontenc}
\usepackage[utf8]{inputenc}
\usepackage{lmodern}
\usepackage{amsmath,amssymb}
\usepackage[top=3cm,bottom=2cm,left=2cm,right=2cm]{geometry}
\usepackage{fancyhdr}
\usepackage{esvect,esint}
\usepackage{xcolor}
\usepackage{tikz}\usetikzlibrary{calc}

\parskip1em\parindent0pt\let\ds\displaystyle

\begin{document}

\pagestyle{fancy}
\fancyhf{}
\setlength{\headheight}{15pt}
\fancyhead[L]{Mécanique}\fancyhead[R]{Question 27}

% Énoncé
\begin{center}
	\large{\boldmath{\textbf{Bras de levier}}}
\end{center}

% Correction

  \begin{minipage}{0.5\linewidth}
    Soit une force $\vv{F}$ appliquée à $A$\\
    $\vv{M}(O,\vv{F})=\vv{OA}\wedge\vv{F}$\\
    $\vv{M}(O,\vv{F})\cdot\vv{z}=-OH\,||\vv{F}||$
  \end{minipage}
  \begin{minipage}{0.5\linewidth}
    \centering
    \begin{tikzpicture}[scale=1.5]
      \draw (0,0.5) node[below]{$O$} -- (0,1.5) node[above]{$H$} to  ++(1.5,0) node[circle,fill,inner sep=1.5pt]{} node[above]{$A$}; \draw[->] (1.5,1.5) -- ++(1,0) node[above right]{$\vv{F}$};
      \draw[thick] (-0.5,1.8) arc (0:270:0.2);\draw (-0.5,2) node[above left]{$\theta$};
      \draw[->] (-0.5,1.8) arc (0:270:0.2);\draw (-0.5,2) node[above left]{$\theta$};
      \node[draw, circle,inner sep=2pt] at (-0.5, 1){.};
      \draw (-0.5,1) node[right]{$\vv z$}
      \draw (0,1.3) -- ++(.2,0) -- ++(0,.2);
      
    \end{tikzpicture}
  \end{minipage}





\end{document}
