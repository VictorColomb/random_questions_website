\documentclass[a4paper]{article}
\usepackage[T1]{fontenc}
\usepackage[utf8]{inputenc}
\usepackage{lmodern}
\usepackage{amsmath,amssymb}
\usepackage[top=3cm,bottom=2cm,left=2cm,right=2cm]{geometry}
\usepackage{fancyhdr}
\usepackage{esvect,esint}
\usepackage{xcolor}
\usepackage{tikz}\usetikzlibrary{calc}

\parskip1em\parindent0pt\let\ds\displaystyle

\begin{document}

\pagestyle{fancy}
\fancyhf{}
\setlength{\headheight}{15pt}
\fancyhead[L]{Mécanique}\fancyhead[R]{Question 13}

% Énoncé
\begin{center}
	\large{\boldmath{\textbf{Principe fondamental de la dynamique dans un référentiel non galiléen}}}
\end{center}

% Correction

Soit \(\mathcal{R}_g\) un référentiel galiléen et \(\mathcal{R}\) un référentiel quelconque.\\
Soit un corps ponctuel de masse \(m\) situé en un point \(M\).\\
L'application du principe fondamental de la dynamique dans \(\mathcal{R}\) donne :
\begin{center}\fcolorbox{red}{white}{\(m\vv{a}(M\in\mathcal{R})=\sum\vv{F}_{\mathrm{ext}}+\vv{F_{\mathrm{ent}}}+\vv{F_{\mathrm{cor}}}\)}\end{center}où les \(\vv{F_{\mathrm{ext}}}\) sont les forces extérieures appliquées à \(M\) et où \(\vv{F_{\mathrm{ent}}}\) et \(\vv{F_{\mathrm{cor}}}\) sont des forces appelées forces de repère vérifiant :
\begin{center}\(\vv{F_{\mathrm{ent}}}=-m\vv{a_{\mathrm{ent}}}\) et \(\vv{F_{\mathrm{cor}}}=-m\vv{a_{\mathrm{cor}}}\)\end{center}

\emph{Rappel} :
\[\vv a_{\mathrm{ent}} = \vv \Omega \wedge ( \vv \Omega \wedge \vv{O_2M}) + \left. \frac{\mathrm{d} \vv \Omega}{\mathrm{d}t}\right|_{\mathcal R_g} \wedge \vv{O_2 M} + \vv a (O_2 \in \mathcal R_g)\]
\[\vv a_{\mathrm{cor}} = 2 \vv \Omega \wedge \vv v(M \in \mathcal R)\]


\end{document}
