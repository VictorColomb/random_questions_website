\documentclass[a4paper]{article}
\usepackage[T1]{fontenc}
\usepackage[utf8]{inputenc}
\usepackage{lmodern}
\usepackage{amsmath,amssymb}
\usepackage[top=3cm,bottom=2cm,left=2cm,right=2cm]{geometry}
\usepackage{fancyhdr}
\usepackage{esvect}
\usepackage{xcolor}
\usepackage{tikz}\usetikzlibrary{calc}

\parskip 1em\parindent 0pt

\begin{document}

\pagestyle{fancy}
\fancyhf{}
\setlength{\headheight}{15pt}
\fancyhead[L]{Mécanique}\fancyhead[R]{Question 7}

% Énoncé
\begin{center}
	\large{\boldmath{\textbf{Définition des vecteurs vitesse et accélération dans un repère}}}
\end{center}

% Correction

Si \(\mathcal{R}\) est un repère de centre \(O\) et \(M\) un point matériel, on définit:\\
\begin{itemize}
\item
Sa vitesse \fcolorbox{red}{white}{\(\vv{v}(M\in\mathcal{R})=\left.\dfrac{\mathrm{d}\vv{OM}}{\mathrm{d}t}\right|_{\mathcal{R}}\)}
\item
Son accélération \fcolorbox{red}{white}{\(\vv{a}(M\in\mathcal{R})=\left.\dfrac{\mathrm{d}^2\vv{OM}}{\mathrm{d}t^2}\right|_{\mathcal{R}}\)}

\end{document}
