\documentclass[a4paper]{article}
\usepackage[T1]{fontenc}
\usepackage[utf8]{inputenc}
\usepackage{lmodern}
\usepackage{amsmath,amssymb}
\usepackage[top=3cm,bottom=2cm,left=2cm,right=2cm]{geometry}
\usepackage{fancyhdr}
\usepackage{esvect,esint}
\usepackage{xcolor}
\usepackage{tikz}\usetikzlibrary{calc}

\parskip1em\parindent0pt\let\ds\displaystyle

\begin{document}

\pagestyle{fancy}
\fancyhf{}
\setlength{\headheight}{15pt}
\fancyhead[L]{Thermodynamique}\fancyhead[R]{Question 10}

% Énoncé
\begin{center}
	\large{\boldmath{\textbf{Définition des capacités calorifiques à volume constant et à pression constante \\ pour un gaz parfait}}}
\end{center}

% Correction

Pour un gaz parfait, on définit la capacité thermique à volume constant par :\begin{center}\fcolorbox{red}{white}{\(C_v=\left.\dfrac{\partial U}{\partial T}\right|_V\)}\end{center}
On définit de même la capacité thermique à pression constante par :\begin{center}\fcolorbox{red}{white}{\(C_p=\left.\dfrac{\partial H}{\partial T}\right|_P\)}\end{center}
Ces deux grandeurs sont très peu dépendantes de la température dans les conditions de pression et de température usuelles.

\end{document}
